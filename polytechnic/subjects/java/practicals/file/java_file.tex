 % Choices environment

\documentclass{exam}
% \documentclass[a4paper,12pt, openany]{exam}


\usepackage{amssymb}
\usepackage{geometry}
\usepackage{comment}
\usepackage{listings}
\usepackage{import}
\usepackage{graphicx}

\geometry{
            a4paper,
            total={170mm,257mm},
            left=17mm,
            top=15mm,
 }

\linespread{1.3}

\renewcommand{\rmdefault}{phv}
\renewcommand{\sfdefault}{phv}

\newenvironment{questyle}
    {
    \begin{minipage}{\linewidth}
    }
    {

    \end{minipage}
    \vspace{0.08in}
    }

\newenvironment{myTableStyle}
    {
    \begingroup
    \setlength{\tabcolsep}{10pt} % Default value: 6pt
    \renewcommand{\arraystretch}{1.5} % Default value: 1
    }
    {

    \endgroup
    }


\begin{document}


  \begin{enumerate}
   \item Program to show  use of variable and data-types.

   \begin{myTableStyle}
   \begin{center} \begin{tabular}{ |m{16cm}| } \hline
             \lstinputlisting[language=java, basicstyle=\normalsize]{../MainVar.java} \\ \hline
    \end{tabular} \end{center}
  \end{myTableStyle}
  \pagebreak

   \item  Program for  operators.

   \begin{myTableStyle}
   \begin{center} \begin{tabular}{ |m{14cm}| } \hline
             \lstinputlisting[language=C, basicstyle=\normalsize]{../Operator.java} \\ \hline
    \end{tabular} \end{center}
\end{myTableStyle}
  \pagebreak
  \pagebreak

   \item  Program to determine even odd numbers.

   \begin{myTableStyle}
   \begin{center} \begin{tabular}{ |m{14cm}| } \hline
             \lstinputlisting[language=C, basicstyle=\normalsize]{../EvenOdd.java} \\ \hline
    \end{tabular} \end{center}
\end{myTableStyle}
  \pagebreak

   \item  Program to find maximum of 3  numbers.

   \begin{myTableStyle}
   \begin{center} \begin{tabular}{ |m{14cm}| } \hline
             \lstinputlisting[language=C, basicstyle=\normalsize]{../Max.java} \\ \hline
    \end{tabular} \end{center}
\end{myTableStyle}
  \pagebreak

   \item  Program  using loops.

   \begin{myTableStyle}
   \begin{center} \begin{tabular}{ |m{14cm}| } \hline
             \lstinputlisting[language=C, basicstyle=\normalsize]{../Loops.java} \\ \hline \end{tabular} \end{center}
\end{myTableStyle}
  \pagebreak

   \item  Program  using break and continue.

   \begin{myTableStyle}
   \begin{center} \begin{tabular}{ |m{14cm}| } \hline
             \lstinputlisting[language=C, basicstyle=\normalsize]{../Loop.java} \\ \hline
    \end{tabular} \end{center}
\end{myTableStyle}
  \pagebreak

   \item  Program  to print table of  number.

   \begin{myTableStyle}
   \begin{center} \begin{tabular}{ |m{14cm}| } \hline
             \lstinputlisting[language=C, basicstyle=\normalsize]{../Table.java} \\ \hline
    \end{tabular} \end{center}
\end{myTableStyle}
  \pagebreak

   \item  Program using arrays.

   \begin{myTableStyle}
   \begin{center} \begin{tabular}{ |m{14cm}| } \hline
             \lstinputlisting[language=C, basicstyle=\normalsize]{../Arrays.java} \\ \hline
    \end{tabular} \end{center}
\end{myTableStyle}
  \pagebreak

   \item  Program using overloading.

   \begin{myTableStyle}
   \begin{center} \begin{tabular}{ |m{14cm}| } \hline
             \lstinputlisting[language=C, basicstyle=\normalsize]{../Overloading.java} \\ \hline
    \end{tabular} \end{center}
\end{myTableStyle}
  \pagebreak

   \item  Program using inheritance.

   \begin{myTableStyle}
   \begin{center} \begin{tabular}{ |m{14cm}| } \hline
             \lstinputlisting[language=C, basicstyle=\normalsize]{../Child.java} \\ \hline
    \end{tabular} \end{center}
\end{myTableStyle}
  \pagebreak

   \item  Program using interface.

   \begin{myTableStyle}
   \begin{center} \begin{tabular}{ |m{14cm}| } \hline
             \lstinputlisting[language=C, basicstyle=\normalsize]{../MainInterface.java} \\ \hline
    \end{tabular} \end{center}
\end{myTableStyle}
  \pagebreak

   \item  Program using over-riding.

   \begin{myTableStyle}
   \begin{center} \begin{tabular}{ |m{14cm}| } \hline
             \lstinputlisting[language=C, basicstyle=\normalsize]{../Overriding.java} \\ \hline
    \end{tabular} \end{center}
\end{myTableStyle}
  \pagebreak

   \item  Program using  polymorphism.

   \begin{myTableStyle}
   \begin{center} \begin{tabular}{ |m{14cm}| } \hline
             \lstinputlisting[language=C, basicstyle=\normalsize]{../Polym.java} \\ \hline
    \end{tabular} \end{center}
\end{myTableStyle}
  \pagebreak

   \item  Program using  threads.

   \begin{myTableStyle}
   \begin{center} \begin{tabular}{ |m{14cm}| } \hline
             \lstinputlisting[language=C, basicstyle=\normalsize]{../ThreadUser.java} \\ \hline
    \end{tabular} \end{center}
\end{myTableStyle}
  \pagebreak

   \item  Program for input and output.

   \begin{myTableStyle}
   \begin{center} \begin{tabular}{ |m{14cm}| } \hline
             \lstinputlisting[language=C, basicstyle=\normalsize]{../InputReader.java} \\ \hline
    \end{tabular} \end{center}
\end{myTableStyle}
  \pagebreak

   \item  Program to write a file.

   \begin{myTableStyle}
   \begin{center} \begin{tabular}{ |m{14cm}| } \hline
             \lstinputlisting[language=C, basicstyle=\normalsize]{../FWriter.java} \\ \hline
    \end{tabular} \end{center}
\end{myTableStyle}
  \pagebreak

   \item  Program to read a  file.

   \begin{myTableStyle}
   \begin{center} \begin{tabular}{ |m{16cm}| } \hline
             \lstinputlisting[language=C, basicstyle=\small]{../FReader.java} \\ \hline
    \end{tabular} \end{center}
\end{myTableStyle}
  \pagebreak

   \item  Program using exception handline.

   \begin{myTableStyle}
   \begin{center} \begin{tabular}{ |m{14cm}| } \hline
             \lstinputlisting[language=C, basicstyle=\normalsize]{../ExceptionTest.java} \\ \hline
    \end{tabular} \end{center}
\end{myTableStyle}






  \end{enumerate}






\end{document}
