 % Choices environment

\documentclass{exam}
% \documentclass[a4paper,12pt, openany]{exam}


\usepackage{amssymb}
\usepackage{geometry}
\usepackage{comment}
\usepackage{listings}
\usepackage{import}
\usepackage{graphicx}

\geometry{
            a4paper,
            total={170mm,257mm},
            left=17mm,
            top=15mm,
 }

\linespread{1.3}

\renewcommand{\rmdefault}{phv}
\renewcommand{\sfdefault}{phv}

\newenvironment{questyle}
    {
    \begin{minipage}{\linewidth}
    }
    {

    \end{minipage}
    \vspace{0.08in}
    }

\newenvironment{myTableStyle}
    {
    \begingroup
    \setlength{\tabcolsep}{10pt} % Default value: 6pt
    \renewcommand{\arraystretch}{1.5} % Default value: 1
    }
    {

    \endgroup
    }


\begin{document}


  \begin{enumerate}
   \item Program to show use of variable and data-types.

   \begin{myTableStyle}
   \begin{center} \begin{tabular}{ |m{14cm}| } \hline
             \lstinputlisting[language=C, basicstyle=\normalsize]{../dataypes_variables.c} \\ \hline
    \end{tabular} \end{center}
\end{myTableStyle}
  \pagebreak

   \item  Program for arithmatic operators.

   \begin{myTableStyle}
   \begin{center} \begin{tabular}{ |m{14cm}| } \hline
             \lstinputlisting[language=C, basicstyle=\normalsize]{../opertor_arithmatic.c} \\ \hline
    \end{tabular} \end{center}
\end{myTableStyle}
  \pagebreak

   \item  Program for comparative operators.

   \begin{myTableStyle}
   \begin{center} \begin{tabular}{ |m{14cm}| } \hline
             \lstinputlisting[language=C, basicstyle=\normalsize]{../opertor_comp.c} \\ \hline
    \end{tabular} \end{center}
\end{myTableStyle}
  \pagebreak

   \item  Program for logical operators.

   \begin{myTableStyle}
   \begin{center} \begin{tabular}{ |m{14cm}| } \hline
             \lstinputlisting[language=C, basicstyle=\normalsize]{../opertor_log.c} \\ \hline
    \end{tabular} \end{center}
\end{myTableStyle}
  \pagebreak

   \item  Program for bitwise operators.

   \begin{myTableStyle}
   \begin{center} \begin{tabular}{ |m{14cm}| } \hline
             \lstinputlisting[language=C, basicstyle=\normalsize]{../opertor_bit.c} \\ \hline
    \end{tabular} \end{center}
\end{myTableStyle}
  \pagebreak

   \item  Program for bitwise operators.

   \begin{myTableStyle}
   \begin{center} \begin{tabular}{ |m{14cm}| } \hline
             \lstinputlisting[language=C, basicstyle=\normalsize]{../opertor_bit.c} \\ \hline
    \end{tabular} \end{center}
\end{myTableStyle}
  \pagebreak

  \item  zzz
  \end{enumerate}






\end{document}
