 % Choices environment

\documentclass{exam}
% \documentclass[a4paper,12pt, openany]{exam}


\usepackage{amssymb}
\usepackage{geometry}
\usepackage{comment}
\usepackage{listings}
\usepackage{import}
\usepackage{graphicx}
\usepackage{array}  % for table and column width
\usepackage{tabularray} % for table and column width
\usepackage{tabularx}
% \usepackage{minted}     % for code in programming  -shell-escape
\usepackage{hyperref}


\usepackage{xcolor}
%New colors defined below
\definecolor{codegreen}{rgb}{0,0.6,0}
\definecolor{codegray}{rgb}{0.5,0.5,0.5}
\definecolor{codepurple}{rgb}{0.58,0,0.82}
\definecolor{backcolour}{rgb}{0.95,0.95,0.92}

%Code listing style named "mystyle"
\lstdefinestyle{mystyle}{
  % backgroundcolor=\color{backcolour},
  commentstyle=\color{codegreen},
  keywordstyle=\color{magenta},
  numberstyle=\tiny\color{codegray},
  stringstyle=\color{codepurple},
  basicstyle=\ttfamily\footnotesize,
  breakatwhitespace=false,
  breaklines=true,
  captionpos=b,
  keepspaces=true,
  %numbers=left,
  numbersep=5pt,
  showspaces=false,
  showstringspaces=false,
  showtabs=false,
  tabsize=2
}
\lstset{style=mystyle}

\hypersetup{
    colorlinks=true,
    linkcolor=blue,
    filecolor=magenta,
    urlcolor=cyan,
    pdftitle={Overleaf Example},
    pdfpagemode=FullScreen,
    }

\geometry{
            a4paper,
            total={170mm,257mm},
            left=17mm,
            top=15mm,
 }

\linespread{1.3}

\renewcommand{\rmdefault}{phv}
\renewcommand{\sfdefault}{phv}

\newenvironment{questyle}
    {
    \begin{minipage}{\linewidth}
    }
    {

    \end{minipage}
    \vspace{0.08in}
    }


\newcolumntype{M}[1]{>{\centering\arraybackslash}m{#1}}     % custom column : fixed width and centering

% The \begingroup ... \endgroup pair ensures the separation
% parameters only affect this particular table, and not any
% sebsequent ones in the document.
\newenvironment{myTableStyle}
    {
    \begingroup
    \setlength{\tabcolsep}{10pt} % Default value: 6pt
    \renewcommand{\arraystretch}{1.5} % Default value: 1
    }
    {
    \endgroup
    }

\newenvironment{myTableCodeStyle}
    {
    \begingroup
    \setlength{\tabcolsep}{10pt} % Default value: 6pt
    \renewcommand{\arraystretch}{0.5} % Default value: 1
    }
    {
    \endgroup
    }


\begin{document}


  \begin{enumerate}
   \item Program to show use of variable and data-types.

   \begin{myTableStyle}
   \begin{center} \begin{tabular}{ |m{14cm}| } \hline
             \lstinputlisting[language=C, basicstyle=\normalsize]{../dataypes_variables.c} \\ \hline
    \end{tabular} \end{center}
\end{myTableStyle}
  \pagebreak

   \item  Program for arithmatic operators.

   \begin{myTableStyle}
   \begin{center} \begin{tabular}{ |m{14cm}| } \hline
             \lstinputlisting[language=C, basicstyle=\normalsize]{../opertor_arithmatic.c} \\ \hline
    \end{tabular} \end{center}
\end{myTableStyle}
  \pagebreak

   \item  Program for comparative operators.

   \begin{myTableStyle}
   \begin{center} \begin{tabular}{ |m{14cm}| } \hline
             \lstinputlisting[language=C, basicstyle=\normalsize]{../opertor_comp.c} \\ \hline
    \end{tabular} \end{center}
\end{myTableStyle}
  \pagebreak

   \item  Program for logical operators.

   \begin{myTableStyle}
   \begin{center} \begin{tabular}{ |m{14cm}| } \hline
             \lstinputlisting[language=C, basicstyle=\normalsize]{../opertor_log.c} \\ \hline
    \end{tabular} \end{center}
\end{myTableStyle}
  \pagebreak

   \item  Program for bitwise operators.

   \begin{myTableStyle}
   \begin{center} \begin{tabular}{ |m{14cm}| } \hline
             \lstinputlisting[language=C, basicstyle=\normalsize]{../opertor_bit.c} \\ \hline
    \end{tabular} \end{center}
\end{myTableStyle}
  \pagebreak

   \item  Program for bitwise operators.

   \begin{myTableStyle}
   \begin{center} \begin{tabular}{ |m{14cm}| } \hline
             \lstinputlisting[language=C, basicstyle=\normalsize]{../opertor_bit.c} \\ \hline
    \end{tabular} \end{center}
\end{myTableStyle}
  \pagebreak

  \item  zzz
  \end{enumerate}






\end{document}
