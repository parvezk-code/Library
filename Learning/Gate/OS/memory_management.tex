\centerline{\textbf{ \LARGE Memory Management Basics}}




% ----------------------------------------------------------------------------

\begin{questyle}

  \question  The principle of locality justifies the use of:(GATE-1995)

  \begin{oneparchoices}
    \choice Interrupts
    \choice DMA
    \choice Polling
    \choice Cache Memory
  \end{oneparchoices}
\end{questyle}

% ----------------------------------------------------------------------------

\begin{questyle}

  \question  In a paged segmented scheme of memory management, the segment table itself must have
             a page table because (GATE-1995)

  \begin{choices}
    \choice The segment table is often too large to fit in one page
    \choice Each segment is spread over a number of pages
    \choice Segment tables point to page tables and not to the physical locations of the segment
    \choice The processor’s description base register points to a page table
  \end{choices}
\end{questyle}

% ----------------------------------------------------------------------------

\begin{questyle}

  \question  A linker is given object modules for a set of programs that were compiled separately. What information need to be included in an object module? (GATE-1995)

  \begin{choices}
    \choice Object code
    \choice Relocation bits
    \choice Names and locations of all external symbols defined in the object module
    \choice Absolute addresses of internal symbols
  \end{choices}
\end{questyle}

% ----------------------------------------------------------------------------

\begin{questyle}

  \question Locality of reference implies that the page reference being made by a process. (GATE-1997)

  \begin{choices}
    \choice will always be to the page used in the previous page reference
    \choice is likely to be to one of the pages used in the last few page references
    \choice will always be to one of the pages existing in memory
    \choice will always lead to a page fault
  \end{choices}
\end{questyle}

% ----------------------------------------------------------------------------

\begin{questyle}

  \question Thrashing. (GATE-1997)

  \begin{choices}
    \choice reduces page I/O
    \choice decreases the degree of multiprogramming
    \choice implies excessive page I/O
    \choice improve the system performance
  \end{choices}
\end{questyle}

% ----------------------------------------------------------------------------

\begin{questyle}

  \question In a resident – OS computer, which of the following systems must reside in the main memory under all situations?  (GATE-1998)

  \begin{oneparchoices}
    \choice Assembler
    \choice Linker
    \choice Loader
    \choice Compiler
  \end{oneparchoices}

\end{questyle}


% ----------------------------------------------------------------------------

\begin{questyle}

  \question  Which of the following statements is false? (GATE-2001)

  \begin{choices}
    \choice Virtual memory implements the translation of a program‘s address space into physical memory address space
    \choice Virtual memory allows each program to exceed the size of the primary memory
    \choice Virtual memory increases the degree of multiprogramming
    \choice Virtual memory reduces the context switching overhead
  \end{choices}

\end{questyle}

% ----------------------------------------------------------------------------

\begin{questyle}

  \question  The process of assigning load addresses to the various parts of the program and adjusting the code and date in the program to reflect the assigned addresses is called (GATE-2001)

  \begin{oneparchoices}
    \choice Assembly
    \choice Parsing
    \choice Relocation
    \choice Symbol resolution
  \end{oneparchoices}

\end{questyle}

% ----------------------------------------------------------------------------

\begin{questyle}

  \question  Consider a program P that consists of two source modules M1 and M2 contained in two different files. If M1 contains a reference to a function defined in M2, the reference will be resolved at (GATE-2004)

  \begin{oneparchoices}
    \choice Edit-time
    \choice Compile-time
    \choice Link-time
    \choice Load-time
  \end{oneparchoices}

\end{questyle}

% ----------------------------------------------------------------------------

\begin{questyle}

  \question  What is the swap space in the disk used for? (GATE-2005)

  \begin{choices}
    \choice Saving temporary html pages
    \choice Saving process data
    \choice Storing the super-block
    \choice Storing device drivers
  \end{choices}

\end{questyle}

% ----------------------------------------------------------------------------

\begin{questyle}

  \question Match the pairs in the following: (GATE-1989)

  \begin{lstlisting}
            List-I                        List-II
      a). Virtual memory          p). Temporal Locality
      b). Shared memory           q). Spatial Locality
      c). Look-ahead buffer       r). Address Translation
      d). Look-aside buffer       s). Mutual Exclusion

  \end{lstlisting}


\end{questyle}

% ----------------------------------------------------------------------------

\begin{questyle}

  \question Match the pairs in the following: (GATE-1990)

  \begin{lstlisting}
            List-I                        List-II
      a). Critical region           p). Hoare's monitor
      b). Wait/Signal               q). Mutual exclusion
      c). Working Set               r). Principle of locality
      d). Deadlock                  s). Circular Wait

  \end{lstlisting}

\end{questyle}

% ----------------------------------------------------------------------------

\begin{questyle}

  \question  The Link-load-and-go loading scheme required less storage space than the link-and-go loading scheme. True/False (GATE-1990)

\end{questyle}

% ----------------------------------------------------------------------------

\begin{questyle}

  \question Match the pairs in the following: (GATE-1991)

  \begin{lstlisting}
            List-I                        List-II
      a). Buddy system              p). Run time type specification
      b). Interpretation            q). Segmentation
      c). Pointer type              r). Memory allocation
      d). Virtual memory            s). Garbage collection

  \end{lstlisting}

\end{questyle}

% ----------------------------------------------------------------------------

\begin{questyle}

  \question  The total size of address space in a virtual memory system is limited by: (GATE-1991)

  \begin{choices}
    \choice the length of MAR
    \choice the available secondary storage
    \choice the available main memory
    \choice all of the above
  \end{choices}

\end{questyle}

% ----------------------------------------------------------------------------

\begin{questyle}

  \question  A "link editor" is a program that: (GATE-1991)

  \begin{choices}
    \choice matches the parameters of the macro-definition with locations of the parameters of the macro call
    \choice matches external names of one program with their location in other programs
    \choice matches the parameters of subroutine definition with the location of parameters of subroutine call.
    \choice acts as a link between text editor and the user
    \choice acts as a link between compiler and the user program
  \end{choices}

\end{questyle}

% ----------------------------------------------------------------------------

\begin{questyle}

  \question  State true/False (GATE-1991)

  \begin{choices}
    \choice The amount of virtual memory available is limited by the availability of the secondary memory
    \choice Any implementation of a critical section requires the use of an indivisible machine- instruction ,such as test-and-set.
    \choice The use of monitors ensure that no dead-locks will be caused .
    \choice The LRU page-replacement policy may cause thrashing for some type of programs
    \choice The best fit techniques for memory allocation ensures that memory will never be fragmented.
  \end{choices}

\end{questyle}

% ----------------------------------------------------------------------------

\begin{questyle}

  \question A simple two-pass assembler does the following in the first pass: (GATE-1993)

  \begin{choices}
    \choice It allocates space for the literals.
    \choice It computes the total length of the program.
    \choice It builds the symbol table for the symbols and their values.
    \choice It generates code for all the load and store register instructions.
    \choice None of the above.
  \end{choices}

\end{questyle}

% ----------------------------------------------------------------------------

\begin{questyle}

  \question  A part of the system software which under all circumstances must reside in the main memory is: (GATE-1993)

  \begin{choices}
    \choice text editor
    \choice assembler
    \choice linker
    \choice loader
    \choice none of the above
  \end{choices}

\end{questyle}


% ----------------------------------------------------------------------------

\begin{questyle}

  \question Consider the following heap (figure) in which blank regions are not in use and hatched region are in use.
            The sequence of requests for blocks of sizes 300, 25, 125, 50 can be satisfied if we use. (GATE-1994)
  \begin{lstlisting}
        050     150     300     350     600
      _________________________________________
      |       |       |       |       |       |
      | full  |       | full  |       | full  |
      |       |       |       |       |       |
      |______ |_______|_______|_______|_______|
  \end{lstlisting}

  \begin{choices}
    \choice either first fit or best fit policy (any one)
    \choice first fit but not best fit policy
    \choice best fit but not first fit policy
    \choice None of the above
  \end{choices}

\end{questyle}

% ----------------------------------------------------------------------------

\begin{questyle}

  \question  The capacity of a memory unit is defined by the number of words multiplied by the number of bits/word. How many separate address and data lines are needed for a memory of \( \large 4K \times 16 \)(GATE-1995)

  \begin{choices}
    \choice 10 address, 16 data lines
    \choice 11 address, 8 data lines
    \choice 12 address, 16 data lines
    \choice 12 address, 12 data lines
  \end{choices}

\end{questyle}


% ----------------------------------------------------------------------------

\begin{questyle}

  \question  In a virtual memory system the address space specified by the address lines of the CPU must be \fillin[]
            than the physical memory size and \fillin[] than the secondary storage size (GATE-1995)

  \begin{choices}
    \choice smaller, smaller
    \choice smaller, larger
    \choice larger, smaller
    \choice larger, larger
  \end{choices}

\end{questyle}

% ----------------------------------------------------------------------------

\begin{questyle}

  \question  A 1000 Kbyte memory is managed using variable partitions but no compaction. It currently has two
            partitions of sizes 200 Kbytes and 260 Kbytes respectively. The smallest allocation request in Kbytes
            that could be denied is for (GATE-1996)

  \begin{oneparchoices}
    \choice 151
    \choice 181
    \choice 231
    \choice 541
  \end{oneparchoices}

\end{questyle}

% ----------------------------------------------------------------------------

\begin{questyle}

  \question  The overlay tree for a program is as shown below. What will be the size of the partition (in physical memory)
             required to load (and run) this program?  (GATE-1998)
             \begin{lstlisting}
                               ___________________Root________________
                              |                    |                  |
                          4KB [A]              6KB [B]           8KB [C]
                          _____|_____              |                  |
                         |          |              |                  |
                    6KB [D]    8KB [E]        2KB [F]            4KB [G]
              \end{lstlisting}
  \begin{choices}
    \choice 12 KB
    \choice 14 KB
    \choice 10 KB
    \choice 8 KB
  \end{choices}

\end{questyle}

% ----------------------------------------------------------------------------

\begin{questyle}

  \question  Which of the following is/are advantage(s) of virtual memory? (GATE-1999)

  \begin{choices}
    \choice Faster access to memory on an average.
    \choice Processes can be given protected address spaces.
    \choice Linker can assign addresses independent of where the program will be loaded in physical memory.
    \choice Program larger than the physical memory size can be run.
  \end{choices}

\end{questyle}

% ----------------------------------------------------------------------------

\begin{questyle}

  \question  Dynamic linking can cause security concerns because (GATE- 2002)

  \begin{choices}
    \choice Security is dynamic
    \choice The path for searching dynamic libraries is not known till runtime
    \choice Linking is insecure
    \choice Cryptographic procedures are not available for dynamic linking
  \end{choices}

\end{questyle}

% ----------------------------------------------------------------------------

\begin{questyle}

  \question  Which of the following is NOT an advantage of using shared, dynamically linked libraries as opposed to
              using statistically linked libraries? (GATE-2003)

  \begin{choices}
    \choice Smaller sizes of executable files
    \choice Lesser overall page fault rate in the system
    \choice Faster program startup
    \choice Existing programs need not be re-linked to take advantage of newer versions of libraries
  \end{choices}

\end{questyle}

% ----------------------------------------------------------------------------

\begin{questyle}

  \question  A computer system supports 32-bit virtual addresses as well as 32-bit physical addresses.
             Since the virtual address space is of the same size as the physical address space, the operating system
             designers decide to get rid of the virtual memory entirely. Which one of the following is true? (GATE-2006)

  \begin{choices}
    \choice Efficient implementation of multi-user support is no longer possible
    \choice The processor cache organization can be made more efficient now
    \choice Hardware support for memory management is no longer needed
    \choice CPU scheduling can be made more efficient now
  \end{choices}

\end{questyle}


% ----------------------------------------------------------------------------

\begin{questyle}

  \question  Consider six memory partitions of size 200 KB, 400 KB, 600 KB, 500 KB, 300 KB, and 250 KB,
            where KB refers to kilobyte. These partitions need to be allotted to four processes of
            sizes 357 KB, 210 KB, 468 KB and 491 KB in that order. If the best fit algorithm is used,
            which partitions are NOT allotted to any process? (GATE-2015\_set\_2)

  \begin{choices}
    \choice 200 KB and 300 KB
    \choice 200 KB and 250 KB
    \choice 250 KB and 300 KB
    \choice 300 KB and 400 KB
  \end{choices}

\end{questyle}


% ----------------------------------------------------------------------------

% ----------------------------------------------------------------------------

% ----------------------------------------------------------------------------

% ----------------------------------------------------------------------------

% ----------------------------------------------------------------------------

% ----------------------------------------------------------------------------

% ----------------------------------------------------------------------------

% ----------------------------------------------------------------------------

% ----------------------------------------------------------------------------

% ----------------------------------------------------------------------------

% ----------------------------------------------------------------------------

% ----------------------------------------------------------------------------

% ----------------------------------------------------------------------------

% ----------------------------------------------------------------------------

% ----------------------------------------------------------------------------

% ----------------------------------------------------------------------------

% ----------------------------------------------------------------------------

% ----------------------------------------------------------------------------

% ----------------------------------------------------------------------------

% ----------------------------------------------------------------------------

% ----------------------------------------------------------------------------

% ----------------------------------------------------------------------------

% ----------------------------------------------------------------------------

% ----------------------------------------------------------------------------

% ----------------------------------------------------------------------------

% ----------------------------------------------------------------------------

% ----------------------------------------------------------------------------

% ----------------------------------------------------------------------------

% ----------------------------------------------------------------------------

% ----------------------------------------------------------------------------

% ----------------------------------------------------------------------------

% ----------------------------------------------------------------------------

% ----------------------------------------------------------------------------

% ----------------------------------------------------------------------------

% ----------------------------------------------------------------------------

% ----------------------------------------------------------------------------

% ----------------------------------------------------------------------------

% ----------------------------------------------------------------------------

% ----------------------------------------------------------------------------

% ----------------------------------------------------------------------------

% ----------------------------------------------------------------------------

% ----------------------------------------------------------------------------

% ----------------------------------------------------------------------------

% ----------------------------------------------------------------------------

% ----------------------------------------------------------------------------

% ----------------------------------------------------------------------------

% ----------------------------------------------------------------------------

% ----------------------------------------------------------------------------

% ----------------------------------------------------------------------------

% ----------------------------------------------------------------------------

% ----------------------------------------------------------------------------

% ----------------------------------------------------------------------------

% ----------------------------------------------------------------------------

% ----------------------------------------------------------------------------

% ----------------------------------------------------------------------------
