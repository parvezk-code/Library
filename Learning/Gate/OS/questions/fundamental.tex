
\centerline{\textbf{ \LARGE OS Fundamentals}}

% ----------------------------------------------------------------------------

\begin{questyle}

\question A CPU-bound process is.

\begin{oneparchoices}
   \CorrectChoice More computation.
   \choice More I/O.
   \choice Both.
   \choice None.
\end{oneparchoices}

\end{questyle}



% ----------------------------------------------------------------------------

\begin{questyle}

\question A IO-bound process is.

\begin{oneparchoices}
   \choice More computation.
   \CorrectChoice More I/O.
   \choice Both.
   \choice None.
\end{oneparchoices}

\end{questyle}


% ----------------------------------------------------------------------------

\begin{questyle}

\question Device driver is.

  \begin{choices}
    \CorrectChoice Software routine which interface with h/w device.
    \choice Part of a device that allows it to physically function (e.g spin a disk).
    \choice Featutre of a h/w device that allow it to interact with OS.
    \choice A hardware component.
  \end{choices}
\end{questyle}


% ----------------------------------------------------------------------------

\begin{questyle}

  \question Which of the following is an example of a spooled device? (GATE - 92, 96, 98)

  \begin{choices}
    \choice The terminal used to enter the input data for the C program being executed.
    \choice An output device used to print the output of a number of jobs.
    \choice The secondary memory device in a virtual storage system.
    \choice The swapping area on a disk used by the sapper.
  \end{choices}

\end{questyle}


% ----------------------------------------------------------------------------

\begin{questyle}

  \question  Total no of child process in following code. (GATE - 2012) \newline
    fork()  \newline fork()  \newline  fork()

  \begin{oneparchoices}
    \choice 3
    \choice 4
    \CorrectChoice 7
    \choice 8
  \end{oneparchoices}

\end{questyle}


% ----------------------------------------------------------------------------


\begin{questyle}

  \question Maximium number of process in ready queue system in n CPUs. (GATE - 2015\_set\_3)

  \begin{oneparchoices}
    \choice n
    \choice 2n
    \choice 3n
    \CorrectChoice Independent on n
  \end{oneparchoices}

\end{questyle}



% ----------------------------------------------------------------------------


\begin{questyle}

  \question  Multi user and multi-processing OS can not be implemented on harrdware that do not have. (GATE-1999, could be wrong question)

  \begin{oneparchoices}
    \choice Address Translation
    \choice DMA
    \choice 2 modes of CPU
    \choice Demand Paging
  \end{oneparchoices}

\end{questyle}



% ----------------------------------------------------------------------------

\begin{questyle}

  \question  Which of the following actions is/are typically not performed by the
              operating system when switching context from process A to process B? (GATE - 1999)

  \begin{choices}
    \choice Saving current register values and restoring saved register values for process B.
    \choice Changing address translation tables.
    \CorrectChoice Swapping out the memory image of process A to the disk.
    \choice Invalidating the translation look-aside buffer.
  \end{choices}

\end{questyle}



% ----------------------------------------------------------------------------

\begin{questyle}

  \question  Features that are sufficient for multi-programming OS. (GATE-2002)

   \begin{enumerate}
      \item[a] More than one program loaded in main memory at same time.
      \item[b] If a program waits for I/O another program is immediately scheduled.
      \item[c] If program terminates another program is immediately scheduled for execution.
   \end{enumerate}

  \begin{oneparchoices}
    \choice a
    \CorrectChoice a and b
    \choice a and c
    \choice a, b and c
  \end{oneparchoices}

\end{questyle}



% ----------------------------------------------------------------------------

\begin{questyle}

  \question  Let u, v be the values printed by the parent process, and x, y be the
            values printed by the child process. Which one of the following is TRUE?

    \begin{lstlisting}
      if(fork()==0)
      {
          a = a + 5;
          print(a, &a);
      }
      else
      {
          a = a - 5;
          print(a, &a);
      }
    \end{lstlisting}

  \begin{oneparchoices}
    \choice u=x+10, v=y
    \choice u=x+10, v!=y
    \CorrectChoice u=x-10, v=y
    \choice u=x-10, v!=y
  \end{oneparchoices}

  \end{questyle}



% ----------------------------------------------------------------------------

\begin{questyle}

  \question  No of child process created. (GATE-2008)

  \begin{lstlisting}
      for(i=0; i<n; i++) fork();
  \end{lstlisting}

  \begin{oneparchoices}
    \choice n
    \CorrectChoice \(2^n -1 \)
    \choice \(2^n \)
    \choice \(2^{n+1} -1 \)
  \end{oneparchoices}

\end{questyle}



% ----------------------------------------------------------------------------

\begin{questyle}

  \question  Given state transition diagram. Which statements are true. (GATE-2009)
  \begin{lstlisting}
            A               B                 D
    start ----->  Ready <=======> Running   -----> Terminated
                  |         C       |
                E |                 | F
                  |<====Blocked=====|
  \end{lstlisting}
  \begin{enumerate}
      \item[I] Transition D will result in transition A.
      \item[II] Transition E is possible if some process is in running state.
      \item[III] OS use preemptive scheduling.
      \item[IV] OS use non-preemptive scheduling.
   \end{enumerate}

  \begin{oneparchoices}
    \choice 1 \& 2
    \choice 1 \& 3
    \CorrectChoice 2 \& 3
    \choice 2 \& 4
  \end{oneparchoices}

\end{questyle}



% ----------------------------------------------------------------------------

\begin{questyle}

  \question  Suppose a processor does not have any stack pointer register. Which of the following statements is true?(GATE-2001)

  \begin{choices}
    \CorrectChoice It cannot have subroutine call instruction
    \choice It can have subroutine call instruction, but no nested subroutine calls
    \choice Nested subroutine calls are possible, but interrupts are not
    \choice All sequences of subroutine calls and also interrupts are possible
  \end{choices}

  \end{questyle}



% ----------------------------------------------------------------------------

\begin{questyle}

  \question  Where does the swap space reside? (GATE-2001 )

  \begin{oneparchoices}
    \choice RAM
    \choice ROM
    \CorrectChoice Disk
    \choice On-chip cache
  \end{oneparchoices}

  \end{questyle}

  % ----------------------------------------------------------------------------

