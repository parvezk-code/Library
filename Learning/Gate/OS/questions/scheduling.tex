
\centerline{\textbf{ \LARGE Scheduling Algorithms}}


% ----------------------------------------------------------------------------

\begin{questyle}

\question Which policy is best for time sharing OS. (GATE - 95)

\begin{oneparchoices}
   \choice SJF
   \CorrectChoice Round Robin
   \choice FCFS
   \choice SRTF
\end{oneparchoices}

  \end{questyle}



% ----------------------------------------------------------------------------

\begin{questyle}


\question State transition diagram represents . (GATE - 96)
  \begin{lstlisting}
            A               B                 D
    start ----->  Ready <=======> Running   -----> Terminated
                  |         C       |
                E |                 | F
                  |<====Blocked=====|
  \end{lstlisting}

\begin{choices}
   \choice Batch OS
   \CorrectChoice OS with preemptive scheduling
   \choice OS with non-preemptive scheduling
   \choice Uni-programmed OS
\end{choices}


  \end{questyle}



% ----------------------------------------------------------------------------


\begin{questyle}

\question Total time taken in round robin algorithm for n process. Quantam size=q and context switch time=s  (GATE - 98)

% ----------------------------------------------------------------------------

\begin{oneparchoices}
   \CorrectChoice n(s+q)
   \choice n(s-q)
   \choice n/(s+q)
   \choice n-(s*q)
\end{oneparchoices}


  \end{questyle}



% ----------------------------------------------------------------------------

\begin{questyle}

  \question Which scheduling algorithm provides maximum throughput. (GATE -2001 )

  \begin{oneparchoices}
    \choice Round Robin
    \CorrectChoice SJF
    \choice FCFS
    \choice Priority based
  \end{oneparchoices}

  \end{questyle}



% ----------------------------------------------------------------------------


\begin{questyle}

  \question Which is non primitive. (GATE - 02)

  \begin{choices}
    \choice Round Robin
    \CorrectChoice FCFS
    \choice Multi-level queue scheduling
    \choice Multi-level queue scheduling with feedback
  \end{choices}

  \end{questyle}



% ----------------------------------------------------------------------------


\begin{questyle}

  \question Consider three CPU-intensive processes, which require 10, 20 and 30 time units and arrive at
            times 0, 2 and 6, respectively. How many context switches are needed if the operating system
            implements a shortest remaining time first scheduling algorithm? Do not count the context
            switches at time zero and at the end. (GATE - 2006)

  \begin{oneparchoices}
    \choice 1
    \CorrectChoice 2
    \choice 3
    \choice 4
  \end{oneparchoices}

  \end{questyle}



% ----------------------------------------------------------------------------


\begin{questyle}

  \question Match the following. (GATE - 2007)

  \begin{lstlisting}
     P). Gang Scheduling        1). Guranteed Scheduling
     Q). Rate Monotonic         2). Real-time Scheduling
     R). Fair share Scheduling  3). Thread Scheduling

  \end{lstlisting}

  \begin{choices}
    \CorrectChoice P-3, Q-2, R-1
    \choice P-1, Q-2, R-3
    \choice P-2, Q-3, R-1
    \choice P-1, Q-3, R-2
  \end{choices}


  \end{questyle}

% ----------------------------------------------------------------------------

\begin{questyle}

  \question Choose correct options (GATE-2010)

   \begin{enumerate}
      \item[I] SRTF causes starvation.
      \item[II] Premptive scheduling may cause starvation.
      \item[III] Round robin is bettre than FCFS in response time.
   \end{enumerate}

  \begin{oneparchoices}
    \choice I only
    \choice I, III only
    \choice II, III only
    \CorrectChoice I, II, III
  \end{oneparchoices}

  \end{questyle}

% ----------------------------------------------------------------------------


\begin{questyle}

  \question  A scheduling algorithm assigns priority proportional to the waiting time of a process.
            Every process starts with priority zero (the lowest priority). The scheduler re-evaluates
            the process priorities every T time units and decides the next process to schedule.
            Which one of the following is TRUE if the processes have no I/O operations and all arrive at time zero? (GATE - 2013)

  \begin{choices}
    \choice This algorithm is equivalant to FCFS.
    \CorrectChoice This algorithm is equivalant to round robin.
    \choice This algorithm is equivalant to SJF.
    \choice This algorithm is equivalant to SRTF.
  \end{choices}


  \end{questyle}



% ----------------------------------------------------------------------------


\begin{questyle}

  \question  Consider an arbitrary set of CPU-bound processes with unequal CPU burst lengths submitted at
            the same time to a computer system. Which one of the following process scheduling algorithms would
            minimize the average waiting time in the ready queue? (GATE - 2016\_set\_1)

  \begin{choices}
    \CorrectChoice SRTF
    \choice Largest Job First (opposite of SJF).
    \choice Iniform random
    \choice Round robin
  \end{choices}

  \end{questyle}



% ----------------------------------------------------------------------------


\begin{questyle}

  \question Highest response ratio next favours \fillin[shorter burst]  jobs, but it also limits the
            waiting time of \fillin[Long burst] jobs. (GATE - 1990)

  \end{questyle}



% ----------------------------------------------------------------------------


\begin{questyle}

  \question If the pre-emptive shortest remaining time first scheduling algorithm is used to schedule the processes, then
            the average waiting time across all processes is \fillin[3] milliseconds. (GATE - 2017\_set\_1)

  \begin{myTableStyle}
    \begin{center}
    \begin{tabular}{ |l|c|c| } \hline
        Process &   Arrival & Burst    \\ \hline
        P1      &   0       & 7         \\ \hline
        P2      &   3       & 3         \\ \hline
        P3      &   5       & 5         \\ \hline
        P4      &   6       & 2         \\ \hline

    \end{tabular}
    \end{center}
  \end{myTableStyle}

  \end{questyle}



% ----------------------------------------------------------------------------


\begin{questyle}

  \question Assume that  the following jobs are to be executed on a single processor system. (GATE - 1993, 96)

    \begin{myTableStyle}
    \begin{center}
    \begin{tabular}{ |l|c|c| } \hline
        Job Id &   CPU Burst Time     \\ \hline
        P      &   4                \\ \hline
        Q      &   1                \\ \hline
        R      &   8                \\ \hline
        S      &   1                \\ \hline
        T      &   2                \\ \hline
    \end{tabular}
    \end{center}
  \end{myTableStyle}
  \vspace{0.08in}
  The jobs are assumed to have arrived at time 0+ and in the order p, q, r, s, t. Calculate the
  departure time (completion time) for job p if scheduling is round robin with time slice 1.

  \begin{oneparchoices}
    \choice 4
    \choice 10
    \CorrectChoice 11
    \choice 12
    \choice none
  \end{oneparchoices}
  \end{questyle}

% ----------------------------------------------------------------------------


\begin{questyle}

  \question  The sequence \fillin[] is an optimal non-preemptive scheduling sequence for the following jobs
             which leaves the CPU idle for \fillin[] unit(s) of time.(GATE - 1995)

  \begin{myTableStyle}
    \begin{center}
    \begin{tabular}{ |l|c|c| } \hline
        Process &   Arrival & Burst    \\ \hline
        1      &   0.0       & 9         \\ \hline
        2      &   0.6       & 5         \\ \hline
        3      &   1.0       & 1         \\ \hline
    \end{tabular}
    \end{center}
  \end{myTableStyle}
  \vspace{0.08in}
  \begin{oneparchoices}
    \CorrectChoice \{3,2,1\}, 1
    \choice \{2,1,3\}, 0
    \choice \{3,2,1\}, 0
    \choice \{1,2,3\}, 5
  \end{oneparchoices}

  \end{questyle}



% ----------------------------------------------------------------------------


\begin{questyle}

  \question  Burst-[6,3,5, x]. Execution order for minimum average response time. (GATE - 1998)


  \end{questyle}




% ----------------------------------------------------------------------------

\begin{questyle}

  \question  A uni-processor computer system only has 2 processes. Both alternating 10ms CPU and 90ms IO bursts.
             Both arrive at same time. I/O of both process can proceed in parallel. For least CPU utilization. (GATE-2003 )

  \begin{choices}
    \choice FCSC
    \choice SRTF
    \choice priority scheduling with different priorities to both process.
    \CorrectChoice Round robin scheduling with quanta 5.
  \end{choices}


  \end{questyle}



% ----------------------------------------------------------------------------


\begin{questyle}

  \question  Consider the following set of processes, with the arrival times and the CPU-burst times given in milliseconds.
            What is the average turnaround time for these processes with the preemptive shortest remaining processing
            time first (SRPF) algorithm ?(GATE- 2004)

  \begin{myTableStyle}
    \begin{center}
    \begin{tabular}{ |l|c|c| } \hline
        Process &   Arrival & Burst    \\ \hline
        1      &   0       & 5         \\ \hline
        2      &   1       & 3         \\ \hline
        3      &   2       & 3         \\ \hline
        4      &   4       & 1         \\ \hline
    \end{tabular}
    \end{center}
  \end{myTableStyle}
  \vspace{0.08in}

  \begin{oneparchoices}
    \CorrectChoice 5.5
    \choice 5.75
    \choice 6
    \choice 6.25
  \end{oneparchoices}

  \end{questyle}



% ----------------------------------------------------------------------------

\begin{questyle}

  \question  Consider three processes, all arriving at time zero, with total execution time of 10, 20 and 30 units,
            respectively. Each process spends the first 20\% of execution time doing I/O, the next 70\% of time doing
            computation, and the last 10\% of time doing I/O again. The operating system uses a shortest remaining
            compute time first scheduling algorithm and schedules a new process either when the running process
            gets blocked on I/O or when the running process finishes its compute burst. Assume that all I/O operations
            can be overlapped as much as possible. For what percentage of time does the CPU remain idle? (GATE-2006)

  \begin{oneparchoices}
    \choice 0
    \CorrectChoice 10.6
    \choice 30
    \choice 89.4
  \end{oneparchoices}

  \end{questyle}

% ----------------------------------------------------------------------------

\begin{questyle}

  \question An operating system uses Shortest Remaining Time first (SRT) process scheduling algorithm.
            Consider the arrival times and execution times for the following processes.
            What is the total waiting time for process P2? (GATE-2007)

  \begin{myTableStyle}
    \begin{center}
    \begin{tabular}{ |l|c|c| } \hline
        Process &   Arrival & Burst    \\ \hline
        P1      &   00       & 20         \\ \hline
        P2      &   15       & 25         \\ \hline
        P3      &   30       & 10         \\ \hline
        P4      &   45       & 15         \\ \hline

    \end{tabular}
    \end{center}
  \end{myTableStyle}
  \vspace{0.08in}

  \begin{oneparchoices}
    \choice 5
    \CorrectChoice 15
    \choice 40
    \choice 55
  \end{oneparchoices}

  \end{questyle}

% ----------------------------------------------------------------------------

\begin{questyle}

  \question  Consider the following table of arrival time and burst time for three processes P0, P1 and P2.
            The pre-emptive shortest job first scheduling algorithm is used. Scheduling is carried out only
            at arrival or completion of processes. What is the average waiting time for the three processes? (GATE-2011)

  \begin{myTableStyle}
    \begin{center}
    \begin{tabular}{ |l|c|c| } \hline
        Process &   Arrival & Burst    \\ \hline
        P0      &   0       & 9         \\ \hline
        P1      &   1       & 4         \\ \hline
        P2      &   2       & 9         \\ \hline
    \end{tabular}
    \end{center}
  \end{myTableStyle}
  \vspace{0.08in}

  \begin{oneparchoices}
    \CorrectChoice 5
    \choice 4.33
    \choice 6.33
    \choice 7.33
  \end{oneparchoices}

\end{questyle}



% ----------------------------------------------------------------------------

\begin{questyle}

  \question Consider the 3 processes, P1, P2 and P3 shown in the table. The completion order of the 3 processes
            under the policies FCFS and RR2 (round robin scheduling with CPU quantum of 2 time units) are (GATE-2012 )

  \begin{myTableStyle}
    \begin{center}
    \begin{tabular}{ |l|c|c| } \hline
        Process &   Arrival & Burst    \\ \hline
        P1      &   0       & 5         \\ \hline
        P2      &   1       & 7         \\ \hline
        P3      &   3       & 4         \\ \hline
    \end{tabular}
    \end{center}
  \end{myTableStyle}
  \vspace{0.08in}

  \begin{choices}
    \choice FCFS: p1, p2, p3  \qquad   RR2: p1, p2, p3
    \choice FCFS: p1, p3, p2  \qquad   RR2: p1, p3, p2
    \CorrectChoice FCFS: p1, p2, p3  \qquad   RR2: p1, p3, p2
    \choice FCFS: p1, p3, p2  \qquad   RR2: p1, p2, p3
  \end{choices}

  \end{questyle}



% ----------------------------------------------------------------------------

\begin{questyle}

  \question Three processes A, B and C each execute a loop of 100 iterations. In each iteration of the
            loop, a process performs a single computation that requires t\_cpu CPU milliseconds and then initiates
            a single I/O operation that lasts for t\_io milliseconds. It is assumed that the computer where the
            processes execute has sufficient number of I/O devices and the OS of the computer assigns
            different I/O devices to each process. Also, the scheduling overhead of the OS is negligible.
            The processes A, B, and C are started at times 0, 5 and 10 milliseconds respectively, in a pure
            time sharing system (round robin scheduling) that uses a time slice of 50 milliseconds. The time
            in milliseconds at which process C would complete its first I/O operation is \fillin[1000]. (GATE-2014\_set\_2 )

  \begin{myTableStyle}
    \begin{center}
    \begin{tabular}{ |c|c|c|c| } \hline
          ID & Time\_CPU & Time\_IO  & Arrival \\ \hline
          A & 100ms & 500ms & 0 \\ \hline
          B & 350ms & 500ms & 5 \\ \hline
          C & 200ms & 500ms & 10 \\ \hline
    \end{tabular}
    \end{center}
  \end{myTableStyle}
  \vspace{0.08in}

  \end{questyle}

% ----------------------------------------------------------------------------

\begin{questyle}

  \question  An operating system uses shortest remaining time first scheduling algorithm for pre-emptive
            scheduling of processes. Consider the following set of processes with their arrival times and
            CPU burst times (in milliseconds). The average waiting time (in milliseconds) of the
            processes is \fillin[5.5]. (GATE-2014\_set\_3 )

    \begin{myTableStyle}
    \begin{center}
    \begin{tabular}{ |c|c|c|c| } \hline
          ID & Arrival &  Burst     \\ \hline
          P1 & 0 & 12     \\ \hline
          P2 & 2 & 4     \\ \hline
          P3 & 3 & 6     \\ \hline
          P4 & 8 & 5      \\ \hline
    \end{tabular}
    \end{center}
  \end{myTableStyle}
  \vspace{0.08in}

  \end{questyle}



% ----------------------------------------------------------------------------

\begin{questyle}

  \question  Consider a uniprocessor system executing three tasks T1, T2 and T3, each of which is
            composed of an infinite sequence of jobs (or instances) which arrive periodically at intervals
            of 3, 7 and 20 milliseconds, respectively. The priority of each task is the inverse of
            its period and the available tasks are scheduled in order of priority, with the highest
            priority task scheduled first. Each instance of T1, T2 and T3 requires an execution time
            of 1, 2 and 4 milliseconds, respectively. Given that all tasks initially arrive at the
            beginning of the 1st milliseconds and task preemptions are allowed, the first
            instance of T3 completes its execution at the end of \fillin[12]. (GATE-2015\_set\_1 )

  \end{questyle}



% ----------------------------------------------------------------------------
\begin{questyle}

  \question  For the processes listed in the following table, which of the following scheduling schemes
            will give the lowest average turnaround time? (GATE-2015\_set\_3 )

    \begin{myTableStyle}
    \begin{center}
    \begin{tabular}{ |c|c|c|c| } \hline
          ID & Arrival &  Burst     \\ \hline
          P1 & 0 & 3     \\ \hline
          P2 & 1 & 6     \\ \hline
          P3 & 4 & 4     \\ \hline
          P4 & 6 & 2      \\ \hline
    \end{tabular}
    \end{center}
  \end{myTableStyle}
  \vspace{0.08in}

  \begin{choices}
    \choice FCFS
    \choice SJF non pre-emptive
    \CorrectChoice SRTF
    \choice Round robin with time quanta 2.
  \end{choices}

  \end{questyle}



% ----------------------------------------------------------------------------


\begin{questyle}

  \question Pre-emptive SRTF scheduling algorithm is used. Average turn around time is \fillin[8.25]. (GATE-2016\_set\_2 )

    \begin{myTableStyle}
    \begin{center}
    \begin{tabular}{ |c|c|c|c| } \hline
          ID & Arrival &  Burst     \\ \hline
          P1 & 0 & 10     \\ \hline
          P2 & 3 & 6     \\ \hline
          P3 & 7 & 1     \\ \hline
          P4 & 8 & 3      \\ \hline
    \end{tabular}
    \end{center}
  \end{myTableStyle}

  \end{questyle}

% ----------------------------------------------------------------------------

\begin{questyle}

  \question  Which of the following scheduling algorithm is non pre-emptive. (GATE-2002)

  \begin{choices}
    \choice Round Robin
    \CorrectChoice FCFS
    \choice Multi-level queue
    \choice Multi-level queue with feedback
  \end{choices}

  \end{questyle}



% ----------------------------------------------------------------------------
% ----------------------------------------------------------------------------
% ----------------------------------------------------------------------------
% ----------------------------------------------------------------------------
% ----------------------------------------------------------------------------

