
\centerline{\textbf{ \LARGE OS Fundamentals}}

\setcounter{question}{0}

% ----------------------------------------------------------------------------

\begin{minipage}{\linewidth}

\question A CPU-bound process is.

\begin{oneparchoices}
   \choice More computation.
   \choice More I/O.
   \choice Both.
   \choice None.
\end{oneparchoices}

\end{minipage}

\vspace{0.08in}

% ----------------------------------------------------------------------------

\begin{minipage}{\linewidth}

\question A IO-bound process is.

\begin{oneparchoices}
   \choice More computation.
   \choice More I/O.
   \choice Both.
   \choice None.
\end{oneparchoices}

\end{minipage}

\vspace{0.08in}
% ----------------------------------------------------------------------------

\begin{minipage}{\linewidth}

\question Device driver is.

  \begin{choices}
    \choice Software routine which interface with h/w device.
    \choice Part of a device that allows it to physically function (e.g spin a disk).
    \choice Featutre of a h/w device that allow it to interact with OS.
    \choice A hardware component.
  \end{choices}
\end{minipage}

\vspace{0.08in}
% ----------------------------------------------------------------------------

\begin{minipage}{\linewidth}

  \question Example of spooled device. (GATE - 92, 96, 98)

  \begin{choices}
    \choice Line printer which print output of number of jobs.
    \choice Terminal used to enter input data to a running program.
    \choice Seconday storage device in virtual memory.
    \choice A graphical display device.
  \end{choices}

\end{minipage}

\vspace{0.08in}
% ----------------------------------------------------------------------------

\begin{minipage}{\linewidth}

  \question  Total no of child process in following code. (GATE - 2012) \newline
    fork()  \newline fork()  \newline  fork()

  \begin{oneparchoices}
    \choice 3
    \choice 4
    \choice 7
    \choice 8
  \end{oneparchoices}

\end{minipage}

\vspace{0.08in}
% ----------------------------------------------------------------------------


\begin{minipage}{\linewidth}

  \question Maximium number of process in ready queue system in n CPUs. (GATE - 2015\_set\_3)

  \begin{oneparchoices}
    \choice n
    \choice 2n
    \choice 3n
    \choice Independent on n
  \end{oneparchoices}

\end{minipage}

\vspace{0.08in}

% ----------------------------------------------------------------------------


\begin{minipage}{\linewidth}

  \question  Multi user and multi-processing OS can not be implemented on harrdware that do not have. (GATE-1999)

  \begin{choices}
    \choice Address Translation
    \choice DMA
    \choice 2 modes of CPU
    \choice Demand Paging
  \end{choices}

\end{minipage}

\vspace{0.08in}

% ----------------------------------------------------------------------------

\begin{minipage}{\linewidth}

  \question  Context Switching from A to B do not involve. (GATE - 1999)

  \begin{choices}
    \choice Saving and restoring registers of A and B.
    \choice Changing address translation tables.
    \choice Swapping cureent process A to disk.
    \choice Invalidating TLB.
  \end{choices}

\end{minipage}

\vspace{0.08in}

% ----------------------------------------------------------------------------

\begin{minipage}{\linewidth}

  \question  Features that are sufficient for multi-programming OS. (GATE-2002)

   \begin{enumerate}
      \item[a] More than one program loaded in main memory at same time.
      \item[b] If a program waits for I/O another program is immediately scheduled.
      \item[c] If program terminates another program is immediately scheduled for execution.
   \end{enumerate}

  \begin{choices}
    \choice a
    \choice a and b
    \choice a and c
    \choice a, b and c
  \end{choices}

\end{minipage}

\vspace{0.08in}

% ----------------------------------------------------------------------------

\begin{minipage}{\linewidth}

  \question  (u, v) are printed by parent and (x,y) printed by child. (GATE-2005 )

    \begin{lstlisting}
      if(fork()==0)
      {
          a = a + 5;
          print(a, &a);
      }
      else
      {
          a = a - 5;
          print(a, &a);
      }
    \end{lstlisting}

  \begin{choices}
    \choice u=x+10, v=y
    \choice u=x+10, v!=y
    \choice u=x-10, v=y
    \choice u=x-10, v!=y
  \end{choices}

  \end{minipage}

\vspace{0.08in}

% ----------------------------------------------------------------------------

\begin{minipage}{\linewidth}

  \question  No of child process created. (GATE-2008)

  \begin{lstlisting}
      for(i=0; i<n; i++) fork();
  \end{lstlisting}

  \begin{choices}
    \choice n
    \choice \(2^n -1 \)
    \choice \(2^n \)
    \choice \(2^{n+1} -1 \)


  \end{choices}

\end{minipage}

\vspace{0.08in}

% ----------------------------------------------------------------------------

\begin{minipage}{\linewidth}

  \question  Given state transition diagram. Which statements are true. (GATE-2009)

  \begin{enumerate}
      \item[I] Transition D will result in transition A.
      \item[II] Transition E is possible if some process is in running state.
      \item[III] OS use preemptive scheduling.
      \item[IV] OS use non-preemptive scheduling.
   \end{enumerate}

  \begin{choices}
    \choice 1 \& 2
    \choice 1 \& 3
    \choice 2 \& 3
    \choice 2 \& 4
  \end{choices}

\end{minipage}

\vspace{0.08in}

% ----------------------------------------------------------------------------

\begin{comment}

% ----------------------------------------------------------------------------

\begin{minipage}{\linewidth}

  \question  (GATE- )

  \begin{choices}
    \choice
    \choice
    \choice
    \choice
  \end{choices}

\end{minipage}

\vspace{0.08in}

  %% oneparchoices

\end{comment}
