
\centerline{\textbf{ \LARGE Scheduling Algorithms}}


\setcounter{question}{0}

% ----------------------------------------------------------------------------

\begin{minipage}{\linewidth}

\question Which policy is best for time sharing OS. (GATE - 95)

\begin{oneparchoices}
   \choice SJF
   \choice Round Robin
   \choice FCFS
   \choice SRTF
\end{oneparchoices}

  \end{minipage}

\vspace{0.08in}

% ----------------------------------------------------------------------------

\begin{minipage}{\linewidth}


\question State transition diagram represents . (GATE - 96)
\begin{choices}
   \choice Batch OS
   \choice OS with preemptive scheduling
   \choice OS with non-preemptive scheduling
   \choice Uni-programmed OS
\end{choices}


  \end{minipage}

\vspace{0.08in}

% ----------------------------------------------------------------------------


\begin{minipage}{\linewidth}

\question Total time taken in round robin algorithm for n process. Quantam size=q and context switch time=s  (GATE - 98)

% ----------------------------------------------------------------------------

\begin{oneparchoices}
   \choice n(s+q)
   \choice n(s-q)
   \choice n/(s+q)
   \choice n-(s*q)
\end{oneparchoices}


  \end{minipage}

\vspace{0.08in}

% ----------------------------------------------------------------------------

\begin{minipage}{\linewidth}

  \question Which scheduling algorithm provides maximum throughput. (GATE -2001 )

  \begin{oneparchoices}
    \choice Round Robin
    \choice SJF
    \choice FCFS
    \choice Priority based
  \end{oneparchoices}

  \end{minipage}

\vspace{0.08in}

% ----------------------------------------------------------------------------


\begin{minipage}{\linewidth}

  \question Which is non primitive. (GATE - 02)

  \begin{choices}
    \choice Round Robin
    \choice FCFS
    \choice Multi-level queue scheduling
    \choice Multi-level queue scheduling with feedback
  \end{choices}

  \end{minipage}

\vspace{0.08in}

% ----------------------------------------------------------------------------


\begin{minipage}{\linewidth}

  \question No of context switching in SRTF. bursts=[10,20,30] arrival=[0,2,6] (GATE - 2006)

  \begin{oneparchoices}
    \choice 1
    \choice 2
    \choice 3
    \choice 4
  \end{oneparchoices}

  \end{minipage}

\vspace{0.08in}

% ----------------------------------------------------------------------------


\begin{minipage}{\linewidth}

  \question Match the following. (GATE - 2007) \newline
   P). Gang Scheduling  Q). Rate Monotonic  R). Fair share Scheduling \newline
   1). Guranteed Scheduling 2). Real-time Scheduling 3). Thread Scheduling
  \begin{choices}
    \choice P-3, Q-2, R-1
    \choice P-1, Q-2, R-3
    \choice P-2, Q-3, R-1
    \choice P-1, Q-3, R-2
  \end{choices}


  \end{minipage}

\vspace{0.08in}

% ----------------------------------------------------------------------------

\begin{minipage}{\linewidth}

  \question Choose correct options (GATE-2010)

   \begin{enumerate}
      \item[I] SRTF causes starvation.
      \item[II] Premptive scheduling may cause starvation.
      \item[III] Round robin is bettre than FCFS in response time.
   \end{enumerate}

  \begin{oneparchoices}
    \choice I only
    \choice I, III only
    \choice II, III only
    \choice I, II, III
  \end{oneparchoices}


  \end{minipage}

\vspace{0.08in}

% ----------------------------------------------------------------------------


\begin{minipage}{\linewidth}

  \question  Given that priority is propotional to waiting time. After every T seconds priority is re-valuated.
             All process arrive at time zero with zero priority. (GATE - 2013)

  \begin{choices}
    \choice This algorithm is equivalant to FCFS.
    \choice This algorithm is equivalant to round robin.
    \choice This algorithm is equivalant to SJF.
    \choice This algorithm is equivalant to SRTF.
  \end{choices}


  \end{minipage}

\vspace{0.08in}

% ----------------------------------------------------------------------------


\begin{minipage}{\linewidth}

  \question  CPU-bound processes arrive at same time with un-equal bust.
            Which scheduling algorithm will minimize average waiting time. (GATE - 2016\_set\_1)

  \begin{choices}
    \choice SRTF
    \choice Largest Job First (opposite of SJF).
    \choice Iniform random
    \choice Round robin
  \end{choices}

  \end{minipage}

\vspace{0.08in}

% ----------------------------------------------------------------------------


\begin{minipage}{\linewidth}


  \question Highest response ratio next favours \fillin[shorter burst]  jobs, but it also limits the
            waiting time of \fillin[Long burst] jobs. (GATE - 1990)

  \end{minipage}

\vspace{0.08in}

% ----------------------------------------------------------------------------


\begin{minipage}{\linewidth}

  \question \fillin[.........] is the average waiting time for SRTF. (GATE - 2017\_set\_1) \newline
            [P1, P2, P3, P4]
            Arrival-[0,  3,  5,  6 ]
            Bursts-[7,  3,  5,  2 ]

  \end{minipage}

\vspace{0.08in}

% ----------------------------------------------------------------------------


\begin{minipage}{\linewidth}

  \question Jobs[p, q, r, s, t] bursts[4,1,8,1,2]. Completion time of P. Round robin scheduling with time slice of 1.
            Single process system. (GATE - 1993, 96)

  \begin{oneparchoices}
    \choice 4
    \choice 10
    \choice 11
    \choice 12
  \end{oneparchoices}


  \end{minipage}

\vspace{0.08in}

% ----------------------------------------------------------------------------


\begin{minipage}{\linewidth}

  \question  Choose correct sequence and cpu idle time for non preemptive scheduling.(GATE - 1995)
             \newline job-[1,2,3] Bursts-[9,5,1] Arrival-[0, 0.6, 1]

  \begin{oneparchoices}
    \choice \{3,2,1\}, 1
    \choice \{2,1,3\}, 0
    \choice \{3,2,1\}, 0
    \choice \{1,2,3\}, 5
  \end{oneparchoices}

  \end{minipage}

\vspace{0.08in}

% ----------------------------------------------------------------------------


\begin{minipage}{\linewidth}

  \question  Burst-[6,3,5, x]. Execution order for minimum average response time. (GATE - 1998)


  \end{minipage}

\vspace{0.08in}


% ----------------------------------------------------------------------------

\begin{minipage}{\linewidth}

  \question  A uni-processor computer system only has 2 processes. Both alternating 10ms CPU and 90ms IO bursts.
             Both arrive at same time. I/O of both process can proceed in parallel. For least CPU utilization. (GATE-2003 )

  \begin{choices}
    \choice FCSC
    \choice SRTF
    \choice priority scheduling with different priorities to both process.
    \choice Round robin scheduling.
  \end{choices}


  \end{minipage}

\vspace{0.08in}

% ----------------------------------------------------------------------------


\begin{minipage}{\linewidth}

  \question  Jobs-[p1, p2, p3, p4] Bursts-[5,3,3,1] Arrival-[0,1,2,4]. Average turnaroud time for SRTF. (GATE- 2004)

  \begin{oneparchoices}
    \choice 5.5
    \choice 5.75
    \choice 6
    \choice 6.25
  \end{oneparchoices}

  \end{minipage}

\vspace{0.08in}

% ----------------------------------------------------------------------------


\begin{minipage}{\linewidth}

  \question  Bursts-[10,20,30]. Each process does 20\% execution then 70\% IO then 10\% execution.
             SRTF scheduling is done. IO of all rpocess can go in parallel. New process is
             schedled when current process goes for IO or completes its burst.
             Calculate percentage of time CPU is idle. (GATE-2006)

  \begin{oneparchoices}
    \choice 0
    \choice 10.6
    \choice 30
    \choice 89.4
  \end{oneparchoices}

  \end{minipage}

\vspace{0.08in}


% ----------------------------------------------------------------------------

\begin{minipage}{\linewidth}

  \question Using SRTF, calculate waiting tome for P2.
            Process-[1,2,3,4] Bursts-[20, 25, 10, 15] AT-[0, 15, 30, 45] (GATE-2007)

  \begin{oneparchoices}
    \choice 5
    \choice 15
    \choice 40
    \choice 55
  \end{oneparchoices}

  \end{minipage}

\vspace{0.08in}

% ----------------------------------------------------------------------------

\begin{minipage}{\linewidth}

  \question  Process-[0,1,2] Burst-[9,4,9 Arrival-[0,1,2]]. Calculate average waiting time.
             SJF scheduling is used and scheduling is done only when process arrive or completes. (GATE-2011)

  \begin{oneparchoices}
    \choice 5
    \choice 4.33
    \choice 6.33
    \choice 7.33
  \end{oneparchoices}

\end{minipage}

\vspace{0.08in}

% ----------------------------------------------------------------------------

\begin{minipage}{\linewidth}

  \question Process-[P1, p2, p3] Bursts-[5, 7, 4] AT-[0, 1, 3].
            Completion order for FCFS and Round Robin with time quanta of 2. (GATE-2012 )

  \begin{choices}
    \choice FCFS: p1, p2, p3  \qquad   RR2: p1, p2, p3
    \choice FCFS: p1, p3, p2  \qquad   RR2: p1, p3, p2
    \choice FCFS: p1, p2, p3  \qquad   RR2: p1, p3, p2
    \choice FCFS: p1, p3, p2  \qquad   RR2: p1, p2, p3
  \end{choices}

  \end{minipage}

\vspace{0.08in}

% ----------------------------------------------------------------------------

\begin{minipage}{\linewidth}

  \question  Using Round Robin scheduling with quanta 50ms. Each process executes a loop of 100 iterations.
             In each iteration process does a CPU computation and then an IO operation.
             IO of each process can go in parallel. Process C completes its first IO operation at time \fillin[time]. (GATE-2014\_set\_2 )
      \begin{center}
      \begin{tabular}{ c c c c }
          ID & Time\_CPU & Time\_IO  & Arrival\\
          A & 100ms & 500ms & 0 \\
          B & 350ms & 500ms & 5 \\
          C & 200ms & 500ms & 10 \\
      \end{tabular}
    \end{center}

  \end{minipage}

\vspace{0.08in}



% ----------------------------------------------------------------------------

\begin{minipage}{\linewidth}

  \question  Using SRTF pre-emptive scheduling. Average waiting time is \fillin[time]. (GATE-2014\_set\_3 )
      \begin{center}
      \begin{tabular}{ c c c c }
          ID & Burst & Arrival\\
          p1 & 12  & 0 \\
          p2 & 4  & 2 \\
          p3 & 6  & 3 \\
          p4 & 5  & 8 \\
      \end{tabular}
    \end{center}

  \end{minipage}

\vspace{0.08in}

% ----------------------------------------------------------------------------

\begin{minipage}{\linewidth}

  \question  Uniprocessor system executes 3 tasks T1, T2, T3. These 3 tasks arrive priodically every 3, 7, 20 msec respectively.
            CPU bursts are 1, 2, 4 respectively. All arrive at the begnning of 1st msec. Pre-emptive priority based scheduling
            is used. Priority is inverse to its priodicity. Highest priority task is executed first.
            1st instance of T3 complets at \fillin[time]. (GATE-2015\_set\_1 )

  \end{minipage}

\vspace{0.08in}

% ----------------------------------------------------------------------------
\begin{minipage}{\linewidth}

  \question  Which scheduling algorithm gives lowest average turnaround time? (GATE-2015\_set\_3 )
      \begin{center}
      \begin{tabular}{ c c c c }
          ID & Burst & Arrival\\
          p1 & 3  & 0 \\
          p2 & 6  & 1 \\
          p3 & 4  & 4 \\
          p4 & 2  & 6 \\
      \end{tabular}
    \end{center}

  \begin{choices}
    \choice FCFS
    \choice SJF non pre-emptive
    \choice SRTF
    \choice Round robin with time quanta 2.
  \end{choices}

  \end{minipage}

\vspace{0.08in}

% ----------------------------------------------------------------------------


\begin{minipage}{\linewidth}

  \question Pre-emptive SRTF scheduling algorithm is used. Average turn around time is \fillin[time]. (GATE-2016\_set\_2 )
      \begin{center}
      \begin{tabular}{ c c c c }
          ID & Burst & Arrival\\
          p1 & 10  & 0 \\
          p2 & 6  & 3 \\
          p3 & 1  & 7 \\
          p4 & 3  & 8 \\
      \end{tabular}
    \end{center}

  \end{minipage}

\vspace{0.08in}


% ----------------------------------------------------------------------------

\begin{minipage}{\linewidth}

  \question  Which of the following scheduling algorithm is non pre-emptive. (GATE-2002)

  \begin{choices}
    \choice Round Robin
    \choice FCFS
    \choice Multi-level queue
    \choice Multi-level queue with feedback
  \end{choices}

  \end{minipage}

\vspace{0.08in}

% ----------------------------------------------------------------------------
% ----------------------------------------------------------------------------
% ----------------------------------------------------------------------------
% ----------------------------------------------------------------------------
% ----------------------------------------------------------------------------

\begin{comment}

\begin{minipage}{\linewidth}

  \question  (GATE- )

  \begin{choices}
    \choice
    \choice
    \choice
    \choice
  \end{choices}

  \end{minipage}

\vspace{0.08in}


  %% oneparchoices

\end{comment}
