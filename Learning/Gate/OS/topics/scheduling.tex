
\centerline{\textbf{ \LARGE Scheduling Algorithms}}


\setcounter{question}{0}

\question Which policy is best for time sharing OS. (GATE - 95)

\begin{oneparchoices}
   \choice SJF
   \choice Round Robin
   \choice FCFS
   \choice SRTF
\end{oneparchoices}

\question State transition diagram represents . (GATE - 96)
\begin{choices}
   \choice Batch OS
   \choice OS with preemptive scheduling
   \choice OS with non-preemptive scheduling
   \choice Uni-programmed OS
\end{choices}

\question Total time taken in round robin algorithm for n process. Quantam size=q and context switch time=s  (GATE - 98)

\begin{oneparchoices}
   \choice n(s+q)
   \choice n(s-q)
   \choice n/(s+q)
   \choice n-(s*q)
\end{oneparchoices}


  \question Which scheduling algorithm provides maximum throughput. (GATE -2001 )

  \begin{oneparchoices}
    \choice Round Robin
    \choice SJF
    \choice FCFS
    \choice Priority based
  \end{oneparchoices}

  \question Which is non primitive. (GATE - 02)

  \begin{choices}
    \choice Round Robin
    \choice FCFS
    \choice Multi-level queue scheduling
    \choice Multi-level queue scheduling with feedback
  \end{choices}

  \question No of context switching in SRTF. bursts=[10,20,30] arrival=[0,2,6] (GATE - 2006)

  \begin{oneparchoices}
    \choice 1
    \choice 2
    \choice 3
    \choice 4
  \end{oneparchoices}

  \question Match the following. (GATE - 2007) \newline
   P). Gang Scheduling  Q). Rate Monotonic  R). Fair share Scheduling \newline
   1). Guranteed Scheduling 2). Real-time Scheduling 3). Thread Scheduling
  \begin{choices}
    \choice P-3, Q-2, R-1
    \choice P-1, Q-2, R-3
    \choice P-2, Q-3, R-1
    \choice P-1, Q-3, R-2
  \end{choices}

  \question Choose correct options (GATE - )

   \begin{enumerate}
      \item[I] SRTF causes starvation.
      \item[II] Premptive scheduling may cause starvation.
      \item[III] Round robin is bettre than FCFS in response time.
   \end{enumerate}

  \begin{choices}
    \choice I only
    \choice I, III only
    \choice II, III only
    \choice I, II, III
  \end{choices}

  \question  Given that priority is propotional to waiting time. After every T seconds priority is re-valuated.
             All process arrive at time zero with zero priority. (GATE - 2013)

  \begin{choices}
    \choice This algorithm is equivalant to FCFS.
    \choice This algorithm is equivalant to round robin.
    \choice This algorithm is equivalant to SJF.
    \choice This algorithm is equivalant to SRTF.
  \end{choices}

  \question  CPU-bound processes arrive at same time with un-equal bust.
            Which scheduling algorithm will minimize average waiting time. (GATE - 2016\_set\_1)

  \begin{choices}
    \choice SRTF
    \choice Largest Job First (opposite of SJF).
    \choice Iniform random
    \choice Round robin
  \end{choices}

\begin{comment}

  \question  (GATE - )

  \begin{choices}
    \choice
    \choice
    \choice
    \choice
  \end{choices}

  %% oneparchoices

\end{comment}
