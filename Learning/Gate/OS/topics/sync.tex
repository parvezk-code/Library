\centerline{\textbf{ \LARGE Temp}}

\setcounter{question}{0}




% ----------------------------------------------------------------------------

\begin{minipage}{\linewidth}

  \question  A critical section is a program segment (GATE-1996 )

  \begin{choices}
    \choice which should run in a certain specified amount of time
    \choice which avoids deadlocks
    \choice where shared resources are accessed
    \choice which must be enclosed by a pair of semaphore operations, P and V
  \end{choices}

  \end{minipage}

\vspace{0.08in}

% ----------------------------------------------------------------------------

\begin{minipage}{\linewidth}

  \question  When the result of a computation depends on the speed of the processes involved, there is said to be
             (GATE-1998 )

  \begin{oneparchoices}
    \choice cycle stealing
    \choice race condition
    \choice a time lock
    \choice a deadlock
  \end{oneparchoices}

  \end{minipage}

\vspace{0.08in}

% ----------------------------------------------------------------------------

\begin{minipage}{\linewidth}

  \question  Consider the methods used by processes P1 and P2 for accessing their critical sections whenever needed, as given below. The initial values of shared boolean variables S1 and S2 are randomly assigned. (GATE-2010)

  \begin{lstlisting}
      Method Used by P1                  Method Used by P2
      while (S1 == S2) ;                 while (S1 != S2) ;
      Critica1 Section                   Critica1 Section
      S1 = S2;                           S2 = not (S1);
  \end{lstlisting}

  \begin{choices}
    \choice Mutual exclusion but not progress
    \choice Progress but not mutual exclusion
    \choice Neither mutual exclusion nor progress
    \choice Both mutual exclusion and progress
  \end{choices}

  \end{minipage}

\vspace{0.08in}

% ----------------------------------------------------------------------------

\begin{minipage}{\linewidth}

  \question  The following two functions P1 and P2 that share a variable B with an initial
   value of 2 execute concurrently. The number of distinct values that B can possibly
   take after the execution is \fillin[]. (GATE-2015\_set\_2)

  \begin{lstlisting}
        P1()                  P2()
        {                     {
            C = B - 1;            D = 2 * B;
            B = 2*C;              B = D - 1;
        }                     }
  \end{lstlisting}

  \end{minipage}

\vspace{0.08in}

% ----------------------------------------------------------------------------

\begin{minipage}{\linewidth}

  \question  Two processes X and Y need to access a critical section. varP and varQ are shared variables and both are initialized to false. Which one of the following statements is true? (GATE-2015\_set\_2 )

  \begin{lstlisting}
      Process-X                             Process-Y
      while(true)                           while(true)
      {                                     {
          varP = true                           varQ = true
          while(varQ = true)                    while(varP = true)
          {                                     {
              CS                                    CS
              varP = false                          varQ = false
          }                                     }
      }                                     }
  \end{lstlisting}

  \begin{choices}
    \choice The proposed solution prevents deadlock but fails to guarantee mutual exclusion
    \choice The proposed solution guarantees mutual exclusion but fails to prevent deadlock
    \choice The proposed solution guarantees mutual exclusion and prevents deadlock
    \choice The proposed solution fails to prevent deadlock and fails to guarantee mutual exclusion
  \end{choices}

  \end{minipage}

\vspace{0.08in}

% ----------------------------------------------------------------------------

\begin{minipage}{\linewidth}

  \question  A critical region is: (GATE-1987 )

  \begin{choices}
    \choice One which is enclosed by a pair of P and V operations on semaphores.
    \choice A program segment that has not been proved bug-free.
    \choice A program segment that often causes unexpected system crashes.
    \choice A program segment where shared resources are accessed.
  \end{choices}

  \end{minipage}

\vspace{0.08in}

% ----------------------------------------------------------------------------

\begin{minipage}{\linewidth}

  \question A solution to the Dining Philosophers Problem which avoids deadlock is: (GATE-1996 )

  \begin{choices}
    \choice ensure that all philosophers pick up the left fork before the right fork
    \choice ensure that all philosophers pick up the right fork before the left fork
    \choice ensure that one particular philosopher picks up the left fork before the right fork, and that all other philosophers pick up the right fork before the left fork
    \choice None of the above
  \end{choices}

  \end{minipage}

\vspace{0.08in}

% ----------------------------------------------------------------------------

\begin{minipage}{\linewidth}

  \question Precidence Graph: (GATE-1996 )
  \end{minipage}

\vspace{0.08in}

% ----------------------------------------------------------------------------

\begin{minipage}{\linewidth}

  \question  Consider Peterson’s algorithm for processes i and j. The program executed by process i is shown below.(GATE-2001)

  \begin{lstlisting}
      repeat
          flag [i] = true;
          turn = j;

          while ( P ) do no-op;
            CS

          flag [ i ] = false;
          Perform other non-critical section actions.
      until false;
  \end{lstlisting}
  For the program to guarantee mutual exclusion, the predicate P in the while loop should be.
  \begin{choices}
    \choice flag[j] = true and turn = i
    \choice flag[j] = true and turn = j
    \choice flag[i] = true and turn = j
    \choice flag[i] = true and turn = i
  \end{choices}

  \end{minipage}

\vspace{0.08in}

% ----------------------------------------------------------------------------

\begin{minipage}{\linewidth}

  \question  Two processes, P1 and P2, need to access a critical section of code. Consider the following synchronization construct used by the processes:Here, wants1 and wants2 are shared variables, which are initialized to false. Which one of the following statements is TRUE about the above construct? (GATE- 2007)

  \begin{lstlisting}
      while (true)                            while (true)
      {                                       {
          wants1 = true;                          wants2 = true;
          while (wants2 == true);                 while (wants1 == true);
          CS                                      CS
          wants1=false;                           wants2=false;
      }                                       }
  \end{lstlisting}

  \begin{choices}
    \choice  It does not ensure mutual exclusion.
    \choice  It does not ensure bounded waiting.
    \choice  It requires that processes enter the critical section in strict alternation.
    \choice  It does not prevent deadlocks, but ensures mutual exclusion.
  \end{choices}

  \end{minipage}

\vspace{0.08in}

% ----------------------------------------------------------------------------

% ----------------------------------------------------------------------------

% ----------------------------------------------------------------------------

% ----------------------------------------------------------------------------

% ----------------------------------------------------------------------------

% ----------------------------------------------------------------------------

% ----------------------------------------------------------------------------

% ----------------------------------------------------------------------------

% ----------------------------------------------------------------------------

% ----------------------------------------------------------------------------

% ----------------------------------------------------------------------------



\begin{comment}

\begin{minipage}{\linewidth}

  \question  (GATE- )

  \begin{choices}
    \choice
    \choice
    \choice
    \choice
  \end{choices}

  \end{minipage}

\vspace{0.08in}

  %% oneparchoices


\end{comment}
