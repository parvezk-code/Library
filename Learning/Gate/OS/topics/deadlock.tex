

\setcounter{question}{0}

% ----------------------------------------------------------------------------

\begin{minipage}{\linewidth}

  \question  A computer has six tape drives, with n processes competing for them. Each process may need two drives.
             What is the maximum value of n for the system to be deadlock free?(GATE-1998)

  \begin{oneparchoices}
    \choice 6
    \choice 5
    \choice 4
    \choice 3
  \end{oneparchoices}

  \end{minipage}

\vspace{0.08in}

% ----------------------------------------------------------------------------

\begin{minipage}{\linewidth}

  \question  A system has 6 identical resources and N processes competing for them. Each process can request atmost 2 resources. Which one of the following values of N could lead to a deadlock? (GATE-2015 )

  \begin{oneparchoices}
    \choice 1
    \choice 2
    \choice 3
    \choice 4
  \end{oneparchoices}

  \end{minipage}

\vspace{0.08in}

% ----------------------------------------------------------------------------

\begin{minipage}{\linewidth}

  \question  A computer system has 6 tape drives with n processes competing for them. Each process may need 3 tape drives. The maximum value of n for which the system is guaranteed to be deadlock free is (GATE- 1992)

  \begin{oneparchoices}
    \choice 1
    \choice 2
    \choice 3
    \choice 4
  \end{oneparchoices}

  \end{minipage}

\vspace{0.08in}

% ----------------------------------------------------------------------------
\begin{minipage}{\linewidth}

  \question  Consider a system having m resources of the same type. These resources are shared by 3 processes, and  which have peak demands of 3, 4 and 6 respectively.  For what value of deadlock will not occur? (GATE- 1997)

  \begin{oneparchoices}
    \choice 7
    \choice 9
    \choice 10
    \choice 13
  \end{oneparchoices}

  \end{minipage}

\vspace{0.08in}

% ----------------------------------------------------------------------------

\begin{minipage}{\linewidth}

  \question  Which of the following is NOT a valid deadlock prevention scheme? (GATE- 2000)

  \begin{choices}
    \choice Release all resources before requesting a new resource
    \choice Number the resources uniquely and never request a lower numbered resource than the last one requested.
    \choice Never request a resource after releasing any resource
    \choice Request and all required resources be allocated before execution.
  \end{choices}

  \end{minipage}

\vspace{0.08in}

% ----------------------------------------------------------------------------

\begin{minipage}{\linewidth}

  \question  Suppose n processes, P1, …. Pn share m identical resource units, which can be reserved and released one at a time. The maximum resource requirement of process Pi is Si, where Si \(>\) 0. Which one of the following is a sufficient condition for ensuring that deadlock does not occur? (GATE-2005)

  \begin{choices}
    \choice {\large \(  \mathbf{ \sum_{n=1}^{n} s_i < (m+n)   }  \) }
    \choice {\large \(  \mathbf{ \sum_{n=1}^{n} s_i < (m * n) }  \) }
    \choice {\large \(  \mathbf{ \forall i, s_i < m  }  \) }
    \choice {\large \(  \mathbf{ \forall i, s_i < n  }  \) }
  \end{choices}

  \end{minipage}

\vspace{0.08in}

% ----------------------------------------------------------------------------

\begin{minipage}{\linewidth}

  \question  Consider the following snapshot of a system running n processes. Process i is holding \(X_i\) instances of a resource R, \( \mathbf{ 1 <= i <= n } \). currently, all instances of R are occupied. Further, for all i, process i has placed a request for an additional \(Y_i\) instances while holding the \(X_i\) instances it already has. There are exactly two processes p and q such that \( \mathbf{ Y_p = Y_q = 0 } \) . Which one of the following can serve as a necessary condition to guarantee that the system is not approaching a deadlock? (GATE-2006)

  \begin{choices}
    \choice {\large \( \mathbf { min(x_p, x_q)  <  max_{k!=p,q} \; y_k } \) }
    \choice {\large \( \mathbf { x_p+x_q >= min_{k!=p,q} \; y_k } \) }
    \choice {\large \( \mathbf { max(x_p, x_q) > 1 } \) }
    \choice {\large \( \mathbf { min(x_p, x_q) > 1 } \) }
  \end{choices}

  \end{minipage}

\vspace{0.08in}

% ----------------------------------------------------------------------------

\begin{minipage}{\linewidth}

  \question  Which of the following is NOT true of deadlock prevention and deadlock avoidance schemes? (GATE-2008)

  \begin{choices}
    \choice  In deadlock prevention, the request for resources is always granted if the resulting state is safe
    \choice  In deadlock avoidance, the request for resources is always granted if the result state is safe
    \choice  Deadlock avoidance is less restrictive than deadlock prevention
    \choice  Deadlock avoidance requires knowledge of resource requirements a priori
  \end{choices}

  \end{minipage}

\vspace{0.08in}

% ----------------------------------------------------------------------------

\begin{minipage}{\linewidth}

  \question  Consider a system with 4 types of resources R1 (3 units), R2 (2 units), R3 (3 units), R4 (2 units).
             A non-preemptive resource allocation policy is used. At any given instance, a request is not entertained
             if it cannot be completely satisfied. Three processes P1, P2, P3 request the sources as follows
             if executed independently. (GATE-2009)

\begin{lstlisting}

  Process P1:                   Process P2:               Process P3:
  t=0: requests 2               t=0: requests 2           t=0: requests 1
  units of R2                   units of R3               unit of R4

  t=1: requests 1               t=2: requests 1          t=2: requests 2
  unit of R3                    unit of R4               units of R1

  t=3: requests 2               t=4: requests 1          t=5: releases 2
  units of R1                   unit of R1               units of R1

  t=5: releases 1 unit          t=6: releases 1          t=7: requests 1
  of R2 and 1 unit of R1.       unit of R3               units of R2

  t=7: releases 1               t=8: Finishes            t=8: requests 1
  unit of R3                                             units of R3

  t=8: requests 2                                        t=9: Finishes
  units of R4

  t=10: Finishes

\end{lstlisting}

             Which one of the following statements is TRUE if all three processes run concurrently starting at time t=0?
  \begin{choices}
    \choice All processes will finish without any deadlock
    \choice Only P1 and P2 will be in deadlock.
    \choice Only P1 and P3 will be in a deadlock.
    \choice All three processes will be in deadlock
  \end{choices}

  \end{minipage}

\vspace{0.08in}

% ----------------------------------------------------------------------------

\begin{minipage}{\linewidth}

  \question  A system has n resources {\large \( { R_0,…,R_{n-1}  } \) },and k processes {\large \( { P_0,….P_{k-1} } \) } .The implementation of the resource request logic of each process \(P_i\) is as follows:  (GATE-2010)

  R(i+2) is actually \( R_{i+2} \) in following code...
  \begin{lstlisting}
      if (i % 2 == 0) {
          if (i < n) request R(i)
          if (i+2 < n) request R(i+2)
      }
      else {
          if (i < n) request R(n-i)
          if (i+2 < n) request R(n-i-2)
      }
  \end{lstlisting}

  In which one of the following situations is a deadlock possible?

  \begin{oneparchoices}
    \choice n=40, k=26
    \choice n=21, k=12
    \choice n=20, k=10
    \choice n=41, k=19
  \end{oneparchoices}

  \end{minipage}

\vspace{0.08in}

% ----------------------------------------------------------------------------

\begin{minipage}{\linewidth}

  \question  Consider the following policies for preventing deadlock in a system with mutually exclusive resources. (GATE-2015\_set\_3)

  \begin{enumerate}
      \item[I]  Processes should acquire all their resources at the beginning of execution.
                If any resource is not available, all resources acquired so far are released.
      \item[II] The resources are numbered uniquely, and processes are allowed to request
                for resources only in increasing resource numbers.
      \item[III] The resources are numbered uniquely, and processes are allowed to request
                  for resources only in decreasing resource numbers.
      \item[IV]  The resources are numbered uniquely. A process is allowed to request only for a
                  resource with resource number larger than its currently held resources.
  \end{enumerate}
  Which of the above policies can be used for preventing deadlock?

  \begin{choices}
    \choice Any one of I and III but not II or IV
    \choice Any one of I, III and IV but not II
    \choice Any one of II and III but not I or IV
    \choice Any one of I, II, III and IV
  \end{choices}

  \end{minipage}

\vspace{0.08in}

% ----------------------------------------------------------------------------

\begin{minipage}{\linewidth}

  \question  A multithreaded program P executes with x number of threads and uses y number of locks for ensuring
            mutual exclusion while operating on shared memory locations. All locks in the program are
            non-reentrant, i.e., if a thread holds a lock l, then it cannot re-acquire lock l without
            releasing it. If a thread is unable to acquire a lock, it blocks until the lock becomes available.
            The minimum value of x and the minimum value of y together for which execution of P can result
            in a deadlock are:(GATE-2017\_set\_1)

  \begin{choices}
    \choice x = 1, y = 2
    \choice x = 2, y = 1
    \choice x = 2, y = 2
    \choice x = 1, y = 1
  \end{choices}

  \end{minipage}

\vspace{0.08in}

% ----------------------------------------------------------------------------

old gate written questions of 5 mark  are left and not yet included in this library. Till 2002.

% ----------------------------------------------------------------------------

% ----------------------------------------------------------------------------

% ----------------------------------------------------------------------------


% ----------------------------------------------------------------------------

% ----------------------------------------------------------------------------

% ----------------------------------------------------------------------------

% ----------------------------------------------------------------------------


% ----------------------------------------------------------------------------

% ----------------------------------------------------------------------------

% ----------------------------------------------------------------------------

% ----------------------------------------------------------------------------




\begin{comment}

\begin{minipage}{\linewidth}

  \question  (GATE- )

  \begin{choices}
    \choice
    \choice
    \choice
    \choice
  \end{choices}

  \end{minipage}

\vspace{0.08in}

  %% oneparchoices


    \begin{lstlisting}     \end{lstlisting}

  \begin{enumerate}
      \item[I]
      \item[II]
   \end{enumerate}

   \fillin[]

   \(p_{n-1}\)


\end{comment}
