
\centerline{\textbf{ \LARGE Interupts, System calls, CPU Modes}}


% ----------------------------------------------------------------------------


  \begin{enumerate}

  \item CPU executes instructions.

  \item Modes and Instruction.
  \begin{enumerate}
        \item Dual Mode ensures protection and security from unauthorized users.
        \item Dual mode - User Mode and Kernel Mode.
        \item Instructions - privileged and non-privileged.
        \item Privileged instructions provide low-level system operations.
        \item Non-privileged instructions provide general purpose computing.
        \item Interupts cause CPU to change mode from user to kernel mode.
        \item A non-privileged instruction which do not cause interupt changes kernel mode to user mode.
  \end{enumerate}

  \item Privileged Instruction.
  \begin{enumerate}
        \item Executed only in Kernel Mode.
        \item Kernal and privileged process(ex. device driver) contains privileged instruction.
        \item Provides direct access to hardware or other privileged resources(ex. setting up memory mappings)
        \item Provides unrestricted access to the system resources.
        \item System calls contains privileged instruction.
        \item Executing privileged instruction in user mode is illegal. Hardware traps it in the OS.
        \item Privileged Instruction can modify the contents of the Timer.
  \end{enumerate}

  \item Non-Privileged Instruction.
  \begin{enumerate}
        \item Executed only in User Mode.
        \item Ensures that user processes do interfere.
        \item Provides access to user-level resources such as files and memory
        \item Provides direct access to hardware or other privileged resources(ex. setting up memory mappings)
        \item Provides unrestricted access to the system resources.
  \end{enumerate}{enumerate}

  \item Interupts
  \begin{enumerate}
        \item Types - Hardware Interupts and Software Interupts.
        \item ISR : Interupt Service Routine(interupt handler)
        \item Interupt Vector(IV) :
        \item Interupt Vector Table : Maps IV to Kernal Code.
  \end{enumerate}

  \item Interrupt Vector Table(IVT) and Interrupt Service Routine(ISR or interrupt handler)
  \begin{enumerate}
        \item Every interrupt has IS-Routine.
        \item IVT holds address of IS-Routine for each interupt.
        \item IS-Routine is part of Kernal. (Check this point)
  \end{enumerate}

  \item Interupt flow.
  \begin{enumerate}
        \item CPU finishes executing current execution cycle.
        \item CPU checks for interupt.(and detects one)
        \item CPU saves program status word onto stack.
        \item IS-Routine is identified based on IV-Table.
        \item Controll is transfered to IS-Routine.
        \item Restore saved registers from the stack.
        \item Restore ans executes the original process.
  \end{enumerate}


\end{enumerate}

% ----------------------------------------------------------------------------

% ----------------------------------------------------------------------------

% ----------------------------------------------------------------------------

% ----------------------------------------------------------------------------

% ----------------------------------------------------------------------------

% ----------------------------------------------------------------------------

% ----------------------------------------------------------------------------

% ----------------------------------------------------------------------------
