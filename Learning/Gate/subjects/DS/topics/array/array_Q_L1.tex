\centerline{\textbf{ \LARGE Array Level-1}}

\begin{questyle}
  \question In a compact one dimensional array representation for lower triangular matrix (all elements above diagonal are zero) of
            size n x n, non zero elements of each row are stored one after another, starting from first row, the index of (i,j)th
            element in this new representation is  (GATE-1994) \\
            Hint : 2D Array index start from 1. 1D Array index start from 0. Assume the values i=2, j=2 and solve the question.
  \begin{choices}
    \choice         i+j
    \choice         i+j-1
    \CorrectChoice  (j-1) + i(i-1)/2
    \choice         i + j(j-1)/2
  \end{choices}
\end{questyle}

\begin{questyle}
  \question  An n x n array v is defined as follows. The sum of the elements of the array v is.  (GATE-2000)
    \begin{lstlisting}
        v[i,j] = i-j    for all i,j
    \end{lstlisting}
  \begin{choices}
    \CorrectChoice  0
    \choice         n-1
    \choice         \(n^2\) – 3n + 2
    \choice         None of these
  \end{choices}
\end{questyle}

\begin{questyle}
  \question  Suppose you are given an array s[1..n] and a procedure reverse (s, i, j) which reverses the order of
             elements in a between positions i and j (both inclusive). What does the following sequence do, where 1 <= k <= n:  (GATE-2000)\\
             reverse(s, 1, k); \\
             reverse(s, k+1, n); \\
             reverse(s, 1, n);
  \begin{choices}
    \CorrectChoice  Rotates s left by k positions
    \choice         Leaves s unchanged
    \choice         Reverses all elements of s
    \choice         none
  \end{choices}
\end{questyle}

\begin{questyle}
  \question  First element of P[m][n] array is stored at location B. Starting row index is \( R_0 \) and column index is \( C_0 \).
            Address of P[i][j] if row-major order is used is. (GATE- 1998)\\
            \begin{myTableStyle}
                \begin{tabular}{ |m{4cm}|m{8cm}| } \hline
                    Total element count(C)  &     n(i - \( R_0 \)) + (j - \( C_0 \) + 1)         \\ \hline
                    Address of P[i][j]      &     B + (C - 1)    \\ \hline
                \end{tabular}
            \end{myTableStyle}
            \vspace{0.08in}

\end{questyle}


\begin{questyle}
  \question  First element of P[m][n] array is stored at location B. Starting row index is \( R_0 \) and column index is \( C_0 \).
            Address of P[i][j] if column-major order is used is. \\
            \begin{myTableStyle}
                \begin{tabular}{ |m{4cm}|m{8cm}| } \hline
                    Total element count(C)  &     m(j - \( C_0 \)) + (i - \( R_0 \) + 1)         \\ \hline
                    Address of P[i][j]      &     B + (C - 1)    \\ \hline
                \end{tabular}
            \end{myTableStyle}
            \vspace{0.08in}
\end{questyle}

\begin{questyle}
  \question  Two matrices M1 and M2 are to be stored in arrays A and B respectively. Each array can be stored
             either in row-major or column-major order in contiguous memory locations. The time complexity of an algorithm
             to compute M1 × M2 will be  (GATE-2004)

  \begin{choices}
    \choice         best if A is in row-major, and B is in column- major order
    \choice         best if both are in row-major order
    \choice         best if both are in column-major order
    \CorrectChoice  independent of the storage scheme
  \end{choices}
\end{questyle}
