
\centerline{\textbf{ \LARGE Binary Tree, BST Level-1}}

\begin{questyle}
  \question  Consider the following nested representation of binary trees: (X Y Z) indicates Y and Z
            are the left and right sub stress, respectively, of node X. Note that Y and Z may be NULL, or
            further nested. Which of the following represents a valid binary tree?  (GATE-2000)

  \begin{choices}
    \choice         (1 2 (4 5 6 7))
    \choice         (1 (2 3 4) 5 6) 7)
    \CorrectChoice  (1 (2 3 4)(5 6 7))
    \choice         (1 (2 3 NULL) (4 5))
  \end{choices}
\end{questyle}

\begin{questyle}
  \question  The following numbers are inserted into an empty binary search tree in the given
            order: 10, 1, 3, 5, 15, 12, 16. What is the height of the binary search tree
            (the height is the maximum distance of a leaf node from the root)?  (GATE-2004)

  \begin{oneparchoices}
    \choice         2
    \CorrectChoice  3
    \choice         4
    \choice         6
  \end{oneparchoices}
\end{questyle}

\begin{questyle}
  \question  While inserting the elements 71, 65, 84, 69, 67, 83 in an empty binary search tree (BST)
            in the sequence shown, the element in the lowest level is  (GATE-2016\_set\_3)

  \begin{oneparchoices}
    \choice         65
    \CorrectChoice  67
    \choice         69
    \choice         83
  \end{oneparchoices}
\end{questyle}

\begin{questyle}
  \question  A binary search tree is generated by inserting in order the following integers.
            The number of nodes in the left subtree and right subtree of the root respectively is. (GATE-1996)\\
            50, 15, 62, 5, 20, 58, 91, 3, 8, 37, 60, 24

  \begin{oneparchoices}
    \choice         (4, 7)
    \CorrectChoice  (7, 4)
    \choice         (8, 3)
    \choice         (3, 8)
  \end{oneparchoices}
\end{questyle}


\begin{questyle}
  \question  Let T be the sequence of keys encountered during search of item p in Binary Search Tree. Such that
             T = (S U G) and p \(\notin\) T. Where S be a set of keys smaller than p and G be a set of keys greater than p.
             Ordered sequence(T) should satisfy.

             I) keys of S should occur in decreasing order in T.\\
             II) keys of T should occur in increasing order in T. \\
             III) keys of S should occur in increasing order in T. \\
             IV) keys of T should occur in decreasing order in T. \\
  \begin{oneparchoices}
    \choice         I and II
    \choice         I and IV
    \CorrectChoice  III and IV
    \choice         II and III

  \end{oneparchoices}

\end{questyle}

\begin{questyle}
  \question  Let T be the sequence of keys encountered during search of item p in Binary Search Tree. Such that
             T = (S U G) and p \(\notin\) T. Where S be a set of keys smaller than p and G be a set of keys greater than p.
             How many different ordered sequence are possible.

  \begin{oneparchoices}
    \choice         \(\mid S \mid * \mid P \mid \)
    \choice         \( \Myperm[\mid T \mid]{\mid G \mid} \)
    \CorrectChoice  \( \Mycomb[\mid T \mid]{\mid S \mid} \)
    \choice         \( \Myperm[\mid T \mid]{\mid S \mid} \)
  \end{oneparchoices}
\end{questyle}


\begin{questyle}
  \question  We are given a set of n distinct elements and an unlabelled binary tree with n
            nodes. In how many ways can we populate the tree with the given set so that it becomes a
            binary search tree?  (GATE-2011)

  \begin{oneparchoices}
    \choice         0
    \CorrectChoice  1
    \choice         \( {n!} \)
    \choice         \( \frac{1}{n+1} \Mycomb[2n]{n} \)
  \end{oneparchoices}
\end{questyle}

\begin{questyle}
  \question  We are given a set of n distinct elements. How many binary search trees are possible?

  \begin{oneparchoices}
    \choice         0
    \choice         1
    \choice         \( {n!} \)
    \CorrectChoice  \( \frac{1}{n+1} \Mycomb[2n]{n} \)
  \end{oneparchoices}
\end{questyle}


\begin{questyle}
  \question Given a binary search tree with n distinct key and hight (n-1). Note: The height
            of a tree with a single node is 0. Choose the correct option.
    \begin{enumerate}
    \item[I]  Binary tree forms a chain, that is there are no branches.
    \item[II] Only maximum or the minimum key can be the root node.
    \item[III] Only maximum or the minimum key can be the leaf node.
    \item[IV] Number of such binary trees are { \Large \( 2^{(n-1)}\) }
    \item[V] Number of such binary trees are \( \frac{1}{n+1} \Mycomb[2n]{n} \)
    \item[VI] Every internal(non leaf) node has exactly one child.
    \end{enumerate}

  \begin{oneparchoices}
    \choice         I, II and V
    \choice         II, III
    \choice         I, III, IV
    \CorrectChoice  I, II, IV and VI
  \end{oneparchoices}
\end{questyle}
