

\centerline{\textbf{ \LARGE Traversal of BST}}

\begin{questyle}
  \question  Give a BST of unique keys. Order of Keys in inorder traversal is.
  \begin{choices}
    \choice         always decresing.
    \CorrectChoice  always increasing.
    \choice         Can be increasing or decreasing.
    \choice         can not be increasing.
  \end{choices}
\end{questyle}


\begin{questyle}
  \question  Give a BST of unique keys. Preorder traversal sequence(s) of BST is of the form
            \begin{hl} {root(L)(R)}\end{hl}. Where L and R are sequence of keys. Choose the correct option.
    \begin{enumerate}
        \item[I] Keys in L should be greater than root
        \item[II] Keys in L should be smaller than root
        \item[III] Keys in R should be smaller than root
        \item[IV] Keys in R should be greater than root
    \end{enumerate}

  \begin{choices}
    \choice         I and III
    \choice         I and IV
    \CorrectChoice  II and IV \qquad This is true for post order traversal also. \begin{hl} {(L)(R)root}\end{hl}
    \choice         II and III
  \end{choices}
\end{questyle}


\begin{questyle}
  \question  Which of the following is/are correct inorder traversal sequence(s) of binary search tree(s)?
            (GATE-2015\_set\_1)
    \begin{enumerate}
        \item[I] 3, 5, 7, 8, 15, 19, 25
        \item[II] 5, 8, 9, 12, 10, 15, 25
        \item[III] 2, 7, 10, 8, 14, 16, 20
        \item[IV] 4, 6, 7, 9, 18, 20, 25
    \end{enumerate}

  \begin{choices}
    \CorrectChoice  I and IV only
    \choice         II and III only
    \choice         II and IV only
    \choice         II only
  \end{choices}
  Hint : Inorder of a BST must be in increasing order. Inorder of a Binary Tree may not be in increasing order.
\end{questyle}


\begin{questyle}
  \question  A binary search tree contains the values 1, 2, 3, 4, 5, 6, 7, 8. The tree is traversed
            in pre-order and the values are printed out. Which of the following sequences is a valid output?  (GATE-1997)
  \begin{choices}
    \choice         53124786
    \choice         53126487
    \choice         53241678
    \CorrectChoice  53124768  \qquad { root(L)(R) \qquad L \(<\) root \(<\)R }
  \end{choices}

  \begin{enumerate}
        \item[a] Hint: recursively apply - root(L)(R) and check if condition (L\(<\)root\(<\)R) is satisfied.
        \item[b] 53124768 : 5(L=3124)(R=768)
        \item[c] 3124 : 3(L=12)(R=4)
        \item[d] 768 : 7(L=6)(R=8)
        \item[e] 12 : 1(L=null)(R=2)
  \end{enumerate}
\end{questyle}


\begin{questyle}
  \question  Postorder traversal of a given binary search tree T produces the sequence
             10, 9, 23, 22, 27, 25, 15, 50, 95, 60, 40, 29. Which one of the following sequences
             of keys can be the result of an inorder traversal of the tree T? (GATE-2005)
  \begin{choices}
    \CorrectChoice  9, 10, 15, 22, 23, 25, 27, 29, 40, 50, 60, 95
    \choice         9, 10, 15, 22, 40, 50, 60, 95, 23, 25, 27, 29
    \choice         29, 15, 9, 10, 25, 22, 23, 27, 40, 60, 50, 95
    \choice         95, 50, 60, 40, 27, 23, 22, 25, 10, 9, 15, 29
  \end{choices}
  Hint : Inorder traversal always produce sequence in increasing order.
\end{questyle}

\begin{questyle}
  \question  A binary search tree contains the numbers 1, 2, 3, 4, 5, 6, 7, 8. When the tree is
            traversed in pre-order and the values in each node printed out, the sequence of values
            obtained is 5, 3, 1, 2, 4, 6, 8, 7. If the tree is traversed in post-order, the sequence
            obtained would be  (GATE-2005)

  \begin{choices}
    \choice         8, 7, 6, 5, 4, 3, 2, 1
    \choice         1, 2, 3, 4, 8, 7, 6, 5
    \choice         2, 1, 4, 3, 6, 7, 8, 5
    \CorrectChoice  2, 1, 4, 3, 7, 8, 6, 5
  \end{choices}
\end{questyle}

\begin{questyle}
  \question  A Binary Search Tree (BST) stores values in the range 37 to 573. Consider the following
             sequence of keys. Which of the following statements is TRUE? (GATE-2008)
    \begin{enumerate}
    \item[I] 81, 537, 102, 439, 285, 376, 305
    \item[II] 52, 97, 121, 195, 242, 381, 472
    \item[III] 142, 248, 520, 386, 345, 270, 307
    \item[IV] 550, 149, 507, 395, 463, 402, 270
    \end{enumerate}

  \begin{choices}
    \choice         I, II and IV are inorder sequences of three different BSTs
    \choice         I is a preorder sequence of some BST with 439 as the root
    \CorrectChoice  II is an inorder sequence of some BST where 121 is the root and 52 is a leaf
    \choice         IV is a postorder sequence of some BST with 149 as the root
  \end{choices}
\end{questyle}

\begin{questyle}
  \question  You are given the postorder traversal, P, of a binary search tree on the n elements
             1, 2, ... n. You have to determine the unique binary search tree that has P as its postorder
             traversal. What is the time complexity of the most efficient algorithm for doing this?  (GATE-2008)

  \begin{choices}
    \choice         O(Log n)
    \CorrectChoice  O(n) \qquad Hint: Re-create BST in O(n). Then perform postorder in O(n) time.
    \choice         O(n Logn)
    \choice         none of the above, as the tree cannot be uniquely determined.
  \end{choices}
\end{questyle}

\begin{questyle}
  \question  The preorder traversal sequence of a binary search tree is 30, 20, 10, 15, 25, 23, 39, 35, 42.
            Which one of the following is the postorder traversal sequence of the same tree?  (GATE-2013)

  \begin{choices}
    \choice         10, 20, 15, 23, 25, 35, 42, 39, 30
    \choice         15, 10, 25, 23, 20, 42, 35, 39, 30
    \choice         15, 20, 10, 23, 25, 42, 35, 39, 30
    \CorrectChoice  15, 10, 23, 25, 20, 35, 42, 39, 30
  \end{choices}
\end{questyle}


\begin{questyle}
  \question  The pre-order traversal of a binary search tree is given by
             12, 8, 6, 2, 7, 9, 10, 16, 15, 19, 17, 20. Then the post-order traversal of this tree is:  (GATE-2017\_Set\_2)

  \begin{choices}
    \choice         2, 6, 7, 8, 9, 10, 12, 15, 16, 17, 19, 20
    \CorrectChoice  2, 7, 6, 10, 9, 8, 15, 17, 20, 19, 16, 12
    \choice         7, 2, 6, 8, 9, 10, 20, 17, 19, 15, 16, 12
    \choice         7, 6, 2, 10, 9, 8, 15, 16, 17, 20, 19, 12
  \end{choices}
\end{questyle}



\begin{comment}

\begin{questyle}
  \question  zzz  (GATE-zzz)

  \begin{choices}
    \choice         zzz
    \choice         zzz
    \choice         zzz
    \choice         zzz
    \CorrectChoice
  \end{choices}
\end{questyle}

\begin{questyle}
  \question  zzz  (GATE-zzz)

  \begin{choices}
    \choice         zzz
    \choice         zzz
    \choice         zzz
    \choice         zzz
    \CorrectChoice
  \end{choices}
\end{questyle}


\begin{questyle}
  \question  zzz  (GATE-zzz)

  \begin{choices}
    \choice         zzz
    \choice         zzz
    \choice         zzz
    \choice         zzz
    \CorrectChoice
  \end{choices}
\end{questyle}

\begin{questyle}
  \question  zzz  (GATE-zzz)

  \begin{choices}
    \choice         zzz
    \choice         zzz
    \choice         zzz
    \choice         zzz
    \CorrectChoice
  \end{choices}
\end{questyle}

\begin{questyle}
  \question  zzz  (GATE-zzz)

  \begin{choices}
    \choice         zzz
    \choice         zzz
    \choice         zzz
    \choice         zzz
    \CorrectChoice
  \end{choices}
\end{questyle}


\end{comment}
