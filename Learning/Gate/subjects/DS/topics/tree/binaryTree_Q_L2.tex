
\centerline{\textbf{ \LARGE Binary Tree, BST Level-2}}

\begin{questyle}
  \question  The numbers 1, 2, ... n are inserted in a binary search tree in some order. In the resulting
             tree, the right subtree of the root contains p nodes. The first number to be inserted in
             the tree must be  (GATE-2005)

  \begin{oneparchoices}
    \choice         p
    \choice         p + 1
    \CorrectChoice  n – p
    \choice         n – p + 1
  \end{oneparchoices}
\end{questyle}


\begin{questyle}
  \question  Suppose that we have numbers between 1 and 100 in a binary search tree and want to
            search for the number 55. Which of the following sequences CANNOT be the sequence
            of nodes examined?  (GATE-2006)

  \begin{choices}
    \choice         {10, 75, 64, 43, 60, 57, 55}
    \choice         {90, 12, 68, 34, 62, 45, 55}
    \CorrectChoice  {9, 85, 47, 68, 43, 57, 55}
    \choice         {79, 14, 72, 56, 16, 53, 55}
  \end{choices}
  Hint : No smaller than 55 should be in increasing order. No greater than 55 should be in decreasing order
\end{questyle}

\begin{questyle}
  \question  When searching for the key value 60 in a binary search tree, nodes containing
            the key values 10, 20, 40, 50, 70 80, 90 are traversed, not necessarily in the order given.
            How many different orders are possible in which these key values can occur on the
            search path from the root to the node containing the value 60?  (GATE-2007)

  \begin{oneparchoices}
    \CorrectChoice  35
    \choice         64
    \choice         128
    \choice         5040
  \end{oneparchoices}

  Hint: 4 no smaller than 60 and 3 no greater than 60. \( \Mycomb[\mid 7 \mid]{\mid 4 \mid} \) or \( \Mycomb[\mid 7 \mid]{\mid 3 \mid} \)
\end{questyle}

\begin{questyle}
  \question  A Binary Search Tree (BST) stores values in the range 37 to 573. Consider the following
             sequence of keys. Suppose the BST has been unsuccessfully searched for key 273. Which all
             of the above sequences list nodes in the order in which we could have encountered them
             in the search? (GATE-2008)
    \begin{enumerate}
    \item[I] 81, 537, 102, 439, 285, 376, 305
    \item[II] 52, 97, 121, 195, 242, 381, 472
    \item[III] 142, 248, 520, 386, 345, 270, 307
    \item[IV] 550, 149, 507, 395, 463, 402, 270
    \end{enumerate}
  \begin{choices}
    \choice         II and III only
    \choice         I and III only
    \choice         III and IV only
    \CorrectChoice  III only
  \end{choices}
\end{questyle}

\begin{questyle}
  \question  The number of ways in which the numbers 1, 2, 3, 4, 5, 6, 7 can be inserted in an
             empty binary search tree, such that the resulting tree has height 6, is \fillin[ 64 (\(2^6\))] (GATE-2016\_Set\_2).\\
             Note: The height of a tree with a single node is 0.
\end{questyle}

        Hint : n nodes and hight (n-1) forms a chain. Threre are no branches in the tree.
        Every node has one child except for the last one. There are 6 internal node and one leaf node.
        At every internal(non leaf) node we have 2 options either to choose right child or left child.
