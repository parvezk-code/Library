
\centerline{\textbf{ \LARGE Tree Programs}}

\begin{questyle}
  \question  The value returned by the function DoSomething when a pointer to the root of a non-empty
             tree is passed as argument is  (GATE-2004)
    \lstinputlisting[language=C, firstline=4, lastline=22]{tree_programs_code.c}

  \begin{choices}
    \choice         The number of leaf nodes in the tree
    \choice         The number of nodes in the tree
    \choice         The number of internal nodes in the tree
    \CorrectChoice  The height of the tree
  \end{choices}
  Hint : Hight(node) = 1 + max( hight of left child, hight of right child).  Always check what if returned by leaf nodes, internal nodes.
\end{questyle}

\begin{questyle}
  \question  The value returned by GetValue() when a pointer to the root of a binary tree is passed as its argument is.  (GATE-2007)
    \lstinputlisting[language=C, firstline=30, lastline=48]{tree_programs_code.c}

  \begin{choices}
    \choice         the number of nodes in the tree
    \choice         the number of internal nodes in the tree
    \CorrectChoice  the number of leaf nodes in the tree
    \choice         the height of the tree
  \end{choices}
  Hint : LeafCount(node) = LeafCount(left tree) + LeafCount(right tree). Always check what if returned by leaf nodes, internal nodes.
\end{questyle}

\begin{questyle}
  \question  The height of a tree is defined as the number of edges on the longest path in the tree. The function
              shown in the pseudocode below is invoked as height (root) to compute the height of a binary tree
              rooted at the tree pointer root.   (GATE-2012)
            \lstinputlisting[language=C, firstline=57, lastline=73]{tree_programs_code.c}
  \begin{choices}
    \CorrectChoice  B1 : (1 + height(n \(\rightarrow\) right)), \qquad B2 : (1 + max(h1,h2))
    \choice         B1 : (height(n \(\rightarrow\) right)), \qquad B2 : (1 + max(h1,h2))
    \choice         B1 : height(n \(\rightarrow\) right), \qquad B2 : max(h1,h2)
    \choice         B1 : (1 + height(n \(\rightarrow\) right)), \qquad B2 : max(h1,h2)
  \end{choices}
\end{questyle}

\begin{questyle}
  \question  Consider the pseudocode given below. The function DoSomething() takes as argument a pointer
             to the root of an arbitrary tree represented by the leftMostChild-rightSibling representation. When
             the pointer to the root of a tree is passed as the argument to DoSomething, the value returned by
             the function (GATE-2014\_set\_3)
              \lstinputlisting[language=C, firstline=57, lastline=73]{tree_programs_code.c}
  \begin{choices}
    \choice         number of internal nodes in the tree.
    \choice         height of the tree.
    \choice         number of nodes without a right sibling in the tree.
    \CorrectChoice  number of leaf nodes in the tree.
  \end{choices}
\end{questyle}
