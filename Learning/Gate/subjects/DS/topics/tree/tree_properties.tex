
\centerline{\textbf{ \LARGE Tree Properties}}

\leftline{\textbf{ \Large Tree Count }}

\question  Left-Child Right Sibling Representation of a tree.

\begin{questyle}
  \question  The maximum number of binary trees that can be formed with three unlabeled nodes is:  (GATE-2007)

  \begin{oneparchoices}
    \choice         1
    \CorrectChoice  5
    \choice         4
    \choice         3  \qquad Hint : count = \Large{ \(  \frac {\Mycomb[2n]{n}}{n+1} \) }
  \end{oneparchoices}

\end{questyle}

\begin{questyle}
  \question  How many distinct binary search trees can be created out of 4 distinct keys  (GATE-2005)

  \begin{oneparchoices}
    \choice         5
    \CorrectChoice  14
    \choice         24
    \choice         42
  \end{oneparchoices}
\end{questyle}

\begin{questyle}
  \question  How many distinct BSTs can be constructed with 3 distinct keys?  (GATE-2008)

  \begin{oneparchoices}
    \choice         4
    \CorrectChoice  5
    \choice         6
    \choice         9
  \end{oneparchoices}
\end{questyle}


\begin{questyle}
  \question  An array X of n distinct integers is interpreted as a complete binary tree. The index
             of the first element of the array is 0. The index of the parent of element X[i], i!=0 is?  (GATE-2006)

  \begin{oneparchoices}
    \choice         \( \left \lfloor \dfrac i 2 \right \rfloor \)
    \choice         \( \left \lceil \dfrac{i-1}{2} \right \rceil \)
    \choice         \( \left \lceil \dfrac i 2 \right \rceil \)
    \CorrectChoice  \( \left \lceil \dfrac i 2 \right \rceil - 1 \)
  \end{oneparchoices}
    \vspace{0.08in}
    \\Hint : (0, i, 2i+1, 2i+2)      (1, i, 2i, 2i+1)
\end{questyle}



\leftline{\textbf{ \Large Hight}}

  \question Hight(node) = 1 + max( hight of left tree, hight of right tree) \\

\begin{questyle}
  \question  The height of a binary tree is the maximum number of edges in any root to leaf path.
              The maximum number of nodes in a binary tree of height h is:  (GATE-2007)

  \begin{oneparchoices}
    \choice         \( 2^h - 1 \)
    \choice         \( 2^{h-1} - 1 \)
    \CorrectChoice  \( 2^{h+1} - 1 \)
    \choice         \( 2^{h+1}\)
  \end{oneparchoices}
  hint : Assume value for h=0, 1, 2 and match the option
\end{questyle}


\begin{questyle}
  \question  An array X of n distinct integers is interpreted as a complete binary tree. The
            index of the first element of the array is 0. If the root node is at level 0, the level of
            element X[i], i!=0, is  (GATE-2006)

  \begin{oneparchoices}
    \choice         \( \left \lfloor log_2i \right \rfloor \)
    \choice         \( \left \lceil log_2{(i+1)} \right \rceil \)
    \CorrectChoice  \( \left \lfloor log_2{(i+1)} \right \rfloor \)
    \choice         \( \left \lceil log_2{i} \right \rceil \)
  \end{oneparchoices}
\end{questyle}


\leftline{\textbf{ \Large Nodes Count }}

\question LeafCount(node) = LeafCount(left tree) + LeafCount(right tree)

\question Hight of root = 0.

\question Maximum and minimum nodes in Binary Tree for height(h) = (\(2^{h+1} -1\)) and (h+1)

\begin{questyle}
  \question  If the number of leaves in a tree is not a power of 2 then the tree is not a binary tree.  (GATE-1997)
\end{questyle}

  \begin{questyle}

  \question  A binary tree T has n leaf nodes. The number of nodes of degree 2 in T is  (GATE-1995)

  \begin{oneparchoices}
    \choice         \(log_2n\)
    \CorrectChoice  n-1
    \choice         n
    \choice         \(2^n\)
  \end{oneparchoices}

  \vspace{0.08in}
  \begin{myTableStyle} \begin{tabular}{ |m{2cm}|m{3cm}|m{4cm}|m{2cm}| } \hline
          Binary Tree & \( N= n_0 + n_1 + n_2 \)  &  \( E(N-1) =  n_1 + 2.n_2 \) &  \( n_0 =  n_2 + 1 \)  \\ \hline
  \end{tabular} \end{myTableStyle} \vspace{0.08in}

\end{questyle}


\begin{questyle}
  \question  Which of the following statement is false?  (GATE-1998)

  \begin{choices}
    \choice         A tree with n nodes has (n-1) edges.
    \CorrectChoice  A labeled rooted binary tree can be uniquely constructed with postorder and preorder traversal.
    \CorrectChoice  A complete binary tree with n internal nodes has (n+1) leaves.
    \choice         The maximum number of nodes in a binary tree of height h is (\(2^{h+1} -1\)). Root is at height 0.
  \end{choices}
\end{questyle}

\begin{questyle}
  \question  For complete binary tree, which of the following statement is false? (GATE-1998)

  \begin{choices}
    \choice         If total nodes is n, then height of the tree is \fillin[].
    \choice         All the levels except the last level are completely full.
    \choice         There can be atmost 1 internal node with one child, ie atmost one node can have one child.
    \choice         All the levels of the tree are filled completely except the lowest level.
    \CorrectChoice  A complete binary tree with n internal nodes has (n+1) leaves.
    \choice         If height of tree is h, then leaves can exist at height h or (h-1).
    \choice         For k leaves(\(n_0\)), number of nodes in tree is either 2k or 2k-1.
  \end{choices}
\end{questyle}


\begin{questyle}
  \question  A scheme for storing binary trees in an array X is as follows. Indexing of X
            starts at 1 instead of 0. the root is stored at X[1]. For a node stored at X[i], the
            left child, if any, is stored in X[2i] and the right child, if any, in X[2i+1]. To
            be able to store any binary tree on n vertices the minimum size of X should be.  (GATE-2006)

  \begin{oneparchoices}
    \choice         \(log_2n\)
    \choice         n
    \choice         2n + 1
    \CorrectChoice  \(2^n - 1\)
  \end{oneparchoices}

  Hint: Assume a tree with 1, 2, 3 nodes and check which option is correct.\\
  Hint: When n-nodes form chain, then height=n-1

    \begin{myTreeLThree}
      \node [rectangle,draw] [red] (node_root){\(2^0\)}
        child
        {
            node [circle,draw] (node_l1) {null}
        }
        child
        {
            node [rectangle,draw] (node_d){ \(2^1 + 2^0\) }
            child
            {
                node [circle,draw] (node_l2) {null}
            }
            child
            {
                node [rectangle,draw] (node_d){ \(2^2 + 2^1 + 2^0\) }
            }
        };
    \end{myTreeLThree}

\end{questyle}

\begin{questyle}
  \question  In a binary tree, the number of internal nodes of degree 1 is 5, and the number of
             internal nodes of degree 2 is 10. The number of leaf nodes in the binary tree is  (GATE-2006)

  \begin{oneparchoices}
    \choice         10
    \CorrectChoice  11
    \choice         12
    \choice         15
  \end{oneparchoices}
   \\ Hint :  \( n_0 =  n_2 + 1 \)
  \end{questyle}

\begin{questyle}
  \question  The height of a tree is the length of the longest root-to-leaf path in it. The
             maximum and minimum number of nodes in a binary tree of height 5 are  (GATE-2015\_Set\_1)

  \begin{oneparchoices}
    \CorrectChoice  63 and 6
    \choice         64 and 5
    \choice         32 and 6
    \choice         31 and 5
  \end{oneparchoices}
\end{questyle}


\begin{questyle}
  \question  A binary tree T has 20 leaves. The number of nodes in T having two children is \fillin[19]. (GATE-2015\_set\_2)
            (GATE-2015\_set\_3) \\ Hint :  \( n_0 =  n_2 + 1 \)
\end{questyle}

\begin{questyle}
  \question  Let T be a binary search tree with 15 nodes. The minimum and maximum possible heights of T are
             (GATE-2017\_set\_1)(GATE-2015\_Set\_1)

  \begin{oneparchoices}
    \choice         4 and 15
    \CorrectChoice  3 and 14
    \choice         4 and 14
    \choice         3 and 15
  \end{oneparchoices}
\end{questyle}


\begin{questyle}
  \question  A complete n-ary tree is one in which every node has 0 or n sons. If x is the
             number of internal nodes of a complete n-ary tree, the number of leaves in it is given by.  (GATE-1998, 2005)

  \begin{oneparchoices}
    \CorrectChoice  x(n-1)+1
    \choice         xn-1
    \choice         xn+1
    \choice         x(n+1)
  \end{oneparchoices}
\end{questyle}

\begin{questyle}
  \question  A complete n-ary tree is a tree in which each node has n children or no children.
            Let I be the number of internal nodes and L be the number of leaves in a complete
            n-ary tree. If L = 41, and I = 10, what is the value of n? (GATE-2007)

  \begin{oneparchoices}
    \choice         3
    \choice         4
    \CorrectChoice  5
    \choice         6
  \end{oneparchoices}
\end{questyle}



\begin{questyle}
  \question  The number of leaf nodes in a rooted tree of n nodes, with each node having 0 or 3 children is:  (GATE-2002)

  \begin{oneparchoices}
    \choice         n/2
    \choice         (n – 1)/3
    \choice         (n – 1)/2
    \CorrectChoice  (2n + 1)/3
  \end{oneparchoices}
\end{questyle}


\begin{questyle}
  \question  In a binary tree, for every node the difference between the number of nodes in the left and right subtrees is at most 2. If the height of the tree is h \( \textgreater \) 0, then the minimum number of nodes in the tree is:  (GATE-2005)

  \begin{oneparchoices}
    \choice         \(2^{h-1} \)
    \CorrectChoice  \(2^{h-1} + 1 \)
    \choice         \(2^{h-1} \)
    \choice         \(2^{h} \)
  \end{oneparchoices} \\
  hint : Assume value for h=0, 1, 2 and match the option. T(h) = 1 + T(h-1) + \{T(h-1) - 2\} = 2T(h-1) - 1
\end{questyle}


