
\centerline{\textbf{ \LARGE Tree Traversal Level-2}}

\begin{questyle}
  \question  Level order traversal of a rooted tree can be done by starting from the root and performing  (GATE-2004)

  \begin{choices}
    \choice         preorder traversal
    \choice         inorder traversal
    \choice         depth first search
    \CorrectChoice  breadth first search
  \end{choices}
\end{questyle}

\begin{questyle}
  \question  Consider the following rooted tree with the vertex P labeled as root.The order
            in which the nodes are visited during in-order traversal is  (GATE-2014\_Set\_3)

\begin{myTree}
  \node [circle,draw] [red] (node_p){P}
    child
    {
        node [circle,draw] (node_q) {Q}
        child
        {
            node [circle,draw] (node_s) {S}
        }
    }
    child
    {
        node [circle,draw] (node_r){R}
        child
        {
            node [circle,draw] (node_t) {T}
        }
        child
        {
            node [circle,draw] (node_u) {U}
            child
            {
              node [circle,draw] (node_w) {W}
            }
        }
        child
        {
            node [circle,draw] (node_v) {V}
        }
    };
\end{myTree}

  \begin{oneparchoices}
    \CorrectChoice  SQPTRWUV
    \choice         SQPTURWV
    \choice         SQPTWUVR
    \choice         SQPTRUWV
  \end{oneparchoices}
\end{questyle}


\begin{questyle}
  \question  Let LASTPOST, LASTIN and LASTPRE denote the last vertex visited in a postorder,
            inorder and preorder traversal, respectively, of a complete binary tree. Which of
            the following is always true?  (GATE-2000) \vspace{0.18in}

\begin{myTableStyle}
    \begin{tabular}{ |m{5.5cm}|m{4cm}|m{4cm}| } \hline \vspace{0.18in}

    \begin{choices}
      \choice         LASTIN = LASTPOST
      \choice         LASTIN = LASTPRE
      \choice         LASTPRE = LASTPOST
      \CorrectChoice  None of the above
    \end{choices}
    &
          \begin{myTree}
              \node [circle,draw] [red] (node_p){50}
              child{
                      node [circle,draw] [blue] (node_q){30}
              }
              child{
                      node [circle,draw] [black] (node_r){null}
              };
         \end{myTree} &
          \begin{myTree}
              \node [circle,draw] [red] (node_p){50}
              child{
                      node [circle,draw] [black] (node_r){null}
              }
              child{
                      node [circle,draw] [blue] (node_q){30}
              };
         \end{myTree} \\ \hline

    \end{tabular}
\end{myTableStyle}
\vspace{0.08in}

\end{questyle}


\begin{questyle}
  \question  Which one of the following binary trees has its inorder and preorder traversals as
             BCAD  and ABCD, respectively?  (GATE-2004)

             \begin{myTree}
              \node [circle,draw] [red] (node_a){A}
              child{
                      node [circle,draw] [blue] (node_b){B}
                      child{ node [circle,draw] [black] (node_x){NULL} }
                      child{ node [circle,draw] [black] (node_c){C} }
              }
              child{
                      node [circle,draw] [black] (node_d){D}
              };
         \end{myTree}

  \begin{oneparchoices}
    \choice         A
    \choice         B
    \choice         C
    \CorrectChoice  D
  \end{oneparchoices}
\end{questyle}


\begin{questyle}
  \question  Consider the label sequences obtained by the following pairs of traversals on a
             labeled binary tree. Which of these pairs identify a tree uniquely  (GATE-2004)

  \begin{choices}
    \choice         preorder and postorder
    \CorrectChoice  inorder and postorder
    \CorrectChoice  preorder and inorder
    \choice         level order and postorder
  \end{choices}
\end{questyle}


\begin{questyle}
  \question  The inorder and preorder traversal of a binary tree are d b e a f c g and
             a b d e c f g, respectively. The postorder traversal of the binary tree is:  (GATE-2007)

  \begin{choices}
    \CorrectChoice  d e b f g c a
    \choice         e d b g f c a
    \choice         e d b f g c a
    \choice         d e f g b c a
  \end{choices}

  Hint: for inorder assume - d-1 b-2 e-3 a-4 f-5 c-6 g-7, so preorder is 4 2 1 3 6 5 7
\end{questyle}


\begin{questyle}
  \question  The following three are known to be the preorder, inorder and postorder sequences
             of a binary tree. But it is not known which is which.  (GATE-2008) \\
             I) MBCAFHPYK \qquad II) KAMCBYPFH \qquad III) MABCKYFPH

  \begin{choices}
    \choice         I and II are preorder and inorder sequences, respectively
    \choice         I and III are preorder and postorder sequences, respectively
    \choice         II is the inorder sequence, but nothing more can be said about the other two sequences
    \CorrectChoice  II and III are the preorder and inorder sequences, respectively.
  \end{choices}
   Hint : preorder(first key) =  postorder(last key) = root
\end{questyle}


\begin{comment}


\begin{questyle}
  \question  zzz  (GATE-zzz)

  \begin{choices}
    \choice         zzz
    \choice         zzz
    \choice         zzz
    \choice         zzz
    \CorrectChoice
  \end{choices}
\end{questyle}

\begin{questyle}
  \question  zzz  (GATE-zzz)

  \begin{choices}
    \choice         zzz
    \choice         zzz
    \choice         zzz
    \choice         zzz
    \CorrectChoice
  \end{choices}
\end{questyle}


\begin{questyle}
  \question  zzz  (GATE-zzz)

  \begin{choices}
    \choice         zzz
    \choice         zzz
    \choice         zzz
    \choice         zzz
    \CorrectChoice
  \end{choices}
\end{questyle}

\begin{questyle}
  \question  zzz  (GATE-zzz)

  \begin{choices}
    \choice         zzz
    \choice         zzz
    \choice         zzz
    \choice         zzz
    \CorrectChoice
  \end{choices}
\end{questyle}

\begin{questyle}
  \question  zzz  (GATE-zzz)

  \begin{choices}
    \choice         zzz
    \choice         zzz
    \choice         zzz
    \choice         zzz
    \CorrectChoice
  \end{choices}
\end{questyle}

\end{comment}
