
\centerline{\textbf{ \LARGE Infix, Postfix Expressions Level-2}}


\begin{questyle}
  \question  The result evaluating the postfix expression 10 5 + 60 6 / * 8 -  is  (GATE-2015\_set\_3)

  \begin{choices}
    \choice         284
    \choice         213
    \CorrectChoice  142
    \choice         71
  \end{choices}
\end{questyle}


\begin{questyle}
  \question  The postfix expression for the infix expression is.  (GATE-1995)\\
              A + B * (C + D)/F + D * E

  \begin{choices}
    \choice         AB + CD + *F/D +E*
    \CorrectChoice  ABCD + *F/DE* ++
    \choice         A * B + CD/F *DE ++
    \choice         A + *BCD/F* DE ++
  \end{choices}
\end{questyle}


\begin{questyle}
  \question  Compute the Postfix equivalent of the following Infix expression \fillin[]  (GATE-1998) \\
              3* log(x+1) – a/2

\end{questyle}


\begin{questyle}
  \question   Assume that the operators + , -, x are left associative and \( \hat{} \) is right associative. The order of
              precedence (from highest to lowest) is (  \( \hat{} \), x, +, - ). The postfix expression corresponding to the infix. (GATE-2004) \\
              a + b x c - d \( \hat{} \) e \( \hat{} \) f

  \begin{choices}
    \CorrectChoice  abc x + def \( \hat{} \) \( \hat{} \) -
    \choice         abc x + de \( \hat{} \) f \( \hat{} \) -
    \choice         ab + c x d - e \( \hat{} \) f  \( \hat{} \)
    \choice         - + a x bc \( \hat{} \) \( \hat{} \) def
  \end{choices}
\end{questyle}



\begin{questyle}
  \question  The following postfix expression with single digit operands is evaluated using a stack.
             The top two elements of the stack after the first * is evaluated are: (GATE-2007) \\
             8 2 3 \( \hat{} \)  / 2 3 * + 5 1 * -

  \begin{choices}
    \CorrectChoice  6, 1
    \choice         5, 7
    \choice         3, 2
    \choice         1, 5
  \end{choices}
\end{questyle}



\begin{questyle}
  \question  The attributes of three arithmetic operators in some programming language are given below.
            The value of the expression 2 – 5 + 1 – 7 * 3 in this language is \fillin[9]. (GATE-2016\_set\_1)
    \begin{lstlisting}
    Operator  Precedence   Associativity     Arity
    +           High         Left            Binary
    -           Medium       Right           Binary
    *           Low          Left            Binary
    \end{lstlisting}
\end{questyle}































