\centerline{\textbf{ \LARGE Stack Level-1}}

\begin{questyle}
  \question  Consider the following statements:  (GATE-1996)
    \begin{enumerate}
        \item First-in-first out types of computations are efficiently supported by STACKS.
        \item Implementing LISTS on linked lists is more efficient than implementing LISTS on an array for almost all the basic LIST operations.
        \item Implementing QUEUES on a circular array is more efficient than implementing QUEUES on a linear array with two indices.
        \item Last-in-first-out type of computations are efficiently supported by QUEUES.
    \end{enumerate}

  \begin{oneparchoices}
    \choice         II
    \choice         I and II
    \CorrectChoice  III
    \choice         II and IV
  \end{oneparchoices}
\end{questyle}

\begin{questyle}
  \question  The following sequence of operations is performed on a stack. The sequence of values popped out is.  (GATE-1991) \\
             PUSH (10),  PUSH(20),  POP,  PUSH(10),  PUSH(20),  POP,  POP,  POP,  PUSH(20),  POP

  \begin{choices}
    \choice         20, 10, 20, 10, 20
    \CorrectChoice  20, 20, 10, 10, 20
    \choice         10, 20, 20, 10, 20
    \choice         20, 20, 10, 20, 10
  \end{choices}
\end{questyle}


\begin{questyle}
  \question  Which of the following permutations can be obtained in the output (in the same order) using a stack assuming
            that the input is the sequence 1, 2, 3, 4, 5 in that order?  (GATE-1994) \\ Hint : use push and pop operations

  \begin{choices}
    \choice         3, 4, 5, 1, 2
    \CorrectChoice  3, 4, 5, 2, 1
    \choice         1, 5, 2, 3, 4
    \choice         5, 4, 3, 1, 2
  \end{choices}
\end{questyle}

\begin{questyle}
  \question  A program attempts to generate as many permutations as possible of the string, ‘abcd’ by pushing the
            characters a, b, c, d in the same order onto a stack, but it may pop off the top character at any time.
            Which one of the following strings CANNOT be generated using this program?  (GATE-2004)

  \begin{choices}
    \choice         dcba
    \choice         cbad
    \choice         cabd
    \CorrectChoice  abcd
  \end{choices}
\end{questyle}
