
\centerline{\textbf{ \LARGE Linked List Level-2}}


\begin{questyle}
  \question  Consider the C code fragment given below. Assuming that m and n point to valid NULL- terminated linked
            lists, invocation of join will. (GATE-2017\_Set\_1)

    \lstinputlisting[language=Octave]{./linked_list_code/linked_list_01.c}

  \begin{choices}
    \choice         append list m to the end of list n for all inputs
    \CorrectChoice  either cause a null pointer dereference or append list m to the end of list n
    \choice         cause a null pointer dereference for all inputs.
    \choice         append list n to the end of list m for all inputs.
  \end{choices}
\end{questyle}


\begin{questyle}
  \question  Consider the function f defined below. For a given linked list p, the function f returns 1 if and only if  (GATE-2003)
            \lstinputlisting[language=Octave]{./linked_list_code/linked_list_02.c}
  \begin{choices}
    \choice         the list is empty or has exactly one element
    \CorrectChoice  the elements in the list are sorted in non-decreasing order of data value
    \choice         the elements in the list are sorted in non-increasing order of data value
    \choice         not all elements in the list have the same data value.
  \end{choices}
\end{questyle}


\begin{questyle}
  \question  The following C function takes a single-linked list of integers as a parameter and rearranges
            the elements of the list. The function is called with the list containing the integers 1, 2, 3, 4, 5, 6, 7
            in the given order. What will be the contents of the list after the function completes execution?  (GATE-2008)
            \lstinputlisting[language=Octave]{./linked_list_code/linked_list_03.c}
  \begin{choices}
    \choice         1,2,3,4,5,6,7
    \CorrectChoice  2,1,4,3,6,5,7
    \choice         1,3,2,5,4,7,6
    \choice         2,3,4,5,6,7,1
  \end{choices}
\end{questyle}


\begin{questyle}
  \question  The following C function takes a simply-linked list as input argument. It modifies the list by
            moving the last element to the front of the list and returns the modified list. Some part of
            the code is left blank. Choose the correct alternative to replace the blank line. (GATE-2010)
            \lstinputlisting[language=Octave]{./linked_list_code/linked_list_04.c}
  \begin{choices}
    \choice         \begin{verbatim} q = NULL; p->next = head; head = p; \end{verbatim}
    \choice         \begin{verbatim} q->next = NULL; head = p; p->next = head; \end{verbatim}
    \choice         \begin{verbatim} head = p; p->next = q; q->next = NULL;  \end{verbatim}
    \CorrectChoice  \begin{verbatim} q->next = NULL; p->next = head; head = p;  \end{verbatim}
  \end{choices}
\end{questyle}


