
\centerline{\textbf{ \LARGE Queue Level-3}}

\begin{questyle}
  \question  An implementation of a queue Q, using two stacks S1 and S2, is given below:  (GATE-2006)
    \lstinputlisting[language=Octave]{./queue_code/queue_01.c}
    Let n insert and m (\(<=n\)) delete operations be performed in an arbitrary order on an empty queue Q.
    Let x and y be the number of push and pop operations performed respectively in the process.
    Which one of the following is true for all m and n?

  \begin{choices}
    \CorrectChoice  n+m \(<=\) x \(<\) 2n and 2m \(<=\) y \(<=\) n+m
    \choice         n+m \(<=\) x \(<\) 2n and 2m \(<=\) y \(<=\) 2n
    \choice         2m \(<=\) x \(<\) 2n and 2m \(<=\) y \(<=\) n+m
    \choice         2m \(<=\) x \(<\) 2n and 2m \(<=\) y \(<=\) 2n
  \end{choices}
\end{questyle}


\begin{questyle}
  \question  Suppose you are given an implementation of a queue of integers. The operations that can be performed on the queue are:
            isEmpty(Q), delete(Q), insert(Q, i)  (GATE-2007)

            \lstinputlisting[language=Octave]{./queue_code/queue_02.c}

  \begin{choices}
    \choice         Leaves the queue Q unchanged
    \CorrectChoice  Reverses the order of the elements in the queue Q
    \choice         Deletes front element  and inserts it at the rear keeping the other elements in the same order.
    \choice         Empties the queue Q
  \end{choices}
\end{questyle}

\begin{questyle}
  \question  Let Q denote a queue containing sixteen numbers and S be an empty stack. Head(Q) returns
            the element at the head of the queue Q without removing it from Q. Similarly Top(S) returns
            the element at the top of S without removing it from S. Consider the algorithm given below.
            The maximum possible number of iterations of the while loop in the algorithm is \fillin[256]   (GATE-2016\_set\_1)

          \lstinputlisting[language=Octave]{./queue_code/queue_03.c}


\end{questyle}
