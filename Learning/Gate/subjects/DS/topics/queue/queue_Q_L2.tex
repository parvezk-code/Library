
\centerline{\textbf{ \LARGE Queue Level-2}}

\begin{questyle}
  \question  Suppose a circular queue of capacity (n – 1) elements is implemented with an array of n elements.
            Assume that the insertion and deletion operation are carried out using REAR and FRONT as array index variables, respectively. Initially, REAR = FRONT = 0. The conditions to detect queue full and queue empty are  (GATE-2012)

  \begin{choices}
    \CorrectChoice  Full: (REAR+1) mod n == FRONT, empty: REAR == FRONT
    \choice         Full: (REAR+1) mod n == FRONT, empty: (FRONT+1) mod n == REAR
    \choice         Full: REAR == FRONT, empty: (REAR+1) mod n == FRONT
    \choice         Full: (FRONT+1) mod n == REAR, empty: REAR == FRONT
  \end{choices}
\end{questyle}

\begin{questyle}
  \question  Suppose a stack implementation supports an instruction REVERSE, which reverses the
            order of elements on the stack, in addition to the PUSH and POP instructions. Which
            one of the following statements is TRUE with respect to this modified stack?  (GATE-2014\_set\_2)

  \begin{choices}
    \choice         A queue cannot be implemented using this stack.
    \choice         A queue can be implemented where ENQUEUE takes 1 and DEQUEUE takes 2 instructions.
    \CorrectChoice  A queue can be implemented where ENQUEUE takes 3 and DEQUEUE takes 1 instructions.
    \choice         A queue can be implemented where ENQUEUE takes 1 and DEQUEUE takes 1 instructions.
  \end{choices}
\end{questyle}


\begin{questyle}
  \question  A circularly linked list is used to represent a Queue. A single variable p is used to access
            the Queue. To which node should p point such that both the operations enQueue and deQueue can
            be performed in constant time?  (GATE-2004)

  \begin{choices}
    \CorrectChoice  rear node
    \choice         front node
    \choice         not possible with a single pointer
    \choice         node next to front
  \end{choices}
\end{questyle}

