

\begin{questyle}
  \question  The following function computes the value of C(m,n) correctly for all legal values m and n (m\(>=\)1,n\(>=\)0 and m\(>\)n)
            In the below function, which of the following is the correct expression for E? (GATE-2006)

    \begin{lstlisting}
        int func(int m, int n)
        {
            if(E) return 1;
            else
            {
                return(func(m -1, n) + func(m - 1, n - 1));
            }
        }
    \end{lstlisting}

  \begin{choices}
    \choice         (n = = 0) OR  (m = = 1)
    \choice         (n = = 0) AND (m = = 1)
    \CorrectChoice  (n = = 0) OR  (m = = n)
    \choice         (n = = 0) AND (m = = n)
  \end{choices}
\end{questyle}


\begin{questyle}
  \question  Consider the following recursive JAVA function. If get(6) function is being called in
            main() then how many times will the get() function be invoked before returning
            to the main()?   (GATE-2015\_set\_3)

        \begin{lstlisting}
        static void get (int n)
        {
            if (n < 1)
                return;
            get(n - 1);
            get(n - 3);
            System.out.print(n);
        }
        \end{lstlisting}

  \begin{choices}
    \choice         15
    \CorrectChoice  25
    \choice         35
    \choice         45

  \end{choices}
\end{questyle}


\begin{questyle}
  \question  Consider the following New-order strategy for traversing a binary tree:  (GATE-2016\_set\_2)\\
            Visit the root; \\
            Visit the right subtree using New-order \\
            Visit the left subtree using New-order \\
            The New-order traversal of the expression tree corresponding to the reverse polish
            expression \\  \(  3 \; 4 \;  * \;  5 \;  – \;  2 \;  \wedge \;  6 \;  7 \;  * \;  1 \;  + \;  - \) is given by:

  \begin{choices}
    \choice         \(  + -  1 \ 6 \ 7 * 2 \wedge 5 \ - \ 3 \ 4 \ * \)
    \choice         \( - +  1 *  6 \ 7 \wedge 2 - 5 * 3 \ 4  \)
    \CorrectChoice  \( - +  1 * 7 \ 6  \wedge 2 \ - \ 5 \ * \ 4 \ 3  \)
    \choice         \( 1 \ 7 \ 6 *  + \ 2 \ 5  \ 4 \ 3 \ * \ - \wedge -  \)
  \end{choices}
\end{questyle}

\begin{questyle}
  \question  Consider the expression tree shown. Each leaf represents a numerical value, which can
             either be 0 or 1. Over all possible choices of the values at the leaves, the maximum
             possible value of the expression represented by the tree is  \fillin[6] (GATE-2014\_set\_2)

  \begin{myTreeLThree}
    \node [circle,draw] [red] (node_aa){+}
    child
    {
        node [circle,draw] (node_bw) {-}
        child
        {
            node [circle,draw] (node_er) {+}
            child{ node [circle,draw] (node_jj) {0/1} }
            child{ node [circle,draw] (node_kk) {0/1} }
        }
        child
        {
            node [circle,draw] (node_fy) {-}
            child{ node [circle,draw] (node_pp) {0/1} }
            child{ node [circle,draw] (node_ee) {0/1} }
        }
    }
    child
    {
        node [circle,draw] (node_dy){+}
        child
        {
            node [circle,draw] (node_gs) {-}
            child{ node [circle,draw] (node_ww) {0/1} }
            child{ node [circle,draw] (node_ss) {0/1} }
        }
        child
        {
            node [circle,draw] (node_hj) {+}
            child{ node [circle,draw] (node_zz) {0/1} }
            child{ node [circle,draw] (node_ff) {0/1} }
        }
    };
  \end{myTreeLThree}


\end{questyle}


