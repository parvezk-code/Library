
\centerline{\textbf{ \LARGE Pipeline Hazards}}


\begin{questyle}
  \question  For a pipelined CPU with a single ALU, consider the following situations. Which of the
            following can cause a hazard ?  (GATE-2003)

  \begin{enumerate}
    \item The (j+1)-st instruction uses the result of the j-th instruction
    \item The execution of a conditional jump instruction
    \item The j-th and (j+1)-st instructions require the ALU at the same time.
  \end{enumerate}

  \begin{oneparchoices}
    \choice         1 and 2 only
    \choice         2 and 3 only
    \choice         3 only
    \CorrectChoice  All of above
  \end{oneparchoices}
\end{questyle}


\begin{questyle}
  \question  Register renaming is done in pipelined processors  (GATE-2012)

  \begin{choices}
    \choice         as an alternative to register allocation at compile time
    \choice         for efficient access to function parameters and local variables
    \CorrectChoice  to handle certain kinds of hazards
    \choice         as part of address translation
  \end{choices}
\end{questyle}


\begin{questyle}
  \question  The performance of a pipelined processor suffers if  (GATE-2002)

  \begin{choices}
    \choice         the pipeline stages have different delays
    \choice         consecutive instructions are dependent on each other
    \choice         the pipeline stages share hardware resources
    \CorrectChoice  all of the above
  \end{choices}
\end{questyle}


\begin{questyle}
  \question  In an instruction execution pipeline, the earliest that the data TLB (Translation Lookaside Buffer)
            can be accessed is  (GATE-2008)

  \begin{choices}
    \choice         before effective address calculation has started
    \choice         during effective address calculation
    \CorrectChoice  after effective address calculation has completed
    \choice         after data cache lookup has completed
  \end{choices}
\end{questyle}

\begin{questyle}
  \question  Which of the following are NOT true in a pipelined processor?  (GATE-2008)

  \begin{enumerate}
      \item Bypassing can handle all RAW hazards.
      \item Register renaming can eliminate all register carried WAR hazards.
      \item Control hazard penalties can be eliminated by dynamic branch prediction.
  \end{enumerate}

  \begin{oneparchoices}
    \choice         I and II only
    \CorrectChoice  I and III only
    \choice         II and III only
    \choice         I, II and III
  \end{oneparchoices}
\end{questyle}


\begin{questyle}
  \question  The use of multiple register windows with overlap causes a reduction in the number of memory
             accesses for  (GATE-2008)

  \begin{enumerate}
      \item Function locals and parameters
      \item Register saves and restores
      \item Instruction fetches
  \end{enumerate}

  \begin{oneparchoices}
    \CorrectChoice  I only
    \choice         II only
    \choice         III only
    \choice         I, II and III
  \end{oneparchoices}
\end{questyle}


\begin{questyle}
  \question  Delayed branching can help in the handling of control hazards. The following code is
             to run on a pipelined processor with one branch delay slot. Which of the instructions
             I1, I2, I3 or I4 can legitimately occupy the delay slot without any other program modification?(GATE-2008)

             \begin{enumerate}
                \item[I1-] ADD R2 \(\leftarrow\) R7+R8
                \item[I2-] SUB R4 \(\leftarrow\) R5-R6
                \item[I3-] ADD R1 \(\leftarrow\) R2+R3
                \item[I5-] STORE Memory [R4] \(\leftarrow\) [R1] \\
                 BRANCH to Label if R1== 0
            \end{enumerate}

  \begin{oneparchoices}
    \choice         I1
    \choice         I2
    \choice         I3
    \CorrectChoice  I4
  \end{oneparchoices}
\end{questyle}


\begin{comment}


\begin{questyle}
  \question  zzz  (GATE-zzz)

  \begin{choices}
    \choice         zzz
    \choice         zzz
    \choice         zzz
    \choice         zzz
    \CorrectChoice
  \end{choices}
\end{questyle}

\begin{questyle}
  \question  zzz  (GATE-zzz)

  \begin{choices}
    \choice         zzz
    \choice         zzz
    \choice         zzz
    \choice         zzz
    \CorrectChoice
  \end{choices}
\end{questyle}


\end{comment}




