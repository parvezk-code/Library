\centerline{\textbf{ \LARGE Number System}}

\begin{enumerate}
    \item Types \\
    \begin{myTableStyle} \begin{tabular}{ |m{5cm}|m{8cm}| } \hline
        Weighted Number System      &   Binary, Octal, Hexa-decimal, BCD, 2421 etc.. \\ \hline
        Un-weighted Number System   &   Grey code, Excess-3 code  \\ \hline
        Self complementing code     &   Excess-3, 2421\\ \hline
    \end{tabular} \end{myTableStyle} \vspace{0.08in}

    \item Conversion between number systems.\\
    \begin{myTableStyle} \begin{tabular}{ |m{5cm}|m{9cm}| } \hline
        { \Large \( (P.Q)_{10}\) to \( (X.Y)_a\) }      &   Divide P by a. Multiply Q by a \\ \hline
        { \Large \( (X.Y)_a\) to \( (P.Q)_{10}\) }      &   Power of a \\ \hline
        Binary to Octal-Hexa                            &   Right to Left pairing. Octal(3), Hexa(4) \\ \hline
        Octal-Hexa to Binary                            &   Convert each digit to binary equivalant. Octal(3), Hexa(4) \\ \hline
    \end{tabular} \end{myTableStyle} \vspace{0.08in}

    \item BCD(8421) : decimal to binary
    \begin{enumerate}
    \item Each digit in decimal number is converted to its 4(or 8) bit binary equivalent.
    \item Conversion is faster. Less multiplication or division
    \item Havy-weighted : Uses extra bits.
    \item It avoids fractional errors. More accurate representation for numbers with decimal.\\
    \begin{myTableStyle} \begin{tabular}{ |m{3cm}|m{4cm}|m{3cm}| } \hline
        Decimal                 & Binary(Fractional error)     & BCD       \\ \hline
        {\Large \((0.2)_{10}\)} & (0.0011 . . . . \(\infty\))   & (0.0010)  \\ \hline
    \end{tabular} \end{myTableStyle} \vspace{0.08in}

    \item Valid Invalid BCD code \\
    \begin{myTableStyle} \begin{tabular}{ |m{2cm}|m{2cm}|m{2cm}|m{2cm}|m{2cm}| } \hline
                    & Total     & Valid & Invalid   & Range     \\ \hline
        4-bit BCD   &   16      & 10    & 6         & 0 - 9     \\ \hline
        8-bit BCD   &   256     & 100   & 156       & 00 - 99   \\ \hline
    \end{tabular} \end{myTableStyle} \vspace{0.08in}

    \end{enumerate}

    \item Excess–3 code : decimal to binary
    \begin{enumerate}
        \item Add 3 to each digit of the number.
        \item Convert to BCD code.
    \end{enumerate}

    \item Gray Code : Also called Unit-distance codebinary
    \begin{enumerate}
        \item Successive numbers differ by exactly one bit.
        \item Binary to Grey : Copy MSB. XOR adjescent pair.
    \end{enumerate}

    \begin{center} \begin{tikzpicture}
        \node[squarednode]      (MSBB)                     {1};
        \node[squarednode]      (B2)       [right=of MSBB] {0};
        \node[squarednode]      (B3)       [right=of B2]   {0};
        \node[squarednode]      (B4)       [right=of B3]   {1};
        \node[squarednode]      (B5)       [right=of B4]   {1};
        \node[squarednode]      (B6)       [right=of B5]   {0};

        \node[roundnode]      (MSBG)     [below=of MSBB] {1};
        \node[roundnode]      (G2)       [right=of MSBG] {1};
        \node[roundnode]      (G3)       [right=of G2]   {0};
        \node[roundnode]      (G4)       [right=of G3]   {1};
        \node[roundnode]      (G5)       [right=of G4]   {0};
        \node[roundnode]      (G6)       [right=of G5]   {1};

        \begin{vArrow}{MSBB.south}{MSBG.north} \end{vArrow}

        \begin{tArcDown}[blue]{(MSBB.south)} \end{tArcDown}
        \begin{vArrow}[blue]{B2.south}{G2.north} \end{vArrow}

        \begin{tArcDown}{(B2.south)} \end{tArcDown}
        \begin{vArrow}{B3.south}{G3.north} \end{vArrow}

        \begin{tArcDown}[blue]{(B3.south)} \end{tArcDown}
        \begin{vArrow}[blue]{B4.south}{G4.north} \end{vArrow}

        \begin{tArcDown}{(B4.south)} \end{tArcDown}
        \begin{vArrow}{B5.south}{G5.north} \end{vArrow}

        \begin{tArcDown}[blue]{(B5.south)} \end{tArcDown}
        \begin{vArrow}[blue]{B6.south}{G6.north} \end{vArrow}

    \end{tikzpicture} \end{center}

    \begin{minipage}{\linewidth}
    \item B's Complement and (B-1)'s Complement
    \begin{enumerate}
        \item (B-1)'s Complement: Subtract number from largest no of that number system. \\
    \begin{myTableStyle} \begin{tabular}{ |m{4cm}|m{3cm}|m{2cm}|m{2cm}| } \hline
                            & Hexa Decimal           & Octal                & Binary                \\ \hline
        (B-1)'s Complement  & FFFF - \( (hex)_{16}\)   & 7777 - \( (oct)_{8}\)  & 1111 - \( (bin)_{2}\)   \\ \hline
    \end{tabular} \end{myTableStyle} \vspace{0.08in}

        \item B's Complement : 1 + (B-1)'s Complement. \\
    \end{enumerate}
    \end{minipage}


    \item Representaion of signed numbers. (Signed, 1's, 2's)
    \begin{enumerate}
        \item For n digits maximum unsigned number is \( (2^n -1) \)
        \item For signed numbers extra bit used.
        \item MSB=0 for +ive numbers and MSB=1 for negative numbers \\
    \begin{myTableStyle} \begin{tabular}{ |m{3cm}|m{3cm}|m{3cm}|m{3cm}| } \hline
        Size = 1+ (n-1)     & Signed Magnitude           & 1's Complement       & 2's complement        \\ \hline
        +ive Number         & Extra Bit for Sign         & Extra Bit for Sign   & Extra Bit for Sign    \\ \hline
        -ive Number         & Extra Bit for Sign         & 1's Comp of +ive no  & 2's Comp of +ive no   \\ \hline
        Range               & \makecell[l]{ \( -(2^{(n-1)} -1) \) to \\ \( +(2^{(n-1)} -1) \) }
                                                         & \makecell[l]{ \( -(2^{(n-1)} -1) \) to \\ \( +(2^{(n-1)} -1) \) }
                                                         & \makecell[l]{ \( -(2^{(n-1)}) \) to \\ \( +(2^{(n-1)} -1) \) }   \\ \hline
        Zero                & Represented twice          & Represented twice    & Represented Once   \\ \hline
    \end{tabular} \end{myTableStyle} \vspace{0.08in}
    \end{enumerate}

    \item Subtraction(P - Q)\\
    \begin{myTableStyle} \begin{tabular}{ |m{3cm}|m{3cm}|m{3cm}| }  \hline
        P - Q       &   1's Complement      &  2's complement       \\ \hline
        Complement  &   Q1 = Q(1's comp)    &  Q2 = Q(2's comp)     \\ \hline
        Addition    &   R = P + Q1          &  R = P + Q2           \\ \hline
        Carry Yes   &   Result = R + C      &  Result = R           \\ \hline
        Carry No    &   Result = R1         &  Result = R2          \\ \hline
    \end{tabular} \end{myTableStyle} \vspace{0.08in}

    \item condition for overflow.

    \item Shifting in Multiplication and Division.
    \begin{enumerate}
        \item Multiplication by 2 is equivalent to left shift.
        \item Division by 2 is equivalent to right shift.
    \end{enumerate}

    \item Representing real number. www.geeksforgeeks.org/fixed-point-representation/ \\
    \begin{myTableStyle} \begin{tabular}{ |m{3cm}|m{10cm}| } \hline
        Fixed Point(8, 3) &  8-bit number and 3 bits for fraction. \\ \hline
        Floating Point &  \\ \hline
    \end{tabular} \end{myTableStyle} \vspace{0.08in}

    \begin{minipage}{\linewidth}
    \item Fixed Point Representation.
    \begin{enumerate}
        \item Same hardware for integer arithmetic.
        \item Loss in range and precision
    \end{enumerate}

    \begin{myTableStyle} \begin{tabular}{ |m{3cm}|m{3cm}|m{3cm}|m{3cm}| } \hline
        Size = 1+ (n-1)     & Signed Magnitude           & 1's Complement       & 2's complement        \\ \hline
        Zero                & Represented twice          & Represented twice    & Represented Once      \\ \hline
        Range               & Loss                       & Loss                 & No Loss               \\ \hline
    \end{tabular} \end{myTableStyle} \vspace{0.08in}
    \end{minipage}

\end{enumerate}
