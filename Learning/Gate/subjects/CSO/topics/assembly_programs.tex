
\centerline{\textbf{ \LARGE Assembly Language Programs}}


\begin{questyle}
  \question  Consider the following assembly language program for a hypothetical processor.
             A, B, and C are 8 bit registers. The meanings of various instructions are shown
             as comments.  (GATE-2003)
    \begin{myTableStyle} \begin{tabular}{ |m{9cm}|m{6cm}| } \hline
        Assembly Program given in question &  Equivalent C program  \\ \hline
        \lstinputlisting[language={[x86masm]Assembler}, firstline=16, lastline=27]{cso_programs.c} &
        \lstinputlisting[language=C, firstline=33, lastline=43]{cso_programs.c} \\ \hline
    \end{tabular} \end{myTableStyle} \vspace{0.08in}

  \begin{parts}
  \part The value of register B after program execution is.
  \begin{choices}
    \choice         the number of 0 bits in A0
    \CorrectChoice  the number of 1 bits in A0
    \choice         A0
    \choice         8
  \end{choices}

  \part Which of the following instructions when inserted at location X will ensure that the value of register A after program execution is the same as its initial value ?
  \begin{choices}
    \CorrectChoice  RRC A, \#
    \choice         NOP ; no operation
    \choice         LRC A, \# 1 ; left rotate A through carry flag by one bit
    \choice         ADD A, \# 1
  \end{choices}

  \end{parts}

\end{questyle}


\begin{questyle}
  \question  Consider the following program segment for a hypothetical CPU having three
             user registers R1, R2 and R3.  (GATE-2004)
        \lstinputlisting[language={[x86masm]Assembler}, firstline=50, lastline=55]{cso_programs.c}

  \begin{parts}
  \part Consider that the memory is byte addressable with size 32 bits, and the program has
        been loaded starting from memory location 1000 (decimal). If an interrupt occurs
        while the CPU has been halted after executing the HALT instruction, the return
        address (in decimal) saved in the stack will be.
  \begin{oneparchoices}
    \choice         1007
    \choice         1020
    \choice         1024
    \CorrectChoice  1028
  \end{oneparchoices} \\
  Hint : PC is saved in the stack when interupt occurs.\\
  Hint : Address of HALT instruction(5) = B + Size\_of(I1+I2+I3+I4) = 1000 + (6x4) = 1024 \\
  Hint : Address of next instruction after HALT = 1024 + (1x4) = 1028

  \part Let the clock cycles required for various operations be as follows: Register to/ from
        memory transfer: 3 clock cycles ADD with both operands in register : 1 clock cycle Instruction
        fetch and decode : 2 clock cycles per word The total number of clock cycles required to
        execute the program is.

  \begin{oneparchoices}
    \choice         29
    \CorrectChoice  24
    \choice         23
    \choice         20
  \end{oneparchoices} \\
  Hint : (IF+ID+Ex+MA) = I1(2+2+0+3) \; I2(1+1+0+3) \; I3(1+1+1+0) \; I4(2+2+0+3) \; (1+1+0+0)

  \end{parts}

\end{questyle}


\begin{questyle}
  \question  In a simplified computer the instructions are. Assume that all operands are
             initially in memory. The final value of the computation should be in memory.
             What is the minimum number of MOV instructions in the code generated for
             this basic block?  (GATE-2007)
    \begin{myTableStyle} \begin{tabular}{ |m{9cm}|m{6cm}| } \hline
        Available Instruction    &  Basic Block to implement  \\ \hline
        \lstinputlisting[language={[x86masm]Assembler}, firstline=62, lastline=65]{cso_programs.c} &
        \lstinputlisting[language={[x86masm]Assembler}, firstline=71, lastline=74]{cso_programs.c} \\ \hline
    \end{tabular} \end{myTableStyle} \vspace{0.08in}

  \begin{oneparchoices}
    \choice         2
    \CorrectChoice  3
    \choice         5
    \choice         6
  \end{oneparchoices}
\end{questyle}


\begin{questyle}
  \question  Consider the following program segment. Here R1, R2 and R3 are the general purpose
            registers. Assume that the content of memory location 3000 is 10 and the content of
            the register R3 is 2000. The content of each of the memory locations from 2000 to 2010
            is 100. The program is loaded from the memory location 1000. All the numbers are
            in decimal. (GATE-2007)

    \begin{myTableStyle} \begin{tabular}{ |m{9.5cm}|m{5cm}| } \hline
        Assembly Program given in question &  Equivalent C program  \\ \hline
        \lstinputlisting[language={[x86masm]Assembler}, firstline=80, lastline=92]{cso_programs.c} &
        \lstinputlisting[language=C, firstline=98, lastline=105]{cso_programs.c} \\ \hline
    \end{tabular} \end{myTableStyle} \vspace{0.08in}

  \begin{parts}
  \part Assume that the memory is word addressable. The number of memory references for accessing
        the data in executing the program completely is.

  \begin{oneparchoices}
    \choice         10
    \choice         11
    \choice         20
    \CorrectChoice  21  \qquad  Hint: Once before loop. 2 times in every iteration.
  \end{oneparchoices}

  \part Consider the data given in above question. Assume that the memory is word addressable.
        After the execution of this program, the content of memory location 2010 is.

  \begin{oneparchoices}
    \CorrectChoice  100
    \choice         101
    \choice         102
    \choice         110
  \end{oneparchoices}

  \part Consider the data given in above questions. Assume that the memory is byte addressable
       and the word size is 32 bits. If an interrupt occurs during the execution of the
       instruction "INC R3", what return address will be pushed on to the stack?

  \begin{oneparchoices}
    \choice         1005
    \choice         1020
    \CorrectChoice  1024
    \choice         1040
  \end{oneparchoices}

  \end{parts}

\end{questyle}
