
\centerline{\textbf{ \LARGE Other CSO Topics}}


\begin{questyle}

  \question  How many 32K x 1 RAM chips are needed to provide a memory capacity of 256K Bytes bytes? (GATE-XXX)

  \begin{oneparchoices}
    \choice 8
    \choice 32
    \choice 64
    \choice 128
  \end{oneparchoices}

\end{questyle}


\begin{questyle}

  \question  Transferring data in blocks from the main memory to the cache memory enables an interleaved main memory unit to operate unit at its maximum speed. True/False. (GATE-1990)

\end{questyle}




\begin{questyle}

  \question  The capacity of a memory unit is defined by the number of words multiplied by the number of bits/word. How many separate address and data lines are needed for a memory of \( \large 4K \times 16 \)(GATE-1995)

  \begin{choices}
    \choice 10 address, 16 data lines
    \choice 11 address, 8 data lines
    \choice 12 address, 16 data lines
    \choice 12 address, 12 data lines
  \end{choices}

\end{questyle}

\begin{questyle}
  \question  Which of the following statements is true?  (GATE-1995)

  \begin{choices}
    \choice         ROM is a Read/Write memory
    \choice         PC points to the last instruction that was executed
    \CorrectChoice  Stack works on the principle of LIFO
    \choice         All instructions affect the flags
  \end{choices}
\end{questyle}

\begin{questyle}
  \question  Which of the following is not a form of memory?  (GATE-2002)

  \begin{oneparchoices}
    \choice         instruction cache
    \choice         instruction register
    \CorrectChoice  instruction opcode
    \choice         translation lookaside buffer
  \end{oneparchoices}
\end{questyle}

\begin{questyle}
  \question  A CPU has 24-bit instructions. A program starts at address 300 (in decimal). Which one
             of the following is a legal program counter (all values in decimal)?  (GATE-2006)

  \begin{oneparchoices}
    \choice         400
    \choice         500
    \CorrectChoice  600
    \choice         700      \qquad Hint: multiple of 3(24 bits)
  \end{oneparchoices}
\end{questyle}

\begin{questyle}
  \question  For computers based on three-address instruction formats, each address field can be used
             to specify which of the following:  (GATE-2015\_Set\_1) \\
             S1: A memory operand \quad S2: A processor register \quad S3: An implied accumulator register

  \begin{oneparchoices}
    \CorrectChoice  Either S1 or S2
    \choice         Either S2 or S3
    \choice         Only S2 and S3
    \choice         S1, S2 and S3
  \end{oneparchoices}
\end{questyle}


\begin{questyle}
  \question  A processor has 40 distinct instructions and 24 general purpose registers.
             A 32-bit instruction word has an opcode, two register operands and an immediate operand.
             The number of bits available for the immediate operand field is \fillin[16] (GATE-2016\_Set\_2)
             \\ Hint : \qquad 5-bits to index 24 register \qquad 6-bis to index 40 opcode
             \\ HInt : \qquad Instruction(32 bit) = opcode(6) + R1(5) + R2(5) + Immediate Operand(x)
\end{questyle}

\begin{questyle}
  \question  A machine has a 32-bit architecture, with 1-word long instructions. It has 64 registers,
             each of which is 32 bits long. It needs to support 45 instructions, which have an immediate
             operand in addition to two register operands. Assuming that the immediate operand is an
             unsigned integer, the maximum value of the immediate operand is
             \fillin[16383 \( \boldsymbol {(2^{14} -1)} \)] (GATE-2014\_Set\_1)
             \\ Hint : \qquad 6-bits to index 64 register \qquad 6-bis to index 45 opcode(instructions)
             \\ HInt : \qquad Instruction(32 bit) = opcode(6) + R1(6) + R2(6) + Immediate Operand(x)
\end{questyle}

\begin{questyle}
  \question  Consider a processor with 64 registers and an instruction set of size twelve. Each
             instruction has five distinct fields, namely, opcode, two source register identifiers,
             one destination register identifier, and a twelve-bit immediate value. Each instruction
             must be stored in memory in a byte-aligned fashion. If a program has 100 instructions,
             the amount of memory (in bytes) consumed by the program text is \fillin[500] (GATE-2016\_Set\_2)
             \\ Hint : \quad 6-bits to index 64 register \qquad 4-bis to index 12 opcode(instructions)
             \\ Hint : \quad Instruction(34 bit) = opcode(4) + R1(6) + R2(6) + R3(6) + Immediate Operand(12)
             \\ Hint : \quad Total size need to store one instruction in byte-aligned fashion = 40(\(\thickapprox\)34) = 5 bytes.
             \\ Hint : \quad Size to store 100 instructions = 5 x 100 = 500 Bytes
\end{questyle}


\begin{questyle}
  \question  Consider two processors P1 and P2 executing the same instruction set. Assume that under
             identical conditions, for the same input, a program running on P2 takes 25\% less time but
             incurs 20\% more CPI (clock cycles per instruction) as compared to the program
             running on P1. If the clock frequency of P1 is 1GHz, then the clock frequency of
             P2 (in GHz) is \fillin[1.6]  (GATE-2014\_Set\_1)
             \\ Time = T \quad No of Instructions = I \quad Cycles per instruction = CPI \quad Cycle Time = CT = 1/F
             \\ T = I x CPI x CT  \qquad  T = { \Large \( \frac{I * CPI}{F}  \) }
\end{questyle}


\begin{questyle}
  \question  Consider a processor with byte-addressable memory. Assume that all registers,
             including Program Counter (PC) and Program Status Word (PSW), are of size 2 bytes. A
             stack in the main memory is implemented from memory location (0100)16 and it grows
             upward. The stack pointer (SP) points to the top element of the stack. The current
             value of SP is (016E)16. The CALL instruction is of two words, the first word is
             the op-code and the second word is the starting address of the subroutine
             (one word = 2 bytes). The CALL instruction is implemented as follows: (GATE-2015\_Set\_2)
             \\ \(\rightarrow\) Store the current value of PC in the stack.
             \\ \(\rightarrow\) Store the value of PSW register in the stack.
             \\ \(\rightarrow\) Load the starting address of the subroutine in PC. \\
             The content of PC just before the fetch of a CALL instruction is (5FA0)16. After
             execution of the CALL instruction, the value of the stack pointer is

  \begin{oneparchoices}
    \choice         016A
    \choice         016C
    \choice         0170
    \CorrectChoice  0172
  \end{oneparchoices}
  \\ Hint : Call instruction will push contents of PC(2 Bytes) and PSW(2 Bytes) to the stack.
  \\ Hint : Memory is byte addressable. This will increase stack top by 4.

\end{questyle}

