
\centerline{\textbf{ \LARGE Other CSO Topics}}


\begin{questyle}

  \question  How many 32K x 1 RAM chips are needed to provide a memory capacity of 256K Bytes bytes? (GATE-XXX)

  \begin{oneparchoices}
    \choice 8
    \choice 32
    \choice 64
    \choice 128
  \end{oneparchoices}

\end{questyle}


\begin{questyle}

  \question  Transferring data in blocks from the main memory to the cache memory enables an interleaved main memory unit to operate unit at its maximum speed. True/False. (GATE-1990)

\end{questyle}




\begin{questyle}

  \question  The capacity of a memory unit is defined by the number of words multiplied by the number of bits/word. How many separate address and data lines are needed for a memory of \( \large 4K \times 16 \)(GATE-1995)

  \begin{choices}
    \choice 10 address, 16 data lines
    \choice 11 address, 8 data lines
    \choice 12 address, 16 data lines
    \choice 12 address, 12 data lines
  \end{choices}

\end{questyle}

\begin{questyle}
  \question  Which of the following statements is true?  (GATE-1995)

  \begin{choices}
    \choice         ROM is a Read/Write memory
    \choice         PC points to the last instruction that was executed
    \CorrectChoice  Stack works on the principle of LIFO
    \choice         All instructions affect the flags
  \end{choices}
\end{questyle}

\begin{questyle}
  \question  Which of the following is not a form of memory?  (GATE-2002)

  \begin{oneparchoices}
    \choice         instruction cache
    \choice         instruction register
    \CorrectChoice  instruction opcode
    \choice         translation lookaside buffer
  \end{oneparchoices}
\end{questyle}

\begin{questyle}
  \question  A CPU has 24-bit instructions. A program starts at address 300 (in decimal). Which one
             of the following is a legal program counter (all values in decimal)?  (GATE-2006)

  \begin{oneparchoices}
    \choice         400
    \choice         500
    \CorrectChoice  600
    \choice         700      \qquad Hint: multiple of 3(24 bits)
  \end{oneparchoices}
\end{questyle}

\begin{questyle}
  \question  For computers based on three-address instruction formats, each address field can be used
             to specify which of the following:  (GATE-2015\_Set\_1) \\
             S1: A memory operand \quad S2: A processor register \quad S3: An implied accumulator register

  \begin{oneparchoices}
    \CorrectChoice  Either S1 or S2
    \choice         Either S2 or S3
    \choice         Only S2 and S3
    \choice         S1, S2 and S3
  \end{oneparchoices}
\end{questyle}


\begin{questyle}
  \question  A processor has 40 distinct instructions and 24 general purpose registers.
             A 32-bit instruction word has an opcode, two register operands and an immediate operand.
             The number of bits available for the immediate operand field is \fillin[16] (GATE-2016\_Set\_2) \\
             Hint : \qquad 5-bits to index 24 register \qquad 6-bis to index 40 opcode \\
             HInt : \qquad Instruction(32 bit) = opcode(6) + R1(5) + R2(5) + Immediate Operand(x)
\end{questyle}

\begin{comment}

\begin{questyle}
  \question  zzz  (GATE-zzz)

  \begin{choices}
    \choice         zzz
    \choice         zzz
    \choice         zzz
    \choice         zzz
\CorrectChoice
  \end{choices}
\end{questyle}

\begin{questyle}
  \question  zzz  (GATE-zzz)

  \begin{choices}
    \choice         zzz
    \choice         zzz
    \choice         zzz
    \choice         zzz
\CorrectChoice
  \end{choices}
\end{questyle}

\begin{questyle}
  \question  zzz  (GATE-zzz)

  \begin{choices}
    \choice         zzz
    \choice         zzz
    \choice         zzz
    \choice         zzz
\CorrectChoice
  \end{choices}
\end{questyle}

\begin{questyle}
  \question  zzz  (GATE-zzz)

  \begin{choices}
    \choice         zzz
    \choice         zzz
    \choice         zzz
    \choice         zzz
\CorrectChoice
  \end{choices}
\end{questyle}

\begin{questyle}
  \question  zzz  (GATE-zzz)

  \begin{choices}
    \choice         zzz
    \choice         zzz
    \choice         zzz
    \choice         zzz
\CorrectChoice
  \end{choices}
\end{questyle}

\begin{questyle}
  \question  zzz  (GATE-zzz)

  \begin{choices}
    \choice         zzz
    \choice         zzz
    \choice         zzz
    \choice         zzz
\CorrectChoice
  \end{choices}
\end{questyle}

\begin{questyle}
  \question  zzz  (GATE-zzz)

  \begin{choices}
    \choice         zzz
    \choice         zzz
    \choice         zzz
    \choice         zzz
\CorrectChoice
  \end{choices}
\end{questyle}

\begin{questyle}
  \question  zzz  (GATE-zzz)

  \begin{choices}
    \choice         zzz
    \choice         zzz
    \choice         zzz
    \choice         zzz
\CorrectChoice
  \end{choices}
\end{questyle}

\begin{questyle}
  \question  zzz  (GATE-zzz)

  \begin{choices}
    \choice         zzz
    \choice         zzz
    \choice         zzz
    \choice         zzz
\CorrectChoice
  \end{choices}
\end{questyle}

\begin{questyle}
  \question  zzz  (GATE-zzz)

  \begin{choices}
    \choice         zzz
    \choice         zzz
    \choice         zzz
    \choice         zzz
\CorrectChoice
  \end{choices}
\end{questyle}

\begin{questyle}
  \question  zzz  (GATE-zzz)

  \begin{choices}
    \choice         zzz
    \choice         zzz
    \choice         zzz
    \choice         zzz
\CorrectChoice
  \end{choices}
\end{questyle}

\begin{questyle}
  \question  zzz  (GATE-zzz)

  \begin{choices}
    \choice         zzz
    \choice         zzz
    \choice         zzz
    \choice         zzz
\CorrectChoice
  \end{choices}
\end{questyle}

\begin{questyle}
  \question  zzz  (GATE-zzz)

  \begin{choices}
    \choice         zzz
    \choice         zzz
    \choice         zzz
    \choice         zzz
\CorrectChoice
  \end{choices}
\end{questyle}

\begin{questyle}
  \question  zzz  (GATE-zzz)

  \begin{choices}
    \choice         zzz
    \choice         zzz
    \choice         zzz
    \choice         zzz
\CorrectChoice
  \end{choices}
\end{questyle}

\begin{questyle}
  \question  zzz  (GATE-zzz)

  \begin{choices}
    \choice         zzz
    \choice         zzz
    \choice         zzz
    \choice         zzz
\CorrectChoice
  \end{choices}
\end{questyle}

\begin{questyle}
  \question  zzz  (GATE-zzz)

  \begin{choices}
    \choice         zzz
    \choice         zzz
    \choice         zzz
    \choice         zzz
\CorrectChoice
  \end{choices}
\end{questyle}

\begin{questyle}
  \question  zzz  (GATE-zzz)

  \begin{choices}
    \choice         zzz
    \choice         zzz
    \choice         zzz
    \choice         zzz
\CorrectChoice
  \end{choices}
\end{questyle}

\begin{questyle}
  \question  zzz  (GATE-zzz)

  \begin{choices}
    \choice         zzz
    \choice         zzz
    \choice         zzz
    \choice         zzz
\CorrectChoice
  \end{choices}
\end{questyle}

\begin{questyle}
  \question  zzz  (GATE-zzz)

  \begin{choices}
    \choice         zzz
    \choice         zzz
    \choice         zzz
    \choice         zzz
\CorrectChoice
  \end{choices}
\end{questyle}

\begin{questyle}
  \question  zzz  (GATE-zzz)

  \begin{choices}
    \choice         zzz
    \choice         zzz
    \choice         zzz
    \choice         zzz
\CorrectChoice
  \end{choices}
\end{questyle}

\begin{questyle}
  \question  zzz  (GATE-zzz)

  \begin{choices}
    \choice         zzz
    \choice         zzz
    \choice         zzz
    \choice         zzz
\CorrectChoice
  \end{choices}
\end{questyle}

\begin{questyle}
  \question  zzz  (GATE-zzz)

  \begin{choices}
    \choice         zzz
    \choice         zzz
    \choice         zzz
    \choice         zzz
\CorrectChoice
  \end{choices}
\end{questyle}

\end{comment}
