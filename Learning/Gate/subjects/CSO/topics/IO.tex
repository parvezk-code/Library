
\centerline{\textbf{ \LARGE IO }}

\begin{questyle}
  \question  On a non-pipelined sequential processor, a program segment, which is a part of the
             interrupt service routine, is given to transfer 500 bytes from an I/O device to memory.  (GATE-2011)
              \begin{lstlisting}
                    Initialize the address register
                    Initialize the count to 500
              LOOP: Load a byte from device
                    Store in memory at address given by address register
                    Increment the address register
                    Decrement the count
                    If count != 0 go to LOOP
              \end{lstlisting}
              Assume that each statement in this program is equivalent to machine instruction which
              takes one clock cycle to execute if it is a non-load/store instruction. The load-store
              instructions take two clock cycles to execute. The designer of the system also has an
              alternate approach of using DMA controller to implement the same transfer. The DMA
              controller requires 20 clock cycles for initialization and other overheads. Each DMA
              transfer cycle takes two clock cycles to transfer one byte of data from the device to
              the memory. What is the approximate speedup when the DMA controller based design is
              used in place of the interrupt driven program based input-output?
  \begin{choices}
    \CorrectChoice  3.4
    \choice         4.4
    \choice         5.1
    \choice         6.7
  \end{choices}
\end{questyle}
