
\centerline{\textbf{ \LARGE IO }}

\begin{questyle}
  \question  For the daisy chain scheme of connecting I/O devices, which of the following statement is true?  (GATE-1996)

  \begin{choices}
    \CorrectChoice  It gives non-uniform priority to various devices
    \choice         It gives uniform priority to all devices
    \choice         It is only useful for connecting slow devices to a processor
    \choice         It requires a separate interrupt pin on the processor for each device
  \end{choices}
\end{questyle}

\begin{questyle}
  \question  The correct matching for the following pairs is  (GATE-1997)
             \begin{multicols}{2}
                \item[P.] DMA I/O
                \item[Q.] Cache
                \item[R.] Interrupt I/O
                \item[S.] Condition Code Register
                \item[1.] High speed RAM
                \item[2.] Disk
                \item[3.] Printer
                \item[4.] ALU
             \end{multicols}
  \begin{oneparchoices}
    \choice         p-4, q-3, r-1, s-2
    \CorrectChoice  p-2, q-1, r-3, s-4
    \choice         p-4, q-3, r-2, s-1
    \choice         p-2, q-3, r-4, s-1
  \end{oneparchoices}
\end{questyle}


\begin{questyle}
  \question  Normally user programs are prevented from handling I/O directly by I/O instructions in them.
  For CPUs having explicit I/O instructions, such I/O protection is ensured by having the I/O instructions
  privileged. In a CPU with memory mapped I/O, there is no explicit I/O instruction. Which one of the
  following is true for a CPU with memory mapped I/O?  (GATE-2005)

  \begin{choices}
    \CorrectChoice  I/O protection is ensured by operating system routine (s)
    \choice         I/O protection is ensured by a hardware trap
    \choice         I/O protection is ensured during system configuration
    \choice         I/O protection is not possible
  \end{choices}
\end{questyle}

\begin{questyle}
  \question  On a non-pipelined sequential processor, a program segment, which is a part of the
             interrupt service routine, is given to transfer 500 bytes from an I/O device to memory.  (GATE-2011)
              \begin{lstlisting}
                    Initialize the address register
                    Initialize the count to 500
              LOOP: Load a byte from device
                    Store in memory at address given by address register
                    Increment the address register
                    Decrement the count
                    If count != 0 go to LOOP
              \end{lstlisting}
              Assume that each statement in this program is equivalent to machine instruction which
              takes one clock cycle to execute if it is a non-load/store instruction. The load-store
              instructions take two clock cycles to execute. The designer of the system also has an
              alternate approach of using DMA controller to implement the same transfer. The DMA
              controller requires 20 clock cycles for initialization and other overheads. Each DMA
              transfer cycle takes two clock cycles to transfer one byte of data from the device to
              the memory. What is the approximate speedup when the DMA controller based design is
              used in place of the interrupt driven program based input-output?
  \begin{oneparchoices}
    \CorrectChoice  3.4
    \choice         4.4
    \choice         5.1
    \choice         6.7
  \end{oneparchoices}
\end{questyle}


\begin{questyle}
  \question  A device with data transfer rate 10 KB/sec is connected to a CPU. Data is
  transferred byte-wise. Let the interrupt overhead be 4 microsec. The byte transfer time
  between the device interface register and CPU or memory is negligible. What is the
  minimum performance gain of operating the device under interrupt mode over operating
  it under program controlled mode?  (GATE-2005)
  \begin{oneparchoices}
    \choice         15
    \choice         25
    \choice         35
    \CorrectChoice  45
  \end{oneparchoices}
\end{questyle}


\begin{questyle}
  \question  Consider a disk drive with the specifications- 16 surfaces, 512 tracks/surface,
  512 sectors/track, 1 KB/sector, rotation speed 3000 rpm. The disk is operated in cycle
  stealing mode whereby whenever one byte word is ready it is sent to memory; similarly,
  for writing, the disk interface reads a 4 byte word from the memory in each DMA cycle.
  Memory cycle time is 40 nsec. The maximum percentage of time that the CPU gets blocked
  during DMA operation is: (GATE-2005)

  \begin{oneparchoices}
    \choice         10
    \CorrectChoice  25
    \choice         40
    \choice         50
  \end{oneparchoices}
\end{questyle}
  \\ Disk reads one full track(512*1 KB) in one rotation(60/3000 sec)
  \\ Time taken to produce 4 bytes by Disk = 153 ns \(\thickapprox\) 4 cycles
  \\ Time to transfer 4 bytes to memory    = 1 cycle
  \\ For DMA cycle stealing, data production and transfer goes in parallel. Ans = 1/4 = 25\%


\begin{questyle}
  \question  The size of the data count register of a DMA controller is 16 bits. The processor needs
  to transfer a file of 29,154 kilobytes from disk to main memory. The memory is byte addressable.
  The minimum number of times the DMA controller needs to get the control of the system bus
  from the processor to transfer the file from the disk to main memory is \fillin[456] (GATE-2016\_Set\_1)
  \\ Hint: DMA buffer size = \( \text{2}^\text{16} \) = 64 KB
  \\ Hint: File Size = 29154 KB \qquad No of time DMA need buses = 29154/64 = 456
\end{questyle}

















