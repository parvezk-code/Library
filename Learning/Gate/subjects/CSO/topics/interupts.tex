
\centerline{\textbf{ \LARGE Interupts }}

\begin{questyle}
  \question  A device employing INTR line for device interrupt. It puts the CALL instruction
             on the data bus while  (GATE-2002)

  \begin{oneparchoices}
    \CorrectChoice  \( \overline{INTA} \) is active
    \choice         HOLD is active
    \choice         READY is active
    \choice         None of these
  \end{oneparchoices}
  \\ Hint : \( \overline{INTA} \)  is interupt acknowledgement signal
\end{questyle}


\begin{questyle}
  \question  In a vectored interrupt:  (GATE-1995)
  \begin{choices}
    \choice         The branch address is assigned to a fixed location in memory
    \CorrectChoice  The interrupting source supplies the branch information to the processor through an interrupt vector
    \choice         The branch address is obtained from a register in the processor
    \choice         None of the above
  \end{choices}
\end{questyle}

\begin{questyle}
  \question  Which of the following devices should get higher priority in assigning interrupts?  (GATE-1998)

  \begin{oneparchoices}
    \choice         Hard disk
    \choice         Printer
    \choice         Keyboard
    \choice         Floppy disk
  \end{oneparchoices}
\end{questyle}


\begin{questyle}
  \question  Which of the following is true?  (GATE-1998)

  \begin{choices}
    \CorrectChoice  Unless enabled, a CPU will not be able to process interrupts.
    \choice         Loop instructions cannot be interrupted till they complete.
    \choice         A processor checks for interrupts before executing a new instruction.
    \choice         Only level triggered interrupts are possible in microprocessors.
  \end{choices}
\end{questyle}

\begin{questyle}
  \question  A processor needs software interrupt to  (GATE-2001)

  \begin{choices}
    \choice         test the interrupt system of the processor
    \choice         implement co-routines
    \CorrectChoice  obtain system services which need execution of privileged instructions
    \choice         return from subroutine
  \end{choices}
\end{questyle}


\begin{questyle}
  \question  Which one of the following is true for a CPU having a single interrupt request
  line and a single interrupt grant line?  (GATE-2005)

  \begin{choices}
    \choice         Neither vectored interrupt nor multiple interrupting devices are possible.
    \choice         Vectored interrupts are not possible but multiple interrupting devices are possible.
    \CorrectChoice  Vectored interrupts and multiple interrupting devices are both possible.
    \choice         Vectored interrupt is possible but multiple in­terrupting devices are not possible.
  \end{choices}
\end{questyle}


\begin{questyle}
  \question  A CPU generally handles an interrupt by executing an interrupt service routine  (GATE-2009)

  \begin{choices}
    \choice         As soon as an interrupt is raised.
    \choice         By checking the interrupt register at the end of fetch cycle.
    \CorrectChoice  By checking the interrupt register after finishing the execution of the current instruction.
    \choice         By checking the interrupt register at fixed time intervals.
  \end{choices}
\end{questyle}

\begin{questyle}
  \question  A computer handles several interrupt sources of which the following are relevant for this
  question. Which one of these will be handled at the HIGHEST priority? (GATE-2011)

  \begin{choices}
    \choice         Interrupt from Hard Disk, when a disk read is completed
    \choice         Interrupt from Mouse operations
    \choice         Interrupt from Keyboard operations
    \CorrectChoice  Interrupt from CPU temperature sensor, when CPU temperature is too high
  \end{choices}
\end{questyle}


