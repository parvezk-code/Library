
\centerline{\textbf{ \LARGE Addressing Modes}}


\begin{questyle}
  \question  Which of the following addressing modes permits relocation without any
             change whatsoever in the code?  (GATE-1998)

  \begin{choices}
    \choice         Indirect addressing
    \choice         Indexed addressing
    \choice         Base register addressing \qquad Hint: May need code change
    \CorrectChoice  PC relative addressing
  \end{choices}
\end{questyle}

\begin{questyle}
  \question  Which of the following addressing modes are suitable for program relocation at run time ?  (GATE-2004)

  \begin{choices}
    \choice         Absolute addressing, Indirect addressing
    \choice         Absolute addressing, Based addressing
    \CorrectChoice  Based addressing, Relative addressing
    \choice         Absolute, Based and Indirect addressing
  \end{choices}
\end{questyle}

\begin{questyle}
  \question  Match the following  (GATE-2000)

  \begin{multiColList}[2]
    \item[X:] Indirect addressing
    \item[Y:] Immediate addressing
    \item[Z:] Auto decrement addressing
    \item[1:] Loops
    \item[2:] Pointers
    \item[3:] Constants
  \end{multiColList}

  \begin{oneparchoices}
    \choice         X-3, Y-2, Z-1
    \choice         X-I, Y-3, Z-2
    \CorrectChoice  X-2, Y-3, Z-1
    \choice         X-3, Y-l, Z-2
  \end{oneparchoices}
\end{questyle}

\begin{questyle}
  \question  Match the following  (GATE-2001)

  \begin{multiColList}[2]
    \item[X:] Indirect addressing
    \item[Y:] Indexed Addressing
    \item[Z:] Base Register Addressing
    \item[1:] Array implementation
    \item[2:] Writing re-locatable code
    \item[3:] Passing array as parameter
  \end{multiColList}

  \begin{oneparchoices}
    \CorrectChoice  (X, III) (Y, I) (Z, II)
    \choice         (X, II) (Y, III) (Z, I)
    \choice         (X, III) (Y, II) (Z, I)
    \choice         (X, I) (Y, III) (Z, II)
  \end{oneparchoices}
\end{questyle}

\begin{questyle}
  \question  In the absolute addressing mode  (GATE-2002)

  \begin{choices}
    \choice         the operand is inside the instruction
    \CorrectChoice  the address of the operand is inside the instruction
    \choice         the register containing address of the operand is specified inside the instruction
    \choice         the location of the operand is implicit
  \end{choices}
\end{questyle}

\begin{questyle}
  \question  Consider the C struct defines below. The base address of student is available in
             register R1. The field student.grade can be accessed efficiently using  (GATE-2017\_Set\_1)
             \lstinputlisting[language=C, firstline=4, lastline=9]{cso_programs.c}

  \begin{choices}
    \choice         Post-increment addressing mode. (R1)+
    \choice         Pre-decrement addressing mode, -(R1)
    \choice         Register direct addressing mode, R1
    \CorrectChoice  Index addressing mode, X(R1), where X is an offset represented in 2’s complement 16-bit representation.
  \end{choices}
\end{questyle}

\begin{questyle}
  \question  Match each of the high level language statements given on the left hand side with
            the most natural addressing mode from those listed on the right hand side.  (GATE-2005)

  \begin{multiColList}[2]
    \item[1:] A[1] = B[J];
    \item[2:] while [*A++];
    \item[3:] int temp = *x;
    \item[A:] Indirect addressing
    \item[B:] Indexed addressing
    \item[C:] Autoincrement
  \end{multiColList}

  \begin{oneparchoices}
    \choice         (1, c), (2, b), (3, a)
    \choice         (1, a), (2, c), (3, b)
    \CorrectChoice  (1, b), (2, c), (3, a)
    \choice         (1, a), (2, b), (3, c)
  \end{oneparchoices}
\end{questyle}


\begin{questyle}
  \question  Consider a three word machine instruction. \\  ADD A[R0], @ B \\  The first operand
  (destination) “A [R0]” uses indexed addressing mode with R0 as the index register. The second
  operand (source) “@ B” uses indirect addressing mode. A and B are memory addresses residing at
  the second and the third words, respectively. The first word of the instruction specifies the
  opcode, the index register designation and the source and destination addressing modes. During
  execution of ADD instruction, the two operands are added and stored in the destination (first operand).
  The number of memory cycles needed during the execution cycle of the instruction is (GATE-2005)

  \begin{oneparchoices}
    \choice         3
    \CorrectChoice  4
    \choice         5
    \choice         6
  \end{oneparchoices}
  \\ Hint : Fetch Operand1(Indexed Addr) need 1 memory cycle. \\ Hint : Fetch Operand2(Indrect Addr) need 2 memory cycles. \\
            Hint : Memory Result(Indexed Addr) need 1 memory cycle.
\end{questyle}


\begin{questyle}
  \question  Which of the following is/are true of the auto-increment addressing mode?  (GATE-zzz)

  \begin{choices}
    \choice         It is useful in creating self-relocating code.
    \choice         If it is included in an Instruction Set Architecture, then an additional ALU is required
                    for effective address calculation.
    \CorrectChoice  The amount of increment depends on the size of the data item accessed.
    \choice         None
  \end{choices}
\end{questyle}

\begin{questyle}

  \question To change mode from privileged to non-privileged. (GATE - 2001 )

  \begin{choices}
    \choice A hardware interupt is needed.
    \choice A software interupt is needed.
    \choice A privileged instruction (which does not generate an interrupt) is needed.
    \CorrectChoice A non-privileged instruction (which does not generate an interrupt is needed).
  \end{choices}

  \end{questyle}


\begin{questyle}
  \question  Which of the following must be true for the RFE (Return from Exception) instruction
             on a general purpose processor?  (GATE-2008)

             \begin{enumerate}
                  \item It must be a trap instruction
                  \item It must be a privileged instruction
                  \item An exception cannot be allowed to occur during execution of an RFE instruction
              \end{enumerate}

  \begin{oneparchoices}
    \choice         I only
    \choice         II only
    \choice         I and II only
    \CorrectChoice  I, II and III only
  \end{oneparchoices}
\end{questyle}


\begin{questyle}
  \question  Consider a hypothetical processor with an instruction of type LW R1, 20(R2), which
             during execution reads a 32-bit word from memory and stores it in a 32-bit register R1.
             The effective address of the memory location is obtained by the addition of a
             constant 20 and the contents of register R2. Which of the following best reflects
             the addressing mode implemented by this instruction for operand in memory?  (GATE-2011)

  \begin{choices}
    \choice         Immediate Addressing
    \choice         Register Addressing
    \choice         Register Indirect Scaled Addressing
    \CorrectChoice  Base Indexed Addressing
  \end{choices}
\end{questyle}

\begin{comment}
