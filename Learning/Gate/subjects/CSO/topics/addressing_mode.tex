
\centerline{\textbf{ \LARGE Addressing Modes}}


\begin{questyle}
  \question  Which of the following addressing modes permits relocation without any
             change whatsoever in the code?  (GATE-1998)

  \begin{choices}
    \choice         Indirect addressing
    \choice         Indexed addressing
    \choice         Base register addressing \qquad Hint: May need code change
    \CorrectChoice  PC relative addressing
  \end{choices}
\end{questyle}

\begin{questyle}
  \question  Which of the following addressing modes are suitable for program relocation at run time ?  (GATE-2004)

  \begin{choices}
    \choice         Absolute addressing, Indirect addressing
    \choice         Absolute addressing, Based addressing
    \CorrectChoice  Based addressing, Relative addressing
    \choice         Absolute, Based and Indirect addressing
  \end{choices}
\end{questyle}

\begin{questyle}
  \question  Match the following  (GATE-2000)

  \begin{multiColList}[2]
    \item[X:] Indirect addressing
    \item[Y:] Immediate addressing
    \item[Z:] Auto decrement addressing
    \item[1:] Loops
    \item[2:] Pointers
    \item[3:] Constants
  \end{multiColList}

  \begin{oneparchoices}
    \choice         X-3, Y-2, Z-1
    \choice         X-I, Y-3, Z-2
    \CorrectChoice  X-2, Y-3, Z-1
    \choice         X-3, Y-l, Z-2
  \end{oneparchoices}
\end{questyle}


\begin{questyle}
  \question  In the absolute addressing mode  (GATE-2002)

  \begin{choices}
    \choice         the operand is inside the instruction
    \CorrectChoice  the address of the operand is inside the instruction
    \choice         the register containing address of the operand is specified inside the instruction
    \choice         the location of the operand is implicit
  \end{choices}
\end{questyle}

\begin{questyle}
  \question  Consider the C struct defines below. The base address of student is available in
             register R1. The field student.grade can be accessed efficiently using  (GATE-2017\_Set\_1)
             \begin{lstlisting}[style=mystyle]
                struct data
                {
                    int marks [100] ;
                    char grade;
                    int cnumber;
                } student;
             \end{lstlisting}

  \begin{choices}
    \choice         Post-increment addressing mode. (R1)+
    \choice         Pre-decrement addressing mode, -(R1)
    \choice         Register direct addressing mode, R1
    \CorrectChoice  Index addressing mode, X(R1), where X is an offset represented in 2’s complement 16-bit representation.
  \end{choices}
\end{questyle}

\begin{comment}

\begin{questyle}
  \question  zzz  (GATE-zzz)

  \begin{choices}
    \choice         zzz
    \choice         zzz
    \choice         zzz
    \choice         zzz
\CorrectChoice
  \end{choices}
\end{questyle}

\begin{questyle}
  \question  zzz  (GATE-zzz)

  \begin{choices}
    \choice         zzz
    \choice         zzz
    \choice         zzz
    \choice         zzz
\CorrectChoice
  \end{choices}
\end{questyle}

\begin{questyle}
  \question  zzz  (GATE-zzz)

  \begin{choices}
    \choice         zzz
    \choice         zzz
    \choice         zzz
    \choice         zzz
\CorrectChoice
  \end{choices}
\end{questyle}

\begin{questyle}
  \question  zzz  (GATE-zzz)

  \begin{choices}
    \choice         zzz
    \choice         zzz
    \choice         zzz
    \choice         zzz
\CorrectChoice
  \end{choices}
\end{questyle}

\begin{questyle}
  \question  zzz  (GATE-zzz)

  \begin{choices}
    \choice         zzz
    \choice         zzz
    \choice         zzz
    \choice         zzz
\CorrectChoice
  \end{choices}
\end{questyle}

\begin{questyle}
  \question  zzz  (GATE-zzz)

  \begin{choices}
    \choice         zzz
    \choice         zzz
    \choice         zzz
    \choice         zzz
\CorrectChoice
  \end{choices}
\end{questyle}

\begin{questyle}
  \question  zzz  (GATE-zzz)

  \begin{choices}
    \choice         zzz
    \choice         zzz
    \choice         zzz
    \choice         zzz
\CorrectChoice
  \end{choices}
\end{questyle}

\begin{questyle}
  \question  zzz  (GATE-zzz)

  \begin{choices}
    \choice         zzz
    \choice         zzz
    \choice         zzz
    \choice         zzz
\CorrectChoice
  \end{choices}
\end{questyle}

\begin{questyle}
  \question  zzz  (GATE-zzz)

  \begin{choices}
    \choice         zzz
    \choice         zzz
    \choice         zzz
    \choice         zzz
\CorrectChoice
  \end{choices}
\end{questyle}

\begin{questyle}
  \question  zzz  (GATE-zzz)

  \begin{choices}
    \choice         zzz
    \choice         zzz
    \choice         zzz
    \choice         zzz
\CorrectChoice
  \end{choices}
\end{questyle}

\begin{questyle}
  \question  zzz  (GATE-zzz)

  \begin{choices}
    \choice         zzz
    \choice         zzz
    \choice         zzz
    \choice         zzz
\CorrectChoice
  \end{choices}
\end{questyle}

\begin{questyle}
  \question  zzz  (GATE-zzz)

  \begin{choices}
    \choice         zzz
    \choice         zzz
    \choice         zzz
    \choice         zzz
\CorrectChoice
  \end{choices}
\end{questyle}

\begin{questyle}
  \question  zzz  (GATE-zzz)

  \begin{choices}
    \choice         zzz
    \choice         zzz
    \choice         zzz
    \choice         zzz
\CorrectChoice
  \end{choices}
\end{questyle}

\begin{questyle}
  \question  zzz  (GATE-zzz)

  \begin{choices}
    \choice         zzz
    \choice         zzz
    \choice         zzz
    \choice         zzz
\CorrectChoice
  \end{choices}
\end{questyle}

\begin{questyle}
  \question  zzz  (GATE-zzz)

  \begin{choices}
    \choice         zzz
    \choice         zzz
    \choice         zzz
    \choice         zzz
\CorrectChoice
  \end{choices}
\end{questyle}

\begin{questyle}
  \question  zzz  (GATE-zzz)

  \begin{choices}
    \choice         zzz
    \choice         zzz
    \choice         zzz
    \choice         zzz
\CorrectChoice
  \end{choices}
\end{questyle}

\begin{questyle}
  \question  zzz  (GATE-zzz)

  \begin{choices}
    \choice         zzz
    \choice         zzz
    \choice         zzz
    \choice         zzz
\CorrectChoice
  \end{choices}
\end{questyle}

\begin{questyle}
  \question  zzz  (GATE-zzz)

  \begin{choices}
    \choice         zzz
    \choice         zzz
    \choice         zzz
    \choice         zzz
\CorrectChoice
  \end{choices}
\end{questyle}

\begin{questyle}
  \question  zzz  (GATE-zzz)

  \begin{choices}
    \choice         zzz
    \choice         zzz
    \choice         zzz
    \choice         zzz
\CorrectChoice
  \end{choices}
\end{questyle}

\begin{questyle}
  \question  zzz  (GATE-zzz)

  \begin{choices}
    \choice         zzz
    \choice         zzz
    \choice         zzz
    \choice         zzz
\CorrectChoice
  \end{choices}
\end{questyle}

\end{comment}
