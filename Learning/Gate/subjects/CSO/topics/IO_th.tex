
\centerline{\textbf{ \LARGE IO }}

\begin{enumerate}
    \item Modes of IO : Programmed, Interupt, DMA

\begin{tikzpicture}[node distance=2cm]

    \node (ProStart)  [flowchart-start] {Start Programmed};
    \node (ProIOReady) [flowchart-process, below of=ProStart] {Busy Wait for IO Ready \\ CPU to IO};
    \node (ProWordReady) [flowchart-process, below of=ProIOReady] {Busy Wait for IO Word \\ CPU to IO};
    \node (ProREAD) [flowchart-process, below of=ProWordReady] {Read Word};
    \node (ProWrite) [flowchart-process, below of=ProREAD] {Write Word to Memory};
    \node (ProDone) [flowchart-decision, below of=ProWrite] {Finish};
    \node (ProStop) [flowchart-start, below of=ProDone] {Stop};
    \draw [flowchart-arrow] (ProStart) -- (ProIOReady);
    \draw [flowchart-arrow] (ProIOReady) -- (ProWordReady);
    \draw [flowchart-arrow] (ProWordReady) -- (ProREAD);
    \draw [flowchart-arrow] (ProREAD) -- (ProWrite);
    \draw [flowchart-arrow] (ProWrite) -- (ProDone);
    \draw [flowchart-arrow] (ProDone.west) -- +(-1,0) |- (ProStart.west);
    \draw [flowchart-arrow] (ProDone) -- (ProStop);


    \node (IntStart) at (5,0)  [flowchart-start] {Start Interupt};
    \node (IntIOReady) [flowchart-process, below of=IntStart] {Send Read command \\ CPU to IO};
    \node (IntWordReady) [flowchart-process, below of=IntIOReady] {Busy Wait for IO Word \\ CPU to IO};
    \node (IntREAD) [flowchart-process, below of=IntWordReady] {Read Word};
    \node (IntWrite) [flowchart-process, below of=IntREAD] {Write Word to Memory};
    \node (IntDone) [flowchart-decision, below of=IntWrite] {Finish};
    \node (IntStop) [flowchart-start, below of=IntDone] {Stop};
    \draw [flowchart-arrow] (IntStart) -- (IntIOReady);
    \draw [flowchart-arrow] (IntIOReady.south) -- ++(0,-0.5) -- ++(3,0) node[midway,above] {Do something};
    \draw [flowchart-arrow] (IntWordReady)  ++(4,0) --  (IntWordReady) node[midway,above] {Interupt} node[midway,below] {IO Ready};
    \draw [flowchart-arrow] (IntWordReady) -- (IntREAD);
    \draw [flowchart-arrow] (IntREAD) -- (IntWrite);
    \draw [flowchart-arrow] (IntWrite) -- (IntDone);
    \draw [flowchart-arrow] (IntDone.west) -- +(-1,0) |- (IntStart.west);
    \draw [flowchart-arrow] (IntDone) -- (IntStop);

    \node (DMAStart) at (11.5,0)  [flowchart-start] {Start DMA};
    \node (DMAReady) [flowchart-process, below of=DMAStart] {Send Read Block Cmd \\ CPU to DMA};
    \node (DMAFinish) [flowchart-process, below of=DMAReady] {Read DMA Status\\ CPU to DMA};
    \draw [flowchart-arrow] (DMAStart) -- (DMAReady);
    \draw [flowchart-arrow] (DMAReady.south) -- ++(0,-0.5) -- ++(3,0) node[midway,above] {Do something};
    \draw [flowchart-arrow] (DMAFinish)  ++(4,0) --  (DMAFinish) node[midway,above] {Interupt} node[midway,below] {DMA Done};

\end{tikzpicture}

    \item Memory mapped I/O and Isolated I/O \begin{verbatim} https://www.geeksforgeeks.org/memory-mapped-i-o-and-isolated-i-o/\end{verbatim}

    \item DMA has a data count register and data buffer. When the buffer is full DMA will request CPU
          to get controll of buses.

    \item In DMA (cycle stealing) data production by IO and data transfer to memory goes in parallel.
          IO device dont wait for data transfer to complete.
        \begin{myTableStyle} \begin{tabular}{ |m{4cm}|m{4cm}|m{3cm}| } \hline
            Time in programmed IO                       & \(\text{T}_{\text{1}}\)  &     \\ \hline
            Data production time   & \(\text{T}_{\text{produce}}\) (1 word)  &   by IO   \\ \hline
            Data transfer time     & \(\text{T}_{\text{transfer}}\) (1 word)  & to memory    \\ \hline
            Interupt overhead                           & \(\text{T}_{\text{int}}\)  &     \\ \hline
                                                        & Total Time & CPU Busy/Bocked \\ \hline
            Programmed IO                               & \(\text{T}_{\text{1}}\) & \(\text{T}_{\text{1}}\) \\ \hline

            Interupt Modes  & \(\text{T}_{\text{produce}}\) + \(\text{T}_{\text{transfer}}\) + \(\text{T}_{\text{int}}\)
                            & \(\text{T}_{\text{transfer}}\) + \(\text{T}_{\text{int}}\) \\ \hline

            DMA Cycle Stealing & \(\text{T}_{\text{produce}}\) & \(\text{T}_{\text{transfer}}\) \\ \hline
        \end{tabular} \end{myTableStyle} \vspace{0.08in}

\end{enumerate}
