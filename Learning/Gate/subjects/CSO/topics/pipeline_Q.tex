
\centerline{\textbf{ \LARGE Pipeline }}

\begin{questyle}
  \question  Comparing the time T1 taken for a single instruction on a pipelined CPU with time
            T2 taken on a non­ pipelined but identical CPU, we can say that  (GATE-2000)

  \begin{oneparchoices}
    \choice         T1 \textless= T2
    \CorrectChoice  T1 \textgreater= T2
    \choice         T1 \textless T2
    \choice         T1 = T2 + fetch time for one instruction
  \end{oneparchoices}
  \\ Hint : T1 = T2 + buffer delay
\end{questyle}


\begin{questyle}
  \question  A 4-stage pipeline has the stage delays as 150, 120, 160 and 140 nanoseconds respectively.
            Registers that are used between the stages have a delay of 5 nanoseconds each. Assuming
            constant clocking rate, the total time taken to process 1000 data items on this pipeline will be  (GATE-2004)

  \begin{oneparchoices}
    \choice         120.4 ms
    \choice         160.5 ms
    \CorrectChoice  165.5 ms
    \choice         590.0 ms
  \end{oneparchoices}
  \\ Hint : time = (k + (n-1))(160+5)
\end{questyle}


\begin{questyle}
  \question  A 5 stage pipelined CPU has the following sequence of stages:  (GATE-2005)
  \begin{enumerate}
    \item[IF-] Instruction fetch from instruction memory,
    \item[RD-] Instruction decode and register read
    \item[Ex-] Execute ALU operation for data and address computation
    \item[MA-] Data memory access - for write access, the register read
     at RD stage is used
     \item[WB] Register write back.
  \end{enumerate}
  Consider the following sequence of instructions. What is the number of clock cycles taken to complete the above sequence of instructions starting from the fetch of I1 ?
  \begin{enumerate}
    \item[I1:] \quad L R0, Loc1; \quad R0 \textless= M[Loc1]
    \item[I2:] \quad A R0, R0; \quad \; \; R0 \textless= R0 + R0
    \item[I3:] \quad S R2, R0; \quad \; \; R2 \textless= R2 - R0
  \end{enumerate}

  \begin{oneparchoices}
    \CorrectChoice  8
    \choice         10
    \choice         12
    \choice         15
  \end{oneparchoices} \\
  Hint : I2 needs 1 bubble. Total cycles = \quad k + (n-1) + b = \quad 5 +(3-1) + 1 = 8 cycles
\end{questyle}

\begin{questyle}
  \question  A CPU has a five-stage pipeline and runs at 1 GHz frequency. Instruction fetch happens
             in the first stage of the pipeline. A conditional branch instruction computes the target
             address and evaluates the condition in the third stage of the pipeline. The processor stops
             fetching new instructions following a conditional branch until the branch outcome is known.
             A program executes \(10^9\) instructions out of which 20\% are conditional branches. If each
             instruction takes one cycle to complete on average, the total execution time of the program is  (GATE-2006)
  \begin{oneparchoices} \\
    \choice         1.0 seconds
    \choice         1.2 seconds
    \CorrectChoice  1.4 seconds
    \choice         1.6 seconds
  \end{oneparchoices}\\
  Hint : Stages(k) = 5 \quad Instrctions(n) = \(10^9\) \quad Bubles(b) = (20\%\(10^9\))*2 = 0.4x\(10^9\) \\
  Hint : Total time = \quad (k + (n-1) + b) x \(10^{-9} \thickapprox \) (n+b)x\(10^{-9} \) = 1.4 \\
  Hint : Total time = (Cycles per instruction).n.\(10^{-9}\) = (1 + 0.2x2).\(10^9\).\(10^{-9}\) = 1.4
\end{questyle}

\begin{questyle}
  \question  Consider a pipelined processor with the following four stages. (GATE-2007)

  \begin{enumerate}
    \item[IF] Instruction Fetch
    \item[ID] Instruction Decode and Operand Fetch
    \item[Ex] Execute
    \item[WB] Write Back
  \end{enumerate}
  The IF, ID and WB stages take one clock cycle each to complete the operation. The number of clock
  cycles for the EX stage dependson the instruction. The ADD and SUB instructions need 1 clock cycle
  and the MUL instruction needs 3 clock cycles in the EX stage. Operand forwarding is used in the
  pipelined processor. What is the number of clock cycles taken to complete the following
  sequence of instructions?

  \begin{enumerate}
    \item[ADD] R2, R1, R0 \qquad R2 \(\leftarrow\) R0 + R1
    \item[MUL] R4, R3, R2 \qquad R4 \(\leftarrow\) R3 * R2
    \item[SUB] R6, R5, R4 \qquad R6 \(\leftarrow\) R5 - R4
  \end{enumerate}

  \begin{oneparchoices}
    \choice         7
    \CorrectChoice  8
    \choice         10
    \choice         14
  \end{oneparchoices} \\
  Hint : Total cycles  = \quad k + (n-1) + c = \quad4 + (3-1) + 2 = \quad 8 cycles
\end{questyle}

\begin{questyle}
  \question Consider a 4 stage pipeline processor.   The number of cycles needed by the four
             instructions I1, I2, I3, I4 in stages S1, S2, S3, S4 is shown below. What is the number of cycles needed to execute the loop - for(i=1 to 2) {I1; I2; I3; I4;}?

             \begin{myTableStyle} \begin{tabular}{ |m{1cm}|m{1cm}|m{1cm}|m{1cm}|m{1cm}| } \hline
                   & S1 & S2 & S3 & S4 \\ \hline
                I1 & 2  & 1  & 1  & 1  \\ \hline
                I2 & 1  & 3  & 2  & 2  \\ \hline
                I3 & 2  & 1  & 1  & 3  \\ \hline
                I4 & 1  & 2  & 2  & 2  \\ \hline
              \end{tabular} \end{myTableStyle} \vspace{0.08in}

  \begin{oneparchoices}
    \choice         16
    \CorrectChoice  23
    \choice         28
    \choice         30
  \end{oneparchoices} \\  Hint : \quad S[ i ] [ j ] = max( top, left ) + stage\_time \\

   \begin{myTableStyle} \begin{tabular}{ |m{1cm}|m{1cm}|m{1cm}|m{1cm}|m{1cm}| } \hline
                   & S1 & S2 & S3  & S4 \\ \hline
                I1 & 2  & 3  & 4   & 5  \\ \hline
                I2 & 3  & 6  & 8   & 10  \\ \hline
                I3 & 5  & 7  & 9   & 13  \\ \hline
                I4 & 6  & 9  & 11  & 15  \\ \hline
                I1 & 8  & 10 & 12  & 16  \\ \hline
                I2 & 9  & 13 & 15  & 18  \\ \hline
                I3 & 11 & 14 & 16  & 21  \\ \hline
                I4 & 12 & 16 & 18  & 23  \\ \hline
              \end{tabular} \end{myTableStyle} \vspace{0.08in}

\end{questyle}


\begin{questyle}
  \question  A 5-stage pipelined processor has Instruction Fetch(IF),Instruction Decode(ID),Operand
             Fetch(OF),Perform Operation(PO)and Write Operand(WO)stages.The IF,ID,OF and WO stages
             take 1 clock cycle each for any instruction.The PO stage takes 1 clock cycle for ADD
             and SUB instructions,3 clock cycles for MUL instruction,and 6 clock cycles for DIV
             instruction respectively.Operand forwarding is used in the pipeline.What is the number
             of clock cycles needed to execute the following sequence of instructions?  (GATE-2010)

  \begin{enumerate}
    \item[I0] \quad MUL R2 ,R0 ,R1 \qquad R2 \(\leftarrow\) R0 * R1
    \item[I1] \quad DIV R5 ,R3 ,R4  \qquad R5 \(\leftarrow\) R3 / R4
    \item[I2] \quad ADD R2 ,R5 ,R2 \qquad R2 \(\leftarrow\) R5 + R2
    \item[I3] \quad SUB R5 ,R2 ,R6 \qquad R5 \(\leftarrow\) R2 - R6
  \end{enumerate}

  \begin{oneparchoices}
    \choice         13
    \CorrectChoice  15
    \choice         17
    \choice         19
  \end{oneparchoices} \\
  Hint : No bubbles because of operator forwarding and no Load/Store instruction. \\
  Hint : Only one stage(Ex) has different cycles. So following formula can be applied.\\
  Hint : TC = \; k + (n-1) + extra\_cycles = \; 5 + (4-1) + (2+5) = \; 15
\end{questyle}

\begin{questyle}
  \question  Consider an instruction pipeline with four stages (S1, S2, S3 and S4) each with
             combinational circuit only. The pipeline registers are required between each stage and
             at the end of the last stage. Delays for the stages and for the pipeline registers
             are as given in the figure.What is the approximate speed up of the pipeline in steady
             state under ideal conditions when compared to the corresponding non-pipeline implementation?  (GATE-2011)

          \begin{myTableStyle} \begin{tabular}{ |m{2cm}|m{2cm}|m{1.8cm}| } \hline
              Stage & Stage delay & Register delay  \\ \hline
              S1    & 5ns         & 1ns  \\ \hline
              S2    & 6ns         & 1ns  \\ \hline
              S3    & 11ns        & 1ns  \\ \hline
              S4    & 8ns         & 1ns  \\ \hline
          \end{tabular} \end{myTableStyle} \vspace{0.08in}

  \begin{oneparchoices}
    \choice         4.0
    \CorrectChoice  2.5
    \choice         1.1
    \choice         3.0
  \end{oneparchoices} \\
  Hint : Time for non-pipeline \( T_{np} \) = \; 5 + 6 + 11 + 8 = \; 30 \\
  Hint : Max stage delay \(T_C\) = \; 11 + 1 = \; 12 \qquad Speed up = 30/12
\end{questyle}

\begin{questyle}
  \question  Consider an instruction pipeline with five stages without any branch prediction:
   Fetch Instruction (FI), Decode Instruction (DI), Fetch Operand (FO), Execute Instruction (EI) and
   Write Operand (WO). The stage delays for FI, DI, FO, EI and WO are 5 ns, 7 ns, 10 ns, 8 ns and 6 ns,
   respectively. There are intermediate storage buffers after each stage and the delay of each
   buffer is 1 ns. A program consisting of 12 instructions I1 to I12 is executed in this pipelined
   processor. Instruction I4 is the only branch instruction and its branch target is I9.
   If the branch is taken during the execution of this program, the time (in ns) needed
   to complete the program is  (GATE-2013)

  \begin{oneparchoices}
    \choice         132
    \CorrectChoice  165
    \choice         176
    \choice         328
  \end{oneparchoices} \\
  Hint: Total instructions(n) = 8 (I1-I4, I9-I12) \quad Total Stages = 5 \quad Bubbles = 3 \quad cycle\_time = 11(10+1)\\
  Hint: Total cycles = k + (n-1) + b = \quad 5 + (8-1) + 3 = 15 \quad Time = 15 * 11

\end{questyle}


\begin{questyle}
  \question  Consider a 6-stage instruction pipeline, where all stages are perfectly balanced.
            Assume that there is no cycle-time overhead of pipelining. When an application is executing
            on this 6-stage pipeline, the speedup achieved with respect to non-pipelined execution if 25\% of
            the instructions incur 2 pipeline stall cycles is \fillin[4]. (GATE-2014\_set\_1). \\
            Hint : Time(Non-pipeline)= 6 \quad Time(Pipeline) = 1 + 0.25(2) \quad Speed-up = 6/1.5
\end{questyle}


\begin{questyle}
  \question  An instruction pipeline has five stages, namely, instruction fetch (IF), instruction
             decode and register fetch (ID/RF), instruction execution (EX), memory access (MEM), and
             register writeback (WB) with stage latencies 1 ns, 2.2 ns, 2 ns, 1 ns, and 0.75 ns,
             respectively (ns stands for nanoseconds). To gain in terms of frequency, the designers
             have decided to split the ID/RF stage into three stages (ID, RF1, RF2) each of
             latency 2.2/3 ns. Also, the EX stage is split into two stages (EX1, EX2) each of
             latency 1 ns. The new design has a total of eight pipeline stages. A program has 20\%
             branch instructions which execute in the EX stage and produce the next instruction
             pointer at the end of the EX stage in the old design and at the end of the EX2
             stage in the new design. The IF stage stalls after fetching a branch instruction
             until the next instruction pointer is computed. All instructions other than the
             branch instruction have an average CPI of one in both the designs. The execution
             times of this program on the old and the new design are P and Q nanoseconds, respectively.
             The value of P/Q is \fillin[1.54]  (GATE-2014\_set\_3) \\
             Hint : Average cycles per instruction = (1 + BPI) \quad BPI = bubbles per instruction \\
             Hint : P(\(T_{old}\)) = (1 + 0.2x2)2.2 = 3.08  \quad Q(\(T_{new}\)) = (1 + 0.2x5)1 = 2
\end{questyle}


\begin{questyle}
  \question  Consider a non-pipelined processor with a clock rate of 2.5 gigahertz and average
             cycles per instruction of four. The same processor is upgraded to a pipelined processor
             with five stages; but due to the internal pipeline delay, the clock speed is reduced to
             2 gigahertz. Assume that there are no stalls in the pipeline. The speed up achieved in
             this pipelined processor is \fillin[3.2]. (GATE-2015\_Set\_1) \\
             Hint : Non-pipeline = (4/2.5) \quad Pipeline = 1/2 \quad S = (4 x 2)/(2 x 2.5)
\end{questyle}

\begin{questyle}
  \question  Consider the sequence of machine instructions given below. In the above
             sequence, R0 to R8 are general purpose registers. In the instructions shown, the
             first register stores the result of the operation performed on the second and the
             third registers. This sequence of instructions is to be executed in a pipelined
             instruction processor with the following 4 stages: (1) Instruction Fetch and
             Decode (IF), (2) Operand Fetch (OF), (3) Perform Operation (PO) and (4) Write
             back the Result (WB). The IF, OF and WB stages take 1 clock cycle each for any
             instruction. The PO stage takes 1 clock cycle for ADD or SUB instruction, 3 clock cycles
             for MUL instruction and 5 clock cycles for DIV instruction. The pipelined processor
             uses operand forwarding from the PO stage to the OF stage. The number of clock cycles
             taken for the execution of the above sequence of instructions is \fillin[13] (GATE-2015\_set\_2) \\
             Hint : Total cycles = k + (n-1) + c \quad stages(k)=4, n=4,  extra cycles(c) = 2(mul) + 4(div) = 6

             \begin{enumerate}
                \item MUL R5, R0, R1
                \item DIV R6, R2, R3
                \item ADD R7, R5, R6
                \item SUB R8, R7, R4
            \end{enumerate}
\end{questyle}

\begin{comment}

\begin{questyle}
  \question  zzz  (GATE-zzz)

  \begin{choices}
    \choice         zzz
    \choice         zzz
    \choice         zzz
    \choice         zzz
    \CorrectChoice
  \end{choices}
\end{questyle}


  \begin{choices}
    \choice         zzz
    \choice         zzz
    \choice         zzz
    \choice         zzz
    \CorrectChoice
  \end{choices}
\end{questyle}

\begin{questyle}
  \question  zzz  (GATE-zzz)

  \begin{choices}
    \choice         zzz
    \choice         zzz
    \choice         zzz
    \choice         zzz
    \CorrectChoice
  \end{choices}
\end{questyle}

\begin{questyle}
  \question  zzz  (GATE-zzz)

  \begin{choices}
    \choice         zzz
    \choice         zzz
    \choice         zzz
    \choice         zzz
    \CorrectChoice
  \end{choices}
\end{questyle}


  \begin{choices}
    \choice         zzz
    \choice         zzz
    \choice         zzz
    \choice         zzz
    \CorrectChoice
  \end{choices}
\end{questyle}

\begin{questyle}
  \question  zzz  (GATE-zzz)

  \begin{choices}
    \choice         zzz
    \choice         zzz
    \choice         zzz
    \choice         zzz
    \CorrectChoice
  \end{choices}
\end{questyle}

\begin{questyle}
  \question  zzz  (GATE-zzz)

  \begin{choices}
    \choice         zzz
    \choice         zzz
    \choice         zzz
    \choice         zzz
    \CorrectChoice
  \end{choices}
\end{questyle}


  \begin{choices}
    \choice         zzz
    \choice         zzz
    \choice         zzz
    \choice         zzz
    \CorrectChoice
  \end{choices}
\end{questyle}

\begin{questyle}
  \question  zzz  (GATE-zzz)

  \begin{choices}
    \choice         zzz
    \choice         zzz
    \choice         zzz
    \choice         zzz
    \CorrectChoice
  \end{choices}
\end{questyle}

\begin{questyle}
  \question  zzz  (GATE-zzz)

  \begin{choices}
    \choice         zzz
    \choice         zzz
    \choice         zzz
    \choice         zzz
    \CorrectChoice
  \end{choices}
\end{questyle}


  \begin{choices}
    \choice         zzz
    \choice         zzz
    \choice         zzz
    \choice         zzz
    \CorrectChoice
  \end{choices}
\end{questyle}

\begin{questyle}
  \question  zzz  (GATE-zzz)

  \begin{choices}
    \choice         zzz
    \choice         zzz
    \choice         zzz
    \choice         zzz
    \CorrectChoice
  \end{choices}
\end{questyle}

\begin{questyle}
  \question  zzz  (GATE-zzz)

  \begin{choices}
    \choice         zzz
    \choice         zzz
    \choice         zzz
    \choice         zzz
    \CorrectChoice
  \end{choices}
\end{questyle}


\end{comment}


