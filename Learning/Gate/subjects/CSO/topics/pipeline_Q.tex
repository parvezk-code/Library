
\centerline{\textbf{ \LARGE Pipeline }}


\begin{questyle}
  \question  Comparing the time T1 taken for a single instruction on a pipelined CPU with time
            T2 taken on a non­ pipelined but identical CPU, we can say that  (GATE-2000)

  \begin{oneparchoices}
    \choice         T1 \textless= T2
    \CorrectChoice  T1 \textgreater= T2
    \choice         T1 \textless T2
    \choice         T1 = T2 + fetch time for one instruction
  \end{oneparchoices}
  \\ Hint : T1 = T2 + buffer delay
\end{questyle}


\begin{questyle}
  \question  A 4-stage pipeline has the stage delays as 150, 120, 160 and 140 nanoseconds respectively.
            Registers that are used between the stages have a delay of 5 nanoseconds each. Assuming
            constant clocking rate, the total time taken to process 1000 data items on this pipeline will be  (GATE-2004)

  \begin{oneparchoices}
    \choice         120.4 ms
    \choice         160.5 ms
    \CorrectChoice  165.5 ms
    \choice         590.0 ms
  \end{oneparchoices}
  \\ Hint : time = (k + (n-1))(160+5)
\end{questyle}


\begin{questyle}
  \question  A 5 stage pipelined CPU has the following sequence of stages:  (GATE-2005)
  \begin{enumerate}
    \item[IF-] Instruction fetch from instruction memory,
    \item[RD-] Instruction decode and register read
    \item[Ex-] Execute ALU operation for data and address computation
    \item[MA-] Data memory access - for write access, the register read
     at RD stage is used
     \item[WB] Register write back.
  \end{enumerate}
  Consider the following sequence of instructions. What is the number of clock cycles taken to complete the above sequence of instructions starting from the fetch of I1 ?
  \begin{enumerate}
    \item[I1:]  L R0, 1oc1; R0 <= M[1oc1]
    \item[I2:]  A R0, R0;  R0 <= R0 + R0
    \item[I3:]  S R2, R0;  R2 <= R2 - R0
  \end{enumerate}

  \begin{oneparchoices}
    \CorrectChoice  8
    \choice         10
    \choice         12
    \choice         15
  \end{oneparchoices}
\end{questyle}

\begin{questyle}
  \question  A CPU has a five-stage pipeline and runs at 1 GHz frequency. Instruction fetch happens
             in the first stage of the pipeline. A conditional branch instruction computes the target
             address and evaluates the condition in the third stage of the pipeline. The processor stops
             fetching new instructions following a conditional branch until the branch outcome is known.
             A program executes 109 instructions out of which 20% are conditional branches. If each
             instruction takes one cycle to complete on average, the total execution time of the program is  (GATE-2006)
  \begin{oneparchoices} \\
    \choice         1.0 seconds
    \choice         1.2 seconds
    \CorrectChoice  1.4 seconds
    \choice         1.6 seconds
  \end{oneparchoices}\\
  Hint : Average cycles needed(\(T_c\)) = 0.8(1) + 0.2(3), 2 bubbles implies branch instruction will take 3 cycles.
  Hint : Total time = \(T_c\) * \(10^9\) / \(10^9\) = 1.4 * \(10^9\) / \(10^9\) = 1.4
\end{questyle}

\begin{questyle}
  \question  Consider a pipelined processor with the following four stages. (GATE-2007)

  \begin{enumerate}
    \item[IF] Instruction Fetch
    \item[ID] Instruction Decode and Operand Fetch
    \item[Ex] Execute
    \item[WB] Write Back
  \end{enumerate}
  The IF, ID and WB stages
  take one clock cycle each to complete the operation. The number of clock cycles for the
  EX stage dependson the instruction. The ADD and SUB instructions need 1 clock cycle and the
  MUL instruction needs 3 clock cycles in the EX stage. Operand forwarding is used in the
  pipelined processor. What is the number of clock cycles taken to complete the following
  sequence of instructions?

  \begin{enumerate}
    \item[ADD] R2, R1, R0 \qquad R2 \(\leftarrow\) R0 + R1
    \item[MUL] R4, R3, R2 \qquad R4 \(\leftarrow\) R3 * R2
    \item[SUB] R6, R5, R4 \qquad R6 \(\leftarrow\) R5 - R4
  \end{enumerate}

  \begin{oneparchoices}
    \choice         7
    \CorrectChoice  8
    \choice         10
    \choice         14
  \end{oneparchoices} \\
  Hint : For normal execution(1 cycle each for +,-,*) : 4 + (1+0) + (1+0) = 6 cycles
  Hint : For this question : 4 + (1+2) + (1+0) = 8 cycles
\end{questyle}

\begin{comment}

\begin{questyle}
  \question  zzz  (GATE-zzz)

  \begin{choices}
    \choice         zzz
    \choice         zzz
    \choice         zzz
    \choice         zzz
    \CorrectChoice
  \end{choices}
\end{questyle}

\begin{questyle}
  \question  zzz  (GATE-zzz)

  \begin{choices}
    \choice         zzz
    \choice         zzz
    \choice         zzz
    \choice         zzz
    \CorrectChoice
  \end{choices}
\end{questyle}

\begin{questyle}
  \question  zzz  (GATE-zzz)

  \begin{choices}
    \choice         zzz
    \choice         zzz
    \choice         zzz
    \choice         zzz
    \CorrectChoice
  \end{choices}
\end{questyle}

\begin{questyle}
  \question  zzz  (GATE-zzz)

  \begin{choices}
    \choice         zzz
    \choice         zzz
    \choice         zzz
    \choice         zzz
    \CorrectChoice
  \end{choices}
\end{questyle}


  \begin{choices}
    \choice         zzz
    \choice         zzz
    \choice         zzz
    \choice         zzz
    \CorrectChoice
  \end{choices}
\end{questyle}

\begin{questyle}
  \question  zzz  (GATE-zzz)

  \begin{choices}
    \choice         zzz
    \choice         zzz
    \choice         zzz
    \choice         zzz
    \CorrectChoice
  \end{choices}
\end{questyle}

\begin{questyle}
  \question  zzz  (GATE-zzz)

  \begin{choices}
    \choice         zzz
    \choice         zzz
    \choice         zzz
    \choice         zzz
    \CorrectChoice
  \end{choices}
\end{questyle}


  \begin{choices}
    \choice         zzz
    \choice         zzz
    \choice         zzz
    \choice         zzz
    \CorrectChoice
  \end{choices}
\end{questyle}

\begin{questyle}
  \question  zzz  (GATE-zzz)

  \begin{choices}
    \choice         zzz
    \choice         zzz
    \choice         zzz
    \choice         zzz
    \CorrectChoice
  \end{choices}
\end{questyle}

\begin{questyle}
  \question  zzz  (GATE-zzz)

  \begin{choices}
    \choice         zzz
    \choice         zzz
    \choice         zzz
    \choice         zzz
    \CorrectChoice
  \end{choices}
\end{questyle}


  \begin{choices}
    \choice         zzz
    \choice         zzz
    \choice         zzz
    \choice         zzz
    \CorrectChoice
  \end{choices}
\end{questyle}

\begin{questyle}
  \question  zzz  (GATE-zzz)

  \begin{choices}
    \choice         zzz
    \choice         zzz
    \choice         zzz
    \choice         zzz
    \CorrectChoice
  \end{choices}
\end{questyle}

\begin{questyle}
  \question  zzz  (GATE-zzz)

  \begin{choices}
    \choice         zzz
    \choice         zzz
    \choice         zzz
    \choice         zzz
    \CorrectChoice
  \end{choices}
\end{questyle}


  \begin{choices}
    \choice         zzz
    \choice         zzz
    \choice         zzz
    \choice         zzz
    \CorrectChoice
  \end{choices}
\end{questyle}

\begin{questyle}
  \question  zzz  (GATE-zzz)

  \begin{choices}
    \choice         zzz
    \choice         zzz
    \choice         zzz
    \choice         zzz
    \CorrectChoice
  \end{choices}
\end{questyle}

\begin{questyle}
  \question  zzz  (GATE-zzz)

  \begin{choices}
    \choice         zzz
    \choice         zzz
    \choice         zzz
    \choice         zzz
    \CorrectChoice
  \end{choices}
\end{questyle}


  \begin{choices}
    \choice         zzz
    \choice         zzz
    \choice         zzz
    \choice         zzz
    \CorrectChoice
  \end{choices}
\end{questyle}

\begin{questyle}
  \question  zzz  (GATE-zzz)

  \begin{choices}
    \choice         zzz
    \choice         zzz
    \choice         zzz
    \choice         zzz
    \CorrectChoice
  \end{choices}
\end{questyle}

\begin{questyle}
  \question  zzz  (GATE-zzz)

  \begin{choices}
    \choice         zzz
    \choice         zzz
    \choice         zzz
    \choice         zzz
    \CorrectChoice
  \end{choices}
\end{questyle}


  \begin{choices}
    \choice         zzz
    \choice         zzz
    \choice         zzz
    \choice         zzz
    \CorrectChoice
  \end{choices}
\end{questyle}

\begin{questyle}
  \question  zzz  (GATE-zzz)

  \begin{choices}
    \choice         zzz
    \choice         zzz
    \choice         zzz
    \choice         zzz
    \CorrectChoice
  \end{choices}
\end{questyle}

\begin{questyle}
  \question  zzz  (GATE-zzz)

  \begin{choices}
    \choice         zzz
    \choice         zzz
    \choice         zzz
    \choice         zzz
    \CorrectChoice
  \end{choices}
\end{questyle}


\end{comment}


