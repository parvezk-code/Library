
\centerline{\textbf{ \LARGE Page Replacement Algoritms}}


% ----------------------------------------------------------------------------

\begin{questyle}

  \question  Which page replacement policy sometimes leads to more page faults when size of
             memory is increased?(GATE-1992)

  \begin{oneparchoices}
    \choice Optimal
    \choice LRU
    \choice FIFO
    \choice None of these.
  \end{oneparchoices}
\end{questyle}

% ----------------------------------------------------------------------------

\begin{questyle}

  \question  Consider a virtual memory system with FIFO page replacement policy. For an arbitrary page access pattern, increasing the number of page frames in main memory will (GATE-2001)

  \begin{choices}
    \choice always decrease the number of page faults
    \choice always increase the number of page faults
    \choice sometimes increase the number of page faults
    \choice never affect the number of page faults
  \end{choices}

\end{questyle}

% ----------------------------------------------------------------------------

\begin{questyle}

  \question  The optimal page replacement algorithm will select the page that (GATE-2002)

  \begin{choices}
    \choice Has not been used for the longest time in the past
    \choice Will not be used for the longest time in the future
    \choice Has been used least number of times
    \choice Has been used most number of times
  \end{choices}

\end{questyle}

% ----------------------------------------------------------------------------

\begin{questyle}

  \question  In which one of the following page replacement policies, Belady’s anomaly may occur? (GATE-2009)

  \begin{oneparchoices}
    \choice FIFO
    \choice Optimal
    \choice LRU
    \choice MRU
  \end{oneparchoices}

\end{questyle}

% ----------------------------------------------------------------------------

\begin{questyle}

  \question  A system uses FIFO policy for page replacement. It has 4 page frames with no pages loaded to begin with.
              The system first accesses 100 distinct pages in some order and then accesses the same 100 pages but now in the
              reverse order. How many page faults will occur?(GATE-2010)

  \begin{oneparchoices}
    \choice 196
    \choice 192
    \choice 197
    \choice 195
  \end{oneparchoices}

\end{questyle}

% ----------------------------------------------------------------------------

\begin{questyle}

  \question  A system uses 3 page frames for storing process pages in main memory. It uses the Least
            Recently Used (LRU) page replacement policy. Assume that all the page frames are initially empty.
            What is the total number of page faults that will occur while processing the page reference string given below?  (GATE-2014\_set\_3) \\
            4, 7, 6, 1, 7, 6, 1, 2, 7, 2

  \begin{oneparchoices}
    \choice 4
    \choice 5
    \choice 6
    \choice 7
  \end{oneparchoices}

\end{questyle}

% ----------------------------------------------------------------------------


\begin{questyle}

  \question  A memory page containing a heavily used variable that was initialized very early and is in constant use is removed then (GATE-1994)

  \begin{choices}
    \choice LRU page replacement algorithm is used
    \choice FIFO page replacement algorithm is used
    \choice LFU page replacement algorithm is used
    \choice None of the above
  \end{choices}

\end{questyle}

% ----------------------------------------------------------------------------

\begin{questyle}

  \question The address sequence generated by tracing a particular program executing in a pure demand based
            paging system with 100 records per page with 1 free main memory frame is recorded as follows. What is the
            number of page faults? (GATE-1995) \\
            0100, 0200, 0430, 0499, 0510, 0530, 0560, 0120, 0220, 0240, 0260, 0320, 0370

  \begin{oneparchoices}
    \choice 13
    \choice 8
    \choice 7
    \choice 10
  \end{oneparchoices}

\end{questyle}

% ----------------------------------------------------------------------------

\begin{questyle}

  \question  A virtual memory system uses First In First Out (FIFO) page replacement policy and allocates a
            fixed number of frames to a process. Which of the following statements is TRUE? (GATE-2007) \\
  P: Increasing the number of page frames allocated to a process sometimes increases the page fault rate.\\
  Q: Some programs do not exhibit locality of reference.

  \begin{choices}
    \choice Both P and Q are true, and Q is the reason for P
    \choice Both P and Q are true, but Q is not the reason for P.
    \choice P is false but Q is true
    \choice Both P and Q are false.
  \end{choices}

\end{questyle}

% ----------------------------------------------------------------------------

\begin{questyle}

  \question  A process has been allocated 3 page frames. Assume that none of the pages of the process are
            available in the memory initially. The process makes the following sequence of
            page references (reference string): 1, 2, 1, 3, 7, 4, 5, 6, 3, 1 (GATE-2007)

  \begin{parts}
      \part If optimal page replacement policy is used, how many page faults occur for the above reference string?
        \begin{choices}
          \choice 7
          \choice 8
          \choice 9
          \choice 10
        \end{choices}

      \part Least Recently Used (LRU) page replacement policy is a practical approximation to optimal
            page replacement. For the above reference string, how many more page faults occur with LRU
            than with the optimal page replacement policy?
        \begin{choices}
          \choice 0
          \choice 1
          \choice 2
          \choice 3
        \end{choices}

  \end{parts}

\end{questyle}

% ----------------------------------------------------------------------------

\begin{questyle}

  \question  Consider the virtual page reference string \\ 1, 2, 3, 2, 4, 1, 3, 2, 4, 1 \\
             On a demand paged virtual memory system running on a computer system that main memory size of 3 pages
             frames which are initially empty. Let LRU, FIFO and OPTIMAL denote the number of page faults under
             the corresponding page replacements policy. Then (GATE-2012)

  \begin{choices}
    \choice OPTIMAL \(<\) LRU \(<\) FIFO
    \choice OPTIMAL \(<\) FIFO \(<\) LRU
    \choice OPTIMAL = LRU
    \choice OPTIMAL = FIFO
  \end{choices}

\end{questyle}


% ----------------------------------------------------------------------------

\begin{questyle}

  \question Assume that there are 3 page frames which are initially empty. If the page reference
            string is 1, 2, 3, 4, 2, 1, 5, 3, 2, 4, 6, the number of page faults using the optimal
            replacement policy is (GATE-2014\_set\_1)

\end{questyle}


% ----------------------------------------------------------------------------

\begin{questyle}

  \question  A computer has twenty physical page frames which contain pages numbered 101 through 120.
            Now a program accesses the pages numbered 1, 2, …, 100 in that order, and repeats the access sequence
            THRICE. Which one of the following page replacement policies experiences the same number of
            page faults as the optimal page replacement policy for this program? (GATE-2014\_Set\_2)

  \begin{choices}
    \choice Least-recently-used
    \choice First-in-first-out
    \choice Last-in-first-out
    \choice Most-recently-used
  \end{choices}

\end{questyle}

% ----------------------------------------------------------------------------

\begin{questyle}

  \question  Consider a main memory with five page frames and the following sequence of page
             references: 3, 8, 2, 3, 9, 1, 6, 3, 8, 9, 3, 6, 2, 1, 3. Which one of the following is true with respect
             to page replacement policies First-In-First Out (FIFO) and Least Recently Used (LRU)? (GATE-2015\_Set\_1)

  \begin{choices}
    \choice Both incur the same number of page faults
    \choice FIFO incurs 2 more page faults than LRU
    \choice LRU incurs 2 more page faults than FIFO
    \choice FIFO incurs 1 more page faults than LRU
  \end{choices}

\end{questyle}

% ----------------------------------------------------------------------------

\begin{questyle}

  \question  Consider a computer system with ten physical page frames. The system is provided with an access sequence a1, a2, …, a20, a1, a2, …, a20), where each ai number. The difference in the number of page faults between the last-in-first-out page replacement policy and the optimal page replacement policy is \fillin[] (GATE-2016\_set\_1)

\end{questyle}

% ----------------------------------------------------------------------------

\begin{questyle}

  \question  Recall that Belady’s anomaly is that the pages-fault rate may increase as the number of allocated
            frames increases. Which of the following is CORRECT? (GATE-2017\_set\_1) \\
            S1: Random page replacement algorithm (where a page chosen at random is replaced) suffers from Belady’s anomaly.\\
            S2: LRU page replacement algorithm suffers from Belady’s anomaly . \\
  \begin{choices}
    \choice S1 is true, S2 is true
    \choice S1 is true, S2 is false
    \choice S1 is false , S2 is true
    \choice S1 is false, S2 is false
  \end{choices}

\end{questyle}

% ----------------------------------------------------------------------------

% ----------------------------------------------------------------------------

% ----------------------------------------------------------------------------

% ----------------------------------------------------------------------------

% ----------------------------------------------------------------------------

% ----------------------------------------------------------------------------

% ----------------------------------------------------------------------------

% ----------------------------------------------------------------------------

% ----------------------------------------------------------------------------

% ----------------------------------------------------------------------------

% ----------------------------------------------------------------------------

