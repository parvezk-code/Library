
\centerline{\textbf{ \LARGE Interupts, System calls, CPU Modes}}


% ----------------------------------------------------------------------------

\begin{questyle}

  \question System calls are invoked using. (GATE - 99)

  \begin{oneparchoices}
    \CorrectChoice A software interupt
    \choice Pooling.
    \choice An indirect jump.
    \choice A privileged instruction.
  \end{oneparchoices}

    \end{questyle}




% ----------------------------------------------------------------------------

\begin{questyle}

  \question Processor need software interupt to. (GATE - 2001 )

  \begin{choices}
    \choice Test interupt system
    \choice Implement co-routine
    \CorrectChoice obtain system services which need execution of privileged instructions
    \choice Return from subroutine
  \end{choices}

  \end{questyle}



% ----------------------------------------------------------------------------

\begin{questyle}

  \question To change mode from privileged to non-privileged. (GATE - 2001 )

  \begin{choices}
    \choice A hardware interupt is needed.
    \choice A software interupt is needed.
    \choice A privileged instruction (which does not generate an interrupt) is needed.
    \CorrectChoice A non-privileged instruction (which does not generate an interrupt is needed).
  \end{choices}

  \end{questyle}


\begin{questyle}
  \question  Which of the following must be true for the RFE (Return from Exception) instruction
             on a general purpose processor?  (GATE-2008)

             \begin{enumerate}
                  \item It must be a trap instruction
                  \item It must be a privileged instruction
                  \item An exception cannot be allowed to occur during execution of an RFE instruction
              \end{enumerate}

  \begin{oneparchoices}
    \choice         I only
    \choice         II only
    \choice         I and II only
    \CorrectChoice  I, II and III only
  \end{oneparchoices}
\end{questyle}



% ----------------------------------------------------------------------------

\begin{questyle}

  \question CPU handles interupt by checking interupt register (GATE - 2009 )

  \begin{choices}
    \choice Immediately after interupt is raised.
    \choice After fetch cycle
    \CorrectChoice After execution of each instruction.
    \choice After fixed interval of time.
  \end{choices}

  \end{questyle}



% ----------------------------------------------------------------------------

\begin{questyle}

  \question Which interupt is handled at highest priority (GATE - 2011 )

  \begin{choices}
    \choice Interupt after completion of hard disk read operation.
    \choice Interupt after mouse click.
    \choice Interupt after keyboard button press.
    \CorrectChoice Interupt after CPU temperature too high.
  \end{choices}

  \end{questyle}




% ----------------------------------------------------------------------------

\begin{questyle}

  \question The time taken to switch between user and kernel modes of execution be t1 while the time taken to
            switch between two processes be t2. Then (GATE - 2011)

  \begin{oneparchoices}
    \choice t1=t2
    \choice t1 \textgreater  t2
    \CorrectChoice t1 \textless  t2
    \choice nothing can be said.
  \end{oneparchoices}


  \end{questyle}



% ----------------------------------------------------------------------------

\begin{questyle}

  \question Context switch takes 10 micro-sec. Interupt service execution takes  80 micro-sec.
          Given that an interrupt input arrives every 1 msec, what is the percentage  of the  total
          time that the CPU devotes for the main program execution \fillin[90\%]. (GATE - 1993)

  \end{questyle}



% ----------------------------------------------------------------------------

\begin{questyle}

  \question  Which of the following do not interupt a running process. (GATE - 2001)

  \begin{oneparchoices}
    \choice A device
    \choice Timer
    \CorrectChoice Schedular process
    \choice Power failure
  \end{oneparchoices}

  \end{questyle}



% ----------------------------------------------------------------------------
% ----------------------------------------------------------------------------
% ----------------------------------------------------------------------------
% ----------------------------------------------------------------------------
% ----------------------------------------------------------------------------
% ----------------------------------------------------------------------------
% ----------------------------------------------------------------------------
% ----------------------------------------------------------------------------
% ----------------------------------------------------------------------------
