
\centerline{\textbf{ \LARGE Synchronization}}


% ----------------------------------------------------------------------------


\begin{enumerate}

  \item Critical Section(CS) is part of a program where a process manipulates shared variables.
  \item One Solution is to disable context switching in CS. Interupts will be un-handled in CS.
  \item Correct Solution
  \begin{enumerate}
    \item Mutual exclusion : no 2 processes can be in critical sections simultaneously.
    \item Progress : only competing processes(not in CS or Remainder) are considered for decision as to who goes next.
          If CS is free then process in remainder section should not be making decision or blocking those who want to enter CS.
    \item Bounded wait(no starvation) : there is bound on number of times a process is bypassed by another process after
          it has indicated its desire to enter the critical section
    \item If there is deadlock then there is no progress.
  \end{enumerate}
  \item Race Around Condition :

  \item Turn Variable:
  \begin{enumerate}
    \item Only for 2 process.
    \item strict alternation.
    \ item shared variable - turn
    \item while( not\_my\_turn ) do-nothing;
  \end{enumerate}
      \begin{myTableStyle}
        \begin{tabular}{ |m{5cm}|m{5cm}| } \hline
            \lstinputlisting[language=Octave]{./code/turnVariable_01.c}   &     \lstinputlisting[language=Octave]{./code/turnVariable_02.c}   \\ \hline
        \end{tabular}
      \end{myTableStyle}
      \vspace{0.08in}

  \item Peterson's Solution:
  \begin{enumerate}
    \item Only for 2 process.
    \item No alternation.
    \item shared variable - turn, flag
    \item while( not\_my\_turn and other\_wants\_to\_go ) do-nothing;
  \end{enumerate}
      \begin{myTableStyle}
        \begin{tabular}{ |m{6cm}|m{6cm}| } \hline
            \lstinputlisting[language=Octave]{./code/peterson_01.c}   &   \lstinputlisting[language=Octave]{./code/peterson_02.c}   \\ \hline
        \end{tabular}
      \end{myTableStyle}
      \vspace{0.08in}

      \item Reentrant (Recursive) locks: Reentrant (Recursive) locks allow a thread to reacquire the lock.
          That means same process can claim the lock multiple times without blocking on itself. This prevents the
          thread from deadlocking itself. This is main advantage of reentrant locks over non-reentrant locks.
      \item Non-reentrant (Non- recursive) locks : not allow a thread to reacquire the lock. That means same
          process can not claim the lock multiple times without releasing it. So, if a thread/process is unable
          to acquire a lock, it blocks until the lock becomes available. This situation is deadlocked.
\end{enumerate}

% ----------------------------------------------------------------------------

% ----------------------------------------------------------------------------

% ----------------------------------------------------------------------------

% ----------------------------------------------------------------------------

% ----------------------------------------------------------------------------

% ----------------------------------------------------------------------------

% ----------------------------------------------------------------------------

% ----------------------------------------------------------------------------
