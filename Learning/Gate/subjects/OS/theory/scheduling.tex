
\centerline{\textbf{ \LARGE Scheduling Algorithms}}


% ----------------------------------------------------------------------------


\begin{enumerate}

  \item Terminologies for Scheduling Algorithms

  \begin{myTableStyle}
    \begin{tabular}{ |l|l| } \hline
        Burst                 &     CPU time needed by the process.                             \\ \hline
        Arrival Time(AT)      &     Time at which the process arrives.                          \\ \hline
        Completion Time(CT)   &     Time at which the process gets completed                    \\ \hline
        Through-put(speed)    &     No of process completed per unit time.                      \\ \hline
        Turn Around time(TAT) &     CT - AT   =  Burst + Wating Time                            \\ \hline
        Response time         &     Time at which process is scheduled for the first time - AT  \\ \hline
        Wating Time           &     TAT -  Burst                                                \\ \hline
        Response Ratio        &     Burst/WT                                                    \\ \hline
        Pre-emptive           &     CPU can be take from a process forcefully                   \\ \hline
    \end{tabular}
  \end{myTableStyle}

  \vspace{0.08in}

  \item Scheduling

  \begin{myTableStyle}
    \begin{tabular}{ |l|l| } \hline
        FCFS                &     Convoy effect(large burst effect)                             \\ \hline
        SJF                 &     Chooses job with shortest bust.                               \\ \hline
        SRTF                &     Pre-emptive SJF, starvation, minimum Avg TAT, WT, RT.         \\ \hline
        Round Robin         &     Requires h/w support for timer, Increase avg TRT and WT       \\ \hline
        Priority            &     Starvation(live lock), Aging: inc priority with time         \\ \hline
        Multi-level queue   &     Starvation                                                    \\ \hline
        Gang scheduling     &     For threads                                                    \\ \hline
        Rate monotonic      &     Real time scheduling                                          \\ \hline
        Highest-response ratio next & Chose job with least response ratio                       \\ \hline
        Fair share Scheduling & Guranteed Scheduling                                            \\ \hline
    \end{tabular}
  \end{myTableStyle}

  \item Types of schedulers \\
  \begin{myTableStyle}
    \begin{tabular}{ |l|l|m{2.8cm}|m{4cm}| } \hline
        Name      & Long Term Schedular                   & Short-Term Schedular  & Mid Term Schedular (swapping)\\ \hline
        States    & new to ready                          & ready to running      & running to wating/blocked   \\ \hline
        Frequency & runs less frequent                    & more frequent         & in between both             \\ \hline
        M-prog    & controlls degree of multi programming & effects less          & reduces degree of m-prog    \\ \hline
        Storage   & seconday to main memory(mm)           & mm to mm              & main memory to Secondary    \\ \hline
                  & use dispatcher for context switching  & &                                                   \\ \hline
    \end{tabular}
  \end{myTableStyle}



  \begin{minipage}{\linewidth}
  \item Process statistics
  \begin{enumerate}
    \item Desired Completion Time(DCT) = Arrival time + Burst
    \item Actual Completion Time(ACT) : Time at which process is completed.
    \item Waiting Time(WT) = ACT - DCT
    \item Turnaround time(TRT) = ACT - Burst

    \begin{myTableStyle}
    \begin{center}
    \begin{tabular}{ |c|c|c|M{1.8cm}|c|c|c| } \hline
          ID & Arrival &  Burst & Desired CT (Arr+burst) & Actual CT & WT & TAT(ACT -Arrival)  \\ \hline
          P1 & 0 & 12 & 12 & t1 & t1 - 12 & t1 - 0      \\ \hline
          P2 & 2 & 4  & 6  & t2 & t2 - 6  & t2 - 2     \\ \hline
          P3 & 3 & 6  & 9  & t3 & t3 - 9  & t3 - 3     \\ \hline
          P4 & 8 & 5  & 13 & t4 & t4 - 13 & t4 - 8      \\ \hline
    \end{tabular}
    \end{center}
  \end{myTableStyle}
  \vspace{0.08in}
  \end{enumerate}
  \end{minipage}


  \item Gang charts of following questions.
        \begin{enumerate}
          \item SRTF - An operating system uses shortest remaining time first scheduling algorithm for pre-emptive
            scheduling of processes. Consider the following set of processes with their arrival times and
            CPU burst times (in milliseconds). The average waiting time (in milliseconds) of the
            processes is \fillin[5.5]. (GATE-2014\_set\_3 )

    \begin{myTableStyle}
    \begin{center}
    \begin{tabular}{ |c|c|c|c| } \hline
          ID & Arrival &  Burst     \\ \hline
          P1 & 0 & 12     \\ \hline
          P2 & 2 & 4     \\ \hline
          P3 & 3 & 6     \\ \hline
          P4 & 8 & 5      \\ \hline
    \end{tabular}
    \end{center}
  \end{myTableStyle}
  \vspace{0.08in}

          \item RR - Three processes A, B and C each execute a loop of 2 iterations. Round robin scheduling is used with
                time slice of 50 milliseconds. All IO can go in parallel. The time in milliseconds at which process C would complete
                its first I/O operation is. Ans- 1000ms.

  \begin{myTableStyle}
    \begin{center}
    \begin{tabular}{ |c|c|c|c| } \hline
          ID & Time\_CPU & Time\_IO  & Arrival \\ \hline
          A & 100(2t) & 500(10t) & 0 \\ \hline
          B & 350(7t) & 500(10t) & 5 \\ \hline
          C & 200(4t) & 500(10t) & 10 \\ \hline
    \end{tabular}
    \end{center}
  \end{myTableStyle}
  \vspace{0.08in}

        \end{enumerate}

\end{enumerate}

% ----------------------------------------------------------------------------

% ----------------------------------------------------------------------------

% ----------------------------------------------------------------------------

% ----------------------------------------------------------------------------

% ----------------------------------------------------------------------------

% ----------------------------------------------------------------------------

% ----------------------------------------------------------------------------

% ----------------------------------------------------------------------------
