\centerline{\textbf{ \LARGE Deadlock}}

% ----------------------------------------------------------------------------


\begin{enumerate}

  \item Dead-lock conditions (http://www.cs.yale.edu/homes/aspnes/pinewiki/Deadlock.html)\\
     \begin{myTableStyle}
      \begin{tabular}{ |m{3cm}|m{12cm}| } \hline
          Mutual Exclusion    &      The resources involved must be in non-shareable mode    \\ \hline
          No preemption       &      resource can not be taken forcefully    \\ \hline
          Hold and wait       &         \\ \hline
          Circular wait       &      p1 \textrightarrow  p2 \textrightarrow p3 .... \textrightarrow  pn \textrightarrow  p1 \\ \hline
      \end{tabular}
    \end{myTableStyle}
  \vspace{0.08in}

  \item Preventing deadlock : violate one of the 4 conditions.\\
      \begin{myTableStyle}
      \begin{tabular}{ |m{3cm}|m{12cm}| } \hline
          No mutual exclusion    &      most writable data structures, CPU, CD burners, keyboard can't be shared    \\ \hline
          No hold-and-wait       &      give all the needed resources to the process. Need resource prediction.
                                        Brings starvation, low device utilization    \\ \hline
          Allow preemption       &         \\ \hline
          No circular-wait       &      allow request only for  resources of higher ID. Need resource prediction \\ \hline
      \end{tabular}
    \end{myTableStyle}
  \vspace{0.08in}

  \item Preventing deadlock techniques :
  \begin{enumerate}
    \item The resources are numbered uniquely, and processes are allowed to request for resources only in increasing order.
    \item The resources are numbered uniquely, and processes are allowed to request for resources only in decreasing order.
    \item In deadlock avoidance, the request for resources is always granted if the result state is safe.
    \item Deadlock avoidance is less restrictive than deadlock prevention.
    \item Deadlock avoidance requires knowledge of resource requirements a priori.
  \end{enumerate}

  \item A system has n processes sharing m identical resources. Maximum resource requirment of process(\(P_i\)) is  \(S_i\) .
        Condition for no deadlock is. \\
        {\large \(   \sum_{n=1}^{n} s_i < (m+n)   \) }

  \item Resource allocation graph(RAG) : If resources have single instance then a cycle in RAG is necessary
        and sufficient for deadlock.

\end{enumerate}

% ----------------------------------------------------------------------------

% ----------------------------------------------------------------------------

% ----------------------------------------------------------------------------

% ----------------------------------------------------------------------------

% ----------------------------------------------------------------------------

% ----------------------------------------------------------------------------

% ----------------------------------------------------------------------------

% ----------------------------------------------------------------------------
