
\centerline{\textbf{ \LARGE Semaphore}}


% ----------------------------------------------------------------------------

\begin{enumerate}

  \item Binary Semaphore : work for many process. \\
      \begin{myTableStyle}
        \begin{tabular}{ |m{4.5cm}|m{5cm}|m{3.5cm}| } \hline
            \lstinputlisting[language=Octave]{./code/semaphore_01_wait.c}   &
            \lstinputlisting[language=Octave]{./code/semaphore_01_signal.c} &
            \lstinputlisting[language=Octave]{./code/semaphore_01.c} \\ \hline
        \end{tabular}
      \end{myTableStyle}
  \begin{enumerate}
    \item Value of S can be 0 or 1. At s=1 if multiple V operations are performend then value os S will remain 1, since it is binary.
    \item if( s == 1 ) then no   process is in CS
    \item if( s == 0 ) then one process is in CS
    \item Drawback :
    \begin{enumerate}
      \item Busy waiting, wasting cpu cycles because of the while loop.
      \item Priority inversion.
  \end{enumerate}
  \item Counting semaphore
  \begin{enumerate}
    \item Can take values other than 0, 1.
    \item if( s=5 ) then 5 processes can enter CS.
  \end{enumerate}
  \end{enumerate}

  \item Reader Writer Problem
  \begin{enumerate}
    \item Multiple reads can be performed parallelly.
    \item Two semaphore are needed.
      \begin{enumerate}
        \item wrt : for resource access
        \item mutex: for access to readCount variable.
      \end{enumerate}
      \item First reader will wait for the writer to exit. Last reader will signal the writer. \\
      \begin{myTableStyle}
        \begin{tabular}{ |m{4cm}|m{4cm}|m{6cm}| } \hline
            \lstinputlisting[language=Octave]{./code/reader_writer_01.c}   &
            \lstinputlisting[language=Octave]{./code/reader_writer_03.c}   &
            \lstinputlisting[language=Octave]{./code/reader_writer_02.c} \\ \hline
        \end{tabular}
      \end{myTableStyle}
  \end{enumerate}

 \begin{minipage}{\linewidth}
  \item Bounded Buffer : Producer and Consumer
    \begin{enumerate}
      \item order of P() operation is very important. Otherwise dead-lock can happen.
    \end{enumerate}
      \begin{myTableStyle}
        \begin{tabular}{ |m{4.5cm}|m{5cm}| } \hline
            \lstinputlisting[language=Octave]{./code/bounded_buffer_01.c}   &
            \lstinputlisting[language=Octave]{./code/bounded_buffer_02.c}  \\ \hline
        \end{tabular}
      \end{myTableStyle}
      \vspace{0.08in}
  \end{minipage}
  \vspace{0.08in}

 \begin{minipage}{\linewidth}
  \item Hardware approach for synchronization
    \begin{enumerate}
      \item Test\_And\_Set :
      \item Swap :
      \item Lock\_And\_Unlock :
    \end{enumerate}
      \begin{myTableStyle}
        \begin{tabular}{ |m{6cm}|m{5cm}| } \hline
            \lstinputlisting[language=Octave]{./code/test_and_set_01.c}   &
            \lstinputlisting[language=Octave]{./code/test_and_set_02.c}  \\ \hline
            \lstinputlisting[language=Octave]{./code/swap_01.c}   &
            \lstinputlisting[language=Octave]{./code/swap_02.c}  \\ \hline
        \end{tabular}
      \end{myTableStyle}
      % \vspace{0.08in}
  \end{minipage}

  \item Important observations in synchronization
    \begin{enumerate}
      \item Solution should be free from deadlock.
      \item For binary semaphore, make sure that no V operation is performed if its value is 1.
      \item For binary semaphore, two V operations with no P operation between, should not occur on same semaphore variable.
    \end{enumerate}

  \begin{minipage}{\linewidth}
  \item Alternation (C1: producer  C2: consumer)
    \begin{enumerate}
      \item Produce and give turn to waiting consumer to consume.
      \item Consume and give turn to producer to produce.
      \item Case 3: Consumer gives turn to producer and waits for its signal. Producer after producing one item waits for
                    consumer signal. Producer signals consumer to go for consumption. Producer goes to produce one item.
    \end{enumerate}
    \lstinputlisting[language=Octave]{./code/semaphore_02_alternation.c}
  \end{minipage}

\end{enumerate}

% ----------------------------------------------------------------------------

% ----------------------------------------------------------------------------

% ----------------------------------------------------------------------------

% ----------------------------------------------------------------------------

% ----------------------------------------------------------------------------

% ----------------------------------------------------------------------------

% ----------------------------------------------------------------------------

% ----------------------------------------------------------------------------
