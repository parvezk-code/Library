\usepackage{tikz}
\usetikzlibrary{positioning, tikzmark, calc, shapes.geometric, arrows}  % shapes,backgrounds,fit

\newenvironment{myTree}
{
    \begingroup
    \begin{center}
    \begin{tikzpicture}[level distance=1.5cm, level 1/.style={sibling distance=3cm}, level 2/.style={sibling distance=1.5cm}]
}
{
    \end{tikzpicture}
    \end{center}
    \endgroup
}


\newenvironment{myTreeLThree}
{
    \begingroup
    \begin{center}
    \begin{tikzpicture}[level distance=1.5cm, level 1/.style={sibling distance=6cm}, level 2/.style={sibling distance=3cm}, level 3/.style={sibling distance=1.5cm}]
}
{
    \end{tikzpicture}
    \end{center}
    \endgroup
}

\newenvironment{vArrow}[3][black]
{
    \draw[->, #1] (#2) -- (#3);
}
{

}

\newenvironment{tArcDown}[2][black]
{
    \draw[-latex,#1] ($#2$) arc [ start angle=-160, end angle=-20, x radius=0.9cm, y radius =0.7cm ] ;
}
{

}


\newenvironment{tArcUp}[2][red]
{
    \draw[-latex,#1] ($#2$) arc [ start angle=160, end angle=20, x radius=0.9cm, y radius =0.7cm ] ;
}
{

}

\tikzstyle{flowchart-arrow} = [thick,->,>=stealth]

\tikzstyle{roundnode} = [rectangle, draw=green!60, fill=green!5, very thick, minimum size=7mm]

\tikzstyle{squarednode} = [rectangle, draw=red!60, fill=red!5, very thick, minimum size=7mm]

\tikzstyle{flowchart-start} = [rectangle, rounded corners, minimum width=3cm, minimum height=1cm, text centered, draw=black, fill=red!30]

\tikzstyle{flowchart-io} = [trapezium, trapezium stretches=true, trapezium left angle=70, trapezium right angle=110, minimum width=3cm, minimum height=1cm, text centered, draw=black, fill=blue!30]

\tikzstyle{flowchart-process} = [rectangle, minimum width=2cm, minimum height=1cm, text centered, text width=4cm, draw=black, fill=orange!30]

\tikzstyle{flowchart-decision} = [diamond, minimum width=3cm, minimum height=1cm, text centered, draw=black, fill=green!30]


\begin{comment}

Arrows                      :   https://tikz.dev/tikz-arrows
Node alignment              :   https://tikz.net/contents/chapter-03-drawing-positioning-and-aligning-nodes/
Array/Matrix                :   https://tikz.dev/tikz-matrices
draw in table               :   baseline=0 to fit picture in table cell.
package for vectorgraphics  :   PGF, PGFPlots


positioning         :   https://rmwu.github.io/tutorial/latex/2019/11/21/positioning/

\begin{tikzpicture}[ baseline=0, node distance=1.5cm]

path                :   --  -|  |-
cordinate           :   ++(x, y)    ++(2,0)     ++(-2, 0)
Node                :   \node (lable) at (y,x)  [ params ] {text}
Node params         :   xshift=-1cm, yshift=0.2cm, below left=2cm, of=node, draw=blue!60
draw(line,arrow)    :   (node/cordinate)    path    (node/cordinate)

\draw [flowchart-arrow]  node1 path1 (node/cordinate) path2 (node/cordinate) path3 (node/cordinate)


arrow from bottom to up taking left turn : draw [flowchart-arrow] (bottomNode.west) -- +(-2,0) |- (topNode.west);
\draw (0,0) -- ++(1,0) -- ++(0,1) -- ++(-1,0) -- cycle;

\node (N1) [flowchart-start] {NodeText};
\node (N2) [flowchart-process, below of=N1] { NodeText };
\node (N2) [flowchart-process, below left=2cm of=N1] { NodeText };
intermediate node   :   node [right of= S3, xshift=-1cm, yshift=0.2cm] { NodeText }

\end{comment}
