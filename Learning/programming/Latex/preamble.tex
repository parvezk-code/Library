
\usepackage{amssymb}
\usepackage{geometry}
\usepackage{comment}
\usepackage{listings}
\usepackage{import}
\usepackage{graphicx}
\usepackage{array}  % for table and column width
\usepackage{tabularray} % for table and column width

\geometry{
            a4paper,
            total={170mm,257mm},
            left=17mm,
            top=15mm,
 }

\linespread{1.3}

\renewcommand{\rmdefault}{phv}
\renewcommand{\sfdefault}{phv}

\newenvironment{questyle}
    {
    \begin{minipage}{\linewidth}
    }
    {

    \end{minipage}
    \vspace{0.08in}
    }


\newcolumntype{M}[1]{>{\centering\arraybackslash}m{#1}}     % custom column : fixed width and centering
\newcolumntype{P}[1]{>{\centering\arraybackslash}p{#1}}

% The \begingroup ... \endgroup pair ensures the separation
% parameters only affect this particular table, and not any
% sebsequent ones in the document.
\newenvironment{myTableStyle}
    {
    \begingroup
    \setlength{\tabcolsep}{10pt} % Default value: 6pt
    \renewcommand{\arraystretch}{1.5} % Default value: 1
    }
    {

    \endgroup
    }
