\documentclass[../../main.tex]{subfiles}
\begin{document}

\chapter{Tables in \LaTeX}
% -------------------
\section{Simple Table}
\begin{myTableStyle}
% ==============================
% Environment for Simple Table
% ==============================
\newenvironment{simpleTable}{
  \begingroup
  \begin{center}
  \renewcommand{\arraystretch}{1.2}
}{
  \end{center}
  \endgroup
}

% ==============================
% Simple Table
% ==============================
\begin{simpleTable}
\begin{tabular}{|l|c|r|}
\hline
Item & Quantity & Price (\$) \\
\hline
Apple & 10 & 5.50 \\
Banana & 7 & 3.25 \\
Cherry & 3 & 2.10 \\
\hline
\end{tabular}
\end{simpleTable}
}
\end{myTableStyle}

\noindent\textbf{Code:}
\lstinputlisting[
    language=TeX,
    caption={Simple Table Example},
    linerange={16-26}
]{\subfix{example_codes/table_01_simple_example.tex}}

\clearpage

\lstinputlisting[
    language=TeX,
    caption={Simple Table Example},
    linerange={4-11}
]{\subfix{example_codes/table_01_simple_example.tex}}

% -------------------
\section{Colored Table}
\begin{myTableStyle}
\begin{table}[H]
    \centering
    \caption{Colored Table Example}
    \arrayrulecolor{white}
    \rowcolors{2}{blue!10}{gray!10}
    \begin{tabular}{p{3cm}p{3cm}p{3cm}}
    \rowcolor{blue!30}
    \textbf{Name} & \textbf{Department} & \textbf{Score} \\
    John & Sales & 92 \\
    Maria & Marketing & 89 \\
    Lee & R\&D & 95 \\
    Nina & HR & 88 \\
    \end{tabular}
    \end{table}
    }
\end{myTableStyle}

\noindent\textbf{Code:}
\lstinputlisting[
      language=TeX, 
      caption={Colored Table Example},
      linerange={4-13}
]{\subfix{example_codes/table_02_color_table.tex}}

\clearpage

\lstinputlisting[
      language=TeX, 
      caption={Colored Table Example},
      linerange={18-29}
]{\subfix{example_codes/table_02_color_table.tex}}

% -------------------
\section{Fancy Table with Borders and Multirow}
\begin{myTableStyle}
\begin{table}[H]
    \centering
    \caption{Fancy Table Example with Multirow and Borders}
    \renewcommand{\arraystretch}{1.3}
    \begin{tabular}{|l|c|r|}
    \hline
    \rowcolor{gray!20}
    \textbf{Name} & \textbf{Department} & \textbf{Score} \\
    \hline
    \multirow{2}{*}{Alice} & Sales & 92 \\
     & Support & 87 \\
    \hline
    Bob & Marketing & 89 \\
    \hline
    Charlie & R\&D & 95 \\
    \hline
    \end{tabular}
    \end{table}
    }
\end{myTableStyle}

\noindent\textbf{Code:}
\lstinputlisting[
      language=TeX, 
      caption={Fancy Table with Borders Example},
      linerange={4-11}
]{\subfix{example_codes/table_03_fancy_table_example.tex}}

\clearpage

\lstinputlisting[
      language=TeX, 
      caption={Fancy Table with Borders Example},
      linerange={16-27}
]{\subfix{example_codes/table_03_fancy_table_example.tex}}

\end{document}
