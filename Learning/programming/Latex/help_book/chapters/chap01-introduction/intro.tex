\documentclass[../../main.tex]{subfiles}

\begin{document}

Here is a list of tips to make your LaTeX projects more modular, readable, and maintainable:

\begin{itemize}
    \item A latex page often have - Lists, Tables, Images. Create seprate tex files for them and use input command to include then in the page.

    \item \textbf{Use \texttt{subfiles}} to divide the content into separate files, allowing independent compilation of chapters or sections. \textbf{Use \texttt{\textbackslash input} or \texttt{\textbackslash include}} to separate sections or chapters into different files for better structure.
        
    \item \textbf{Define reusable variables and commands} with \texttt{\textbackslash newcommand} or \texttt{\textbackslash def} to keep your code cleaner and avoid repetition.
    
    \item \textbf{Use \texttt{\textbackslash begingroup} and \texttt{\textbackslash endgroup}} to limit the scope of variables or formatting changes, which are automatically destroyed after use. Use a group to safely define macros in one subfile and use them in another, without polluting the global scope. For example:
    \begin{verbatim}
        \begingroup
            \input{\subfix{cat.tex}}   % defines macros (e.g., \myVar)
            \input{\subfix{apple.tex}} % uses \myVar
        \endgroup
    \end{verbatim}
    This ensures macros like \texttt{\textbackslash myVar} do not affect other parts of the document.

    
    \item \textbf{Create custom environments} with \texttt{NewDocumentEnvironment} to define consistent styles or formatting for repeated structures. 

    
    \item \textbf{Organize packages and settings} in a separate preamble file or style file (\texttt{.sty}) to avoid cluttering the main document.
    
    \item \textbf{Comment generously} and maintain a changelog for collaborative projects or long-term maintenance.
    
    \item \textbf{Use bibliographies modularly} with BibTeX or BibLaTeX in separate files for easier management of references.
    
    \item \textbf{Avoid hard-coded formatting} in the main document; define styles and colors globally using commands or custom environments.
    
    \item \textbf{Use consistent naming conventions} for labels, commands, and files to reduce confusion and improve readability.
    
    \item \textbf{Test small sections independently} before compiling the whole document to quickly catch errors.
\end{itemize}

\clearpage

\end{document}