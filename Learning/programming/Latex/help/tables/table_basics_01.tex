\documentclass[12pt,a4paper]{article}

% ====== PACKAGES ======
\usepackage[margin=1in]{geometry}
\usepackage[table]{xcolor}
\usepackage{booktabs}
\usepackage{tabularx}
\usepackage{multirow}
\usepackage{colortbl}
\usepackage{array}
\usepackage{caption}
\usepackage{diagbox}
\usepackage{makecell}
\usepackage{threeparttable}
\usepackage{float}
\usepackage{longtable}
\renewcommand{\arraystretch}{1.2}

\setlength{\tabcolsep}{8pt}

\begin{document}
	\section*{Comprehensive Table Gallery}
	
	% ==========================================================
	% 1. Basic bordered table
	% ==========================================================
	\begin{table}[H]
		\centering
		\caption{Basic Bordered Table}
		\begin{tabular}{|l|c|r|}
			\hline
			Item & Quantity & Price (\$) \\
			\hline
			Apple & 10 & 5.50 \\
			Banana & 7 & 3.25 \\
			Cherry & 3 & 2.10 \\
			\hline
		\end{tabular}
	\end{table}
	
	% ==========================================================
	% 2. Booktabs professional table
	% ==========================================================
	\begin{table}[H]
		\centering
		\caption{Booktabs Professional Table}
		\begin{tabular}{lcr}
			\toprule
			\textbf{Item} & \textbf{Quantity} & \textbf{Price (\$)} \\
			\midrule
			Apple & 10 & 5.50 \\
			Banana & 7 & 3.25 \\
			Cherry & 3 & 2.10 \\
			\bottomrule
		\end{tabular}
	\end{table}
	
	% ==========================================================
	% 3. Alternate Row Coloring
	% ==========================================================
	\rowcolors{2}{gray!10}{white}
	\begin{table}[H]
		\centering
		\caption{Alternate Row Coloring}
		\begin{tabular}{l c r}
			\toprule
			\textbf{Fruit} & \textbf{Stock} & \textbf{Cost (\$)} \\
			\midrule
			Apple & 50 & 25.00 \\
			Mango & 35 & 22.00 \\
			Banana & 80 & 30.00 \\
			Cherry & 25 & 18.00 \\
			\bottomrule
		\end{tabular}
	\end{table}
	
	% ==========================================================
	% 4. Header Color Band
	% ==========================================================
	\begin{table}[H]
		\centering
		\caption{Colored Header Band}
		\arrayrulecolor{gray!60}
		\rowcolors{2}{gray!05}{white}
		\begin{tabular}{>{\columncolor{gray!20}}l c c r}
			\toprule
			\rowcolor{blue!20}
			\textbf{Product} & \textbf{ID} & \textbf{Quantity} & \textbf{Price} \\
			\midrule
			Keyboard & K-101 & 40 & 1100 \\
			Mouse & M-201 & 55 & 600 \\
			Monitor & MN-301 & 20 & 7000 \\
			Laptop & L-401 & 10 & 65000 \\
			\bottomrule
		\end{tabular}
	\end{table}
	
	% ==========================================================
	% 5. Table with Multirow / Multicolumn
	% ==========================================================
	\begin{table}[H]
		\centering
		\caption{Multirow and Multicolumn Example}
		\begin{tabular}{|c|c|c|c|}
			\hline
			\multirow{2}{*}{\textbf{Year}} & \multicolumn{3}{c|}{\textbf{Sales (in \$)}} \\
			\cline{2-4}
			& Q1 & Q2 & Q3+Q4 \\
			\hline
			2023 & 1500 & 1700 & 4000 \\
			2024 & 1800 & 2100 & 4300 \\
			\hline
		\end{tabular}
	\end{table}
	
	% ==========================================================
	% 6. Table with Diagonal Cell (diagbox)
	% ==========================================================
	\begin{table}[H]
		\centering
		\caption{Diagonal Header Cell}
		\begin{tabular}{|c|c|c|}
			\hline
			\diagbox{Score}{Subject} & Math & Science \\
			\hline
			A & 12 & 9 \\
			B & 8 & 11 \\
			C & 4 & 7 \\
			\hline
		\end{tabular}
	\end{table}
	
	% ==========================================================
	% 7. Tabularx (Auto-width columns)
	% ==========================================================
	\begin{table}[H]
		\centering
		\caption{Auto Width Table with TabularX}
		\begin{tabularx}{0.9\textwidth}{l c X}
			\toprule
			\textbf{ID} & \textbf{Dept.} & \textbf{Description} \\
			\midrule
			1 & HR & Handles recruitment, payroll, and training operations \\
			2 & IT & Manages hardware, network, and software systems \\
			3 & Finance & Responsible for budgeting and audits \\
			\bottomrule
		\end{tabularx}
	\end{table}
	
	% ==========================================================
	% 8. Colored Block Table
	% ==========================================================
	\begin{table}[H]
		\centering
		\caption{Color-blocked Sections}
		\arrayrulecolor{white}
		\rowcolors{1}{gray!20}{gray!10}
		\begin{tabular}{p{3cm}p{3cm}p{3cm}}
			\rowcolor{blue!30}
			\textbf{Name} & \textbf{Department} & \textbf{Score} \\
			\rowcolor{blue!10}
			John & Sales & 92 \\
			Maria & Marketing & 89 \\
			Lee & R\&D & 95 \\
			\rowcolor{blue!10}
			Nina & HR & 88 \\
		\end{tabular}
	\end{table}
	
	% ==========================================================
	% 9. Table with Notes (threeparttable)
	% ==========================================================
	\begin{table}[H]
		\centering
		\caption{Table with Notes}
		\begin{threeparttable}
			\begin{tabular}{lcc}
				\toprule
				\textbf{Experiment} & \textbf{Trials} & \textbf{Accuracy (\%)} \\
				\midrule
				Test A & 50 & 89.4 \\
				Test B & 45 & 91.2 \\
				Test C & 60 & 94.7 \\
				\bottomrule
			\end{tabular}
			\begin{tablenotes}
				\footnotesize
				\item Note: Accuracy is averaged over three datasets.
			\end{tablenotes}
		\end{threeparttable}
	\end{table}
	
	% ==========================================================
	% 10. Table with custom column colors
	% ==========================================================
	\begin{table}[H]
		\centering
		\caption{Custom Column Colors}
		\arrayrulecolor{white}
		\begin{tabular}{
				>{\columncolor{gray!15}}l 
				>{\columncolor{gray!10}}c 
				>{\columncolor{gray!05}}r}
			\toprule
			\textbf{Country} & \textbf{Population (M)} & \textbf{GDP (B\$)} \\
			\midrule
			India & 1400 & 3300 \\
			USA & 335 & 26000 \\
			China & 1450 & 17600 \\
			Japan & 125 & 4200 \\
			\bottomrule
		\end{tabular}
	\end{table}
	
	% ==========================================================
	% 11. Table with Merged Cells & Color Bands
	% ==========================================================
	\begin{table}[H]
		\centering
		\caption{Merged Cells and Color Bands}
		\arrayrulecolor{gray!60}
		\begin{tabular}{|c|c|c|c|}
			\hline
			\rowcolor{gray!20}
			\multicolumn{2}{|c|}{\textbf{Group A}} & \multicolumn{2}{c|}{\textbf{Group B}} \\
			\hline
			X1 & X2 & Y1 & Y2 \\
			\hline
			12 & 15 & 19 & 22 \\
			14 & 18 & 21 & 25 \\
			\hline
		\end{tabular}
	\end{table}
	
	% ==========================================================
	% 12. Longtable Example (auto-splits across pages)
	% ==========================================================
	\begin{longtable}{l c c}
		\caption{Longtable Example (auto page split)}\\
		\toprule
		\textbf{Name} & \textbf{Age} & \textbf{City} \\
		\midrule
		\endfirsthead
		\toprule
		\textbf{Name} & \textbf{Age} & \textbf{City} \\
		\midrule
		\endhead
		Alice & 22 & New York \\
		Bob & 24 & Chicago \\
		Charlie & 28 & Boston \\
		David & 31 & San Francisco \\
		Edward & 27 & Austin \\
		Frank & 33 & Miami \\
		Grace & 26 & Seattle \\
		Hannah & 29 & Denver \\
		Ian & 30 & Dallas \\
		James & 25 & Portland \\
		Kate & 32 & Phoenix \\
		Leo & 34 & Detroit \\
		Mona & 23 & Atlanta \\
		Nick & 35 & Baltimore \\
		Olivia & 28 & Tampa \\
		Paul & 29 & Orlando \\
		Quinn & 26 & Raleigh \\
		Rita & 24 & Houston \\
		Steve & 27 & Nashville \\
		Tina & 25 & San Diego \\
		\bottomrule
	\end{longtable}
	
\end{document}
